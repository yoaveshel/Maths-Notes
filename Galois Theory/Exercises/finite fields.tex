\subsection{Finite Fields}
\begin{ex}{22.4}
    Let $f\in\F_q[x]$ ($q=p^n$) be monic irreducible polynomial of degree $d$. Then every zero $\alpha$ of $f$ in $\overline{\F_p}$ satisfies the equality
    $$f=\prod_{i=0}^{d-1} \left(x-\alpha^{q^i}\right)\in\overline{F_p}[x]$$
\end{ex}
\begin{proof}
    Let $f\in\F_q[x]$ be a monic irreducible polynomial of degree $d$ and $\alpha\in\overline{\F_p}$ a root of $f$.
    Let $F:\F_q\to\F_q$ be given by $F(x)=x^q$ which is an automorphism on $\F_q(\alpha)$. Then $F^d$ is the identity and so $\abs{F}\leq d$.
    Suppose $\abs{F}=k\mid d$. Then $a^{q^k}=a$ for all $a\in\F_q(\alpha)$, but $x^{q^k}-x$ has only $q^k$ roots while $\F_q(\alpha)$ has $q^d$ elements.
    Therefore the order of $F$ in $\text{Aut}(\F_q(\alpha))$ is $d$.

    Suppose $f(x)=\sum_{i=0}^d a_ix^i$. Then for any $\sigma\in\text{Aut}(\F_q(\alpha))$
    $$0=\sigma(0)=\sigma(f(\alpha))=\sum_{i=0}^d\sigma(a_i \alpha^i)=\sum_{i=0}^d a_i\sigma(\alpha)^i$$
    since $\sigma(a)=a^q=a$ for all $a\in\F_q$. Therefore $\sigma(\alpha)$ is a root of $f$ for any $\sigma\in\text{Aut}(\F_q(\alpha))$. 
    Since the map $\text{Aut}(\F_q(\alpha))\to\{\alpha\in\overline{\F_p}\mid f(\alpha)=0\}$ taking $\sigma$ to $\sigma(\alpha)$ is injective, it follows that $\abs{\text{Aut}(\F_q(\alpha))}\leq d$.
    Therefore $\text{Aut}(\F_q(\alpha))=d$ and so the group of automorphism on $\F_q(\alpha)$ is generated by $F$. Therefore it follows that
    $$f=\prod_{\sigma\in\text{Aut}(\F_q(\alpha))} \left(x-\sigma(\alpha)\right)=\prod_{i=0}^{d-1} \left(x-\alpha^{q^i}\right)\in\overline{F_p}[x]$$
    for any root $\alpha$ of $f$.
\end{proof}

\begin{ex}{22.6}
    Give an explicit isomorphism $\F_5[x]/(x^2+x+1)\xrightarrow\sim \F_5(\sqrt{2})$
\end{ex}
\begin{sol}
    Let $\varphi:\F_5[x]/(x^2+x+1)\xrightarrow \F_5(\sqrt{2})$ be an isomorphism and $\alpha$ the equivalence class of $x$ in $\F_5[x]/(x^2+x+1)$. 
    Since $\varphi$ is identity on $\F_5$, we only need to find where $\alpha$ is mapped to.
    Suppose
    $$\varphi(\alpha)=c+d\sqrt{2}.$$
    Then $\left(c+d\sqrt{2}\right)^2+c+d\sqrt{2}+1=0$. Hence 
    \begin{equation*}
        \begin{cases}
            d(2c+1)=0\\
            c^2+2d^2+c+1=0\\
        \end{cases}.
    \end{equation*}
     Since $d=0$ would be a contradiction it follows from the first equation that $c=2$.
     Substituting into the second we get $4+2d^2+3=2d^2+2$ and so $2d^2=3$. Therefore $d=2,3$ and either value will give us an isomorphism.
     So let
     $$\varphi(a+b\alpha)=a+b(2+2\sqrt{2})= 2a+2b\sqrt{2}.$$
\end{sol}

\begin{ex}{22.7}
    Show that $f(x)=x^2+2x+2$ and $g=x^2+x+3$ are irreducible in $\F_7[x]$ and give an explicit isomorphism
    $\F_7[x]/(f)\xrightarrow{\sim}\F_7[x]/(g)$.
\end{ex}
\begin{sol}
    Check that every element of $\F_7$ is not a root.

    We need to find $c+d\beta\in\F_7[x]/(g)$ such that
    $$(c+d\beta)^2+2(c+d\beta)+2=0.$$
    Using that $\beta^2=6\beta+4$ we get
    \begin{align*}
        0 &= c^2+2cd\beta+d^2\beta^2+2c+2d\beta+2\\
        &=(2cd+6d^2+2d)\beta+(c^2+4d^2+2c+2).
    \end{align*}
    which is equivalent to
    $$\begin{cases}
        d(2c+6d+2)=0\\
        c^2+4d^2+2c+2=0\\
    \end{cases}.$$
    Since $d\neq0$ it follows that $c=4d+6$. Hence $d^2=1$ and so $d=1,6$.
    So $c=3,2$ and either pair would give us an isomorphism. 
\end{sol}

\begin{ex}{22.8}
    Calculate the orders $1-\sqrt{2}, 2-\sqrt{2}$ and $3-\sqrt{2}$ in $\F_5(\sqrt{2})^2$.
\end{ex}
\begin{sol}
    Note that $(1-\sqrt{2})^3=2$ and the order of 2 is 4. Hence the order of $(1-\sqrt{2})$ divides 12.
    Since $(1-\sqrt{2})^2=3-2\sqrt{2}, (1-\sqrt{2})^3=2, (1-\sqrt{2})^4=4$ and $(1-\sqrt{2})^6=(1-\sqrt{2})^2(1-\sqrt{2})^4=8=3$
    it follows that the order of $1-\sqrt{2}$ in $\F_5(\sqrt{2})$ is 12.
\end{sol}

\begin{ex}{22.11}
    Let $p$ be a prime. Show that $\F_p(x)/(x^2+x+1)$ is a field if and only if $p\equiv 2\mod 3$.
\end{ex}
\begin{proof}
    $(\Rightarrow)$ Suppose $\F_p(x)/(x^2+x+1)$ is a field. So $f(x)=x^2+x+1$ is irreducible in $\F_p[x]$. 
    Therefore $\F_p$ does not contain a non-trivial cube root of unity and so 3 doesn't divide $\abs{\F_p^*}=p-1$. 
    Since $f(x)=(x+2)^2$ in $\F_3[x]$ it can't be the case that $p$ is congruent to $0\mod 3$ and so $p\equiv 2\mod 3$. 
    
    $(\Leftarrow)$ Suppose $\F_p(x)/(x^2+x+1)$ is not a field. Then $f(x)=x^2+x+1$ is reducible in $\F_p[x]$ and so $f$ has a root $\alpha\in\F_p$.
    Then $\alpha\neq 0$ and $\alpha^3=1$. Therefore 3 divides $\abs{\F_p^*}=p-1$ and so $p\equiv 1\mod 3$.

\end{proof}

\begin{ex}{22.12}
    Let $q$ be a prime power.
    \begin{enumerate}
        \item For what $q$ is the quadratic extension $\F_{q^2}$ of $\F_q$ of the form $\F_q(\sqrt{x})$ with $x\in\F_q$?
        \item For what $q$ is the cubic extension $\F_{q^3}$ of $\F_q$ of the form $\F_q(\sqrt[3]{x})$ with $x\in\F_q$?
    \end{enumerate}
\end{ex}
\begin{sol}
    ${}$
    \begin{enumerate}
        \item Let $\varphi:\F_q\to\F_q$ be given by $\varphi(a)=a^2$. If $q=p^n$ is even, then $p=2$ and $\varphi$ is a field isomorphism.
            Therefore $\F_q(\sqrt{b})=\F_q$ for all $x\in\F_q$. If $q$ is odd, then $(-1)^2=1=1^2$ so the map is not injective and so it's not surjective.
            Therefore there exists $b\in\F_q$ such that $\sqrt{b}\not\in\F_q$. Then $x^2-b$ is the minimal polynomial of $b$ and so $\F_q(\sqrt{b})$ is a quadratic extension.
            Hence $\F_q(\sqrt{b})=\F_{q^2}$.
        \item Let $\varphi:\F_q\to\F_q$ be given by $\varphi(a)=a^2$. 
            If $q\equiv 0\mod 3$ then $\varphi$ is a field isomorphism and so $\F_q(\sqrt[3]{b})=\F_q$ for all $b\in\F_q$.
            If $q\equiv 1\mod 3$ then $\abs{\F_q^*}\equiv 0\mod 3$. 
            Hence there exists an element $a\in\F_q$ of order three so $\varphi(a)=1=\varphi(1)$ and so $\varphi$ is not injective.
            It follows that there exists $b\in\F_q$ such that $\sqrt[3]{b}\not\in\F_q$. Then $F_q(\sqrt[3]{b})$ is a cubic extension and so it is equal to $F_{q^3}$.
            Lastly if $q\equiv 2\mod 3$ then $\abs{\F_q^*}\equiv 1\mod 3$ and so there is no element of order three in $\F_q$.
            Then for $a,b\in\F_q$ such that $\varphi(a)=\varphi(b)$ we have
            \begin{align*}
                1&=\varphi(aa^{-1})\\
                &=\left(aa^{-1}\right)^3\\
                &=a^3\left(a^{-1}\right)^{3}\\
                &=\left(b a^{-1}\right)^3
            \end{align*}
            and so $ba^{-1}=1$ since there is not element of order 3 in $\F_q$.
            Therefore $\varphi$ is injective hence surjective and so every element has a cube root. It follows that $\F_q(\sqrt[3]{b})=\F_q, \forall b\in\F_q$. 
    \end{enumerate} 
\end{sol}

\begin{ex}{22.13}
    Let $p$ be an odd prime. 
    \begin{enumerate}
        \item Show that $\F_{p^2}$ contains a primitive eighth root of unity $\zeta$ and that $\alpha=\zeta+\zeta^{-1}$ satisfies $\alpha^2=2$.
        \item Prove: $\alpha\in\F_p\iff p\equiv\pm 1\mod 8$. Conclude that 2 is a square modulo $p$ if and only if $p\equiv\pm 1\mod 8$.
    \end{enumerate}
\end{ex}
\begin{proof}
    ${}$
    \begin{enumerate}
        \item Since $p^2-1=(p-1)(p+1)$ is a product of two consecutive even numbers it follows that $8\mid p^2-1$ and so there exists an element $\zeta\in\F_{p^2}^*$ of order 8.
            Let $\alpha=\zeta+\zeta^{-1}$. Then $\alpha^2=\zeta^2+2+\zeta^{-2}$. Noting that $\zeta^4=-1$ we have
            \begin{align*}
                \left(\zeta^2+\zeta^{-2}\right)^2&=\zeta^4+2+\zeta^{-4}\\
                &=2\zeta^4+2&&(\text{Since }\zeta^{-4}=\zeta^4)\\
                &=0
            \end{align*}
            and so $\alpha^2=2$. 
        \item Suppose $p\equiv\pm 1\mod 8$. Then 
            \begin{align*}
                \alpha^p &= \zeta^p+\zeta^{-p}\\
                &=\zeta^{8k\pm 1}+\zeta^{-8k\mp 1}\\
                &=\zeta^{\pm 1}+\zeta^{\mp 1}\\
                &=\alpha
            \end{align*}
            and so $\alpha\in\F_p$ as $\F_p=\{a\in\F_{p^2}\mid a^p = a\}$. 
            This proves that $p\equiv\mod 8\implies 2$ is a square modulo $p$. 

            Conversely suppose that $p\not\equiv\pm 1\mod 8$. 
            Since $p$ can't be congruent to 0,2,4 or 6 modulo 8 it follows that $p\equiv 3\mod 8$ or $p\equiv 5\mod 8$.
            Equivalently, $p\equiv\pm 3\mod 8$. So we have
            \begin{align*}
                \alpha^p&=\zeta^p+\zeta^{-p}\\
                &=\zeta^{8k\pm 3}+\zeta^{-8k\mp 3}\\
                &=\zeta^{\pm 3}+\zeta^{\mp 3}\\
                &=\zeta^3+\zeta^{-3}\\
                &=\zeta^4\left(\zeta^{-1}+\zeta^{-7}\right)\\
                &=\zeta^4\left(\zeta^{-1}+\zeta^1\right)\\
                &=-\alpha.x
            \end{align*}
            So $\alpha\not\in\F_p$. 
            Since $\pm\alpha$ are the roots of 2 in $\F_{p^2}$, it follows that $2$ is not a square in $\F_p$.
    \end{enumerate}
\end{proof}

\begin{ex}{22.15}
    Determine all the primes for which $\F_p[x]/(x^4+1)$ is a field. 
\end{ex}
\begin{sol}
    If $p=2$ then $x^4+1=(x^2+1)^2=(x+1)^4$ so $p$ is odd. 
    Then $x^4+1$ divides $x^8-1$ in $\F_p[x]$. 
    Since $p^2-1=(p+1)(p-1)$ is a product of two consecutive even numbers it follows that $8\mid p^2-1$.
    Hence $x^8-1$ splits completely in $\F_{p^2}[x]$. So
    $$x^8-1=(x-1)(x-\beta)\cdots(x-\beta^7)=(x^4+1)(x^4-1)$$
    for some $\beta\in\F_{p^2}$. 
    If $x^4+1$ is irreducible in $\F_p[x]$, then for any root $\alpha$ of $x^4+1$, $\F_p(\alpha)$ is an extension of degree four.
    But $x^4+1$ splits completely in a quadratic extension and so it is reducible in $\F_p[x]$.

    \noindent\textbf{Alternative solution}
    
    If $-1$ is a square in $\F_p$ then $a^2=-1$ for some $a\in\F_p$. So
    $$x^4+1=x^4-a^2=(x^2+a)(x^2-a).$$
    If $2$ is a square in $\F_p$ then $b^2=2$ for some $b\in\F_p$ and so
    $$x^4+1=(x^2+1)^2-(bx)^2=(x^2+1+bx)(x^2+1-bx).$$
    Lastly, if neither $-1$ nor $2$ are squares in $\F_p$, then $p$ is odd (since $-1=1=1^2$ in $\Z/2\Z$).
    Then $\F_p^*=\{1,\alpha, \alpha^2,\dots,\alpha^{p-1}\}$ is cyclic subgroup of even order. 
    Since $-1$ and $2$ are odd powers of $\alpha$, it follows that their product $-2$ is an even power of $\alpha$ and so it is a square.
    So let $c\in\F_p$ such that $c^2=-2$. Then
    $$x^4+1=(x^2-1)-(cx)^2=(x^2-1-cx)(x^2-1+cx).$$
    Therefore $x^4+1$ is reducible modulo every prime.
\end{sol}

\begin{ex}{22.16}
    Prove: $f(x)=x^3+2$ is irreducible in $\F_{49}[x]$. Is $f$ irreducible over $\F_{7^n}$ for all even $n$?
\end{ex}
\begin{proof}
    Let $\alpha\in\overline{\F_7}$ be a root of $f$. Then $\alpha^3=-2$ and so $\alpha^{18}=1$.
    Since $\alpha^9=-1$ and $\alpha^6=4$ it follows that the order of $\alpha$ is 18. 
    Hence $\alpha\in\F_{7^m}\iff 18\mid 7^m-1$. Therefore $7^m\equiv 1\mod 18$ and by the Chinese Remainder Theorem and the fact that $7^m$ is always odd it follows that $7^m\equiv 1\mod 3$.
    Since the order of $7$ is 3 in $\Z/3\Z$ ($7^3=7\cdot 7^2=7\cdot 49=7\cdot 4=28=1$) it follows that $3\mid m$.
    Then $\alpha\in\F_{7^3}$. 
    Since the degrees of the extensions $\F_{7^2}/\F_7$ and $\F_{7^3}/\F_7$ are coprime it follows that $[\F_{7^2}(\alpha):\F_{7^2}]=3$.
    Hence $\deg f_{\F_{7^2}}^\alpha = 3$ and $f\mid f_{\F_{7^2}}^\alpha$.
    Since $f$ is a monic polynomial of degree 3 it follows that $f= f_{\F_{7^2}}^\alpha$ and so it is irreducible over $\F_{7^2}[x]$.

    Lastly, it is not true that $f$ is irreducible in $\F_{7^n}$ for all even $n$ since $F_{7^3}\subset\F_{7^6}$ and we already showed that $f$ has a root in $\F_{7^3}$.
\end{proof}

\begin{ex}{22.17}
    Prove: $f(x)=x^4+2$ is irreducible in $\F_{125}[x]$. Is $f$ irreducible over $\F_{5^n}$ for all $n$ odd?
\end{ex}
\begin{proof}
    Let $\alpha\in\overline{\F_5}$ be a root of $f$. Then $\alpha^4=-2$ and so $\alpha^16=1$. 
    Since $\alpha^{\frac{16}{2}}=\alpha^8=(-2)^2=4$ it follows that the order of $\alpha$ is 16.
    Then $f$ has a root in $\F_{5^m}\iff 16\mid 5^m-1$. 
    Thus $5^m\equiv 1\mod 16$ and so $4\mid m$ since 5 has order 4 in $\Z/16\Z$.
    Then $\alpha\in\F_{5^4}$ and so $\deg f_{\F_5}^\alpha = 4$ and $f_{\F_5}^\alpha\mid f$.
    But $f$ is a monic polynomial of degree 4 and so $f=f_{\F_5}^\alpha$. 
    
    Let $n$ be an odd number. Since $\F_5(\alpha)=\F_{5^4}$ is an extension of degree 4, and $\gcd(4,n)=1$ it follows that $[\F_{5^n}(\alpha):\F_{5^n}]=4$
    and so $\deg f_{\F_{5^n}}^\alpha = 4$ and $f_{\F_{5^n}}^\alpha\mid f$. Since $f$ is a monic polynomial of degree 4, $f=f_{\F_{5^n}}^\alpha$ and so $f$ is irreducible in $\F_{5^n}[x]$ for $n$ odd.
\end{proof}

\begin{ex}{22.19}
    Let $F=\F_{2^5}$.
    \begin{enumerate}
        \item Prove: for all $x\in F\setminus\F_2$, we have $F^*=\langle x\rangle$.
        \item For how many polynomials $f\in\F_2[x]$ do we have $\F_2[x]/(f)\simeq F$?
    \end{enumerate}
\end{ex}
\begin{sol}
    ${}$
    \begin{enumerate}
        \item Let $x\in F\setminus\F_2$. Then the order of $x$ divides $31=2^5-1$, i.e. the order of $x$ is 1 or 31.
        Since $x\neq 1\in\F_2$ it follows that $\abs{x}=31$ and so $\langle x\rangle=F^*$.
        \item Since  $\F_2[x]/(f)\simeq F$ if and only if $f$ is an irreducible polynomial of degree 5, we need to find the number of degree five irreducible polynomials in $\F_2$.
            If $x_d$ is the number of monic irreducible polynomials of degree $d$ in $\F_2[x]$ than
            $$32 = \sum_{d\mid 5} d\cdot x_d = 2+5 x_5.$$
            Hence the number of irreducible polynomials of degree $5$ in $\F_2[x]$ is $x_5=6$.
    \end{enumerate}
\end{sol}

\begin{ex}{22.24}
    Show that there exists $\frac{p^2+p}{2}$ monic polynomial of degree 2 in $\F_p[x]$ that are reducible.
    Conclude, $x_2=\frac{p^2-p}{2}$. Also determine $x_3$.
\end{ex}
\begin{sol}
    All the monic reducible polynomials of degree 2 have the form $(x-a)(x-b)$. If $a=b$ there are $p$ choices.
    If $a\neq b$ then there are ${p\choose 2}=\frac{p(p-1)}{2}$. Hence there are 
    $$\frac{p^2+p}{2}$$
    monic reducible polynomials of degree 2 in $\F_p[x]$. Since all monic polynomials of degree 2 have the form $x^2+ax+b$, there are $p^2$ such polynomials.
    Hence $x_2=p^2-\frac{p^2+p}{2}=\frac{p^2-p}{2}$.

    Similarly, the total number of degree 3 monic polynomials is $p^3$. 
    There are 4 types of reducible monic polynomials of degree 3:
    $$(x-a)^3, (x-a)^2(x-b), (x-a)(x-b)(x-c),\text{and } (x-a)(x^2+bx+c).$$
    There are ${p\choose 1}$ of the first type $2{p\choose 2}$ of the second (since twice $a$ and once $b$ is different to once $a$ and twice $b$), ${p\choose 3}$ of the third type and $p\frac{p^2-p}{2}$ of the fourth type ($\frac{p^2-p}{2}$ irreducible polynomials of degree 2 by the previous paragraph).
    Putting it all together we get
    $$x_3 = p^3- {p\choose 1}-2{p\choose 2}-{p\choose 3}-p\frac{p^2-p}{2}=\frac{p^3-p}{3}.$$
\end{sol}

\begin{ex}{22.30}
    Prove that $f(x)=x^p-x-a\in\F_p[x]$ is irreducible for all $a\in\F_p^*$. 
    How does the polynomial $x^q-x-a\in\F_q[x]$ decompose into irreducible factors for an arbitrary finite field $\F_q$?
\end{ex}
\begin{proof}
    Since for any $k\in\F_p$, $k^p-k=0$ and $a\neq 0$ by assumption, thus $f$ has not roots in $\F_p$.
    So let $\alpha\in\overline{\F_p}$ be a root of $f$.  Since $f'=-1$, $f$ has $p$ distinct roots.
    Observe that for any $k\in\F_p$ we have
    \begin{align*}
        f(\alpha+k)&=(\alpha+k)^p-(\alpha-k)-a\\
        &=\alpha^p+k^p-\alpha-k-a\\
        &=\alpha^p-\alpha-a&&(\text{Since }k^p=k)\\
        &=0
    \end{align*}
    and so all the zeros of $f$ are of the form $\alpha+k$ for $k\in\F_p$.
    Let $f=g_1\cdots g_n$ for $g_1,\dots,g_n$ irreducible over $\F_p[x]$. 
    Then each $g_i$ must be the minimal polynomial of at least one of the roots of $f$. Since 
    $$\F_p(\alpha)=\F_p(\alpha+k)$$
    for all $k\in\F_p$ it follows that the degree of all the $g_i$ must the same. Hence $p =\deg f = n\deg g$.
    We know that $\deg g\neq 1$ since $f$ has no roots in $\F_p$ and therefore $n=1$ so $f$ is irreducible. 

    Consider $f(x)=x^q-x-a\in\F_q[x]$ where $q=p^n$ and let $\alpha\in\overline{\F_p}$ be a root of $f$.
    Then $\alpha^q=\alpha+a$ and suppose $\alpha^{q^i}=\alpha+ia$. Then 
    \begin{align*}
        \alpha^{q^{i+1}}&=\left(\alpha^{q^i}\right)^q\\
        &=\left(\alpha+i a\right)^q&&(\text{induction hypothesis})\\
        &=\alpha^q+(i a)^q&&(\text{Frobenius})\\
        &=\alpha+(i+1)a&&(\text{since $\alpha$ is a root of $f$})
    \end{align*}
    which completes the induction. Since $\alpha^{q^p}=\alpha+pa=\alpha$ it follows that
    $$f_{\F_q}^\alpha(x)=\prod_{i=0}^{p-1}\left(x-\alpha^{q^i}\right)=\prod_{i=0}^{p-1}\left(x-\alpha-ia\right)\in\overline{\F_p}[x]$$
    is a degree $p$ polynomial.

\end{proof}

\begin{ex}{22.31}
    Let $K\subset L$ be an extension of finite fields and $G=\text{Aut}_K(L)$ the associated automorphism group.
    Prove: for $\alpha\in L$ with $L=K(\alpha)$ we have $f_K^\alpha = \prod_{\sigma\in G}(x-\sigma(\alpha))$. 
    What is the corresponding statement for an arbitrary $\alpha\in L$? 
\end{ex}
\begin{proof}
    Let $K$ be a finite field, $L=K(\alpha)$ and $f(x)=\sum_{i=0}^n a_i x^i$ the minimal polynomial of $\alpha$ in $K[x]$.
    Let $\sigma\in G$ be a field automorphism, then
    \begin{align*}
        f(\sigma(\alpha))&=\sum_{i=0}^n a_i \sigma(\alpha)^i\\
        &=\sum_{i=0}^n \sigma(a_i\alpha^i)&&(\text{since }\sigma(a)=a, \forall a\in K)\\
        &=\sigma\left(\sum_{i=0}^n a_i \alpha^i\right)\\
        &=\sigma(f(\alpha))\\
        &=0,
    \end{align*}
    and so $\sigma(\alpha)$ is a root of $f$.
\end{proof}

\begin{ex}{(extra)}
    Let $K$ be a field, $\overline{K}$ and algebraic closure and $f\in K[x]$ a non-constant polynomial.
    Then $f$ has no repeated roots in $\overline{K}$ if and only if $f$ is coprime to $f'$.
\end{ex}
\begin{proof}
    ($\Leftarrow$) Suppose $f$ has a repeated roots in $\overline{K}$. 
    Then 
    $$f(x)=(x-\alpha)^n g(x)\in\overline{K}[x]$$
    for $n\geq 2$.
    Taking the derivative we get
    $$f'(x)=n(x-\alpha)^{n-1}g(x)+(x-\alpha)^n g'(x)=(x-\alpha)^{n-1}\big(n g(x)+(x-\alpha)g'(x)\big)$$
    so $(x-\alpha)^{n-1}$ divides both $f$ and $f'$ so they are not coprime.
    
    ($\Rightarrow$) Conversely suppose that $f$ and $f'$ are not coprime. Let $\gcd(f,f')=(x-\alpha)d(x)$
    and let $\alpha\in\overline{K}$ be a root of $d$. Then $f(x)=(x-\alpha)h(x)$ and $f'(x)=(x-\alpha)h'(x)+h(x)$.
    Since $d(\alpha)=0$, $f'(\alpha)=0$ and so $h(\alpha)=0$. 
    Hence $h(x)=(x-\alpha)h_1(x)$ and $f(x)=(x-\alpha)^2 h_1(x)$. Hence $\alpha$ is a double root of $f$.
\end{proof}