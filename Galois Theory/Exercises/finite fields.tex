\subsection{Finite Fields}
\begin{ex}{22.6}
    Give an explicit isomorphism $\F_5[x]/(x^2+x+1)\xrightarrow\sim \F_5(\sqrt{2})$
\end{ex}
\begin{sol}
    Let $\varphi:\F_5[x]/(x^2+x+1)\xrightarrow \F_5(\sqrt{2})$ be an isomorphism and $\alpha$ the equivalence class of $x$ in $\F_5[x]/(x^2+x+1)$. 
    Since $\varphi$ is identity on $\F_5$, we only need to find where $\alpha$ is mapped to.
    Suppose
    $$\varphi(\alpha)=c+d\sqrt{2}.$$
    Then $\left(c+d\sqrt{2}\right)^2+c+d\sqrt{2}+1=0$. Hence 
    \begin{equation*}
        \begin{cases}
            d(2c+1)=0\\
            c^2+2d^2+c+1=0\\
        \end{cases}.
    \end{equation*}
     Since $d=0$ would be a contradiction it follows from the first equation that $c=2$.
     Substituting into the second we get $4+2d^2+3=2d^2+2$ and so $2d^2=3$. Therefore $d=2,3$ and either value will give us an isomorphism.
     So let
     $$\varphi(a+b\alpha)=a+b(2+2\sqrt{2})= 2a+2b\sqrt{2}.$$
\end{sol}

\begin{ex}{22.7}
    Show that $f(x)=x^2+2x+2$ and $g=x^2+x+3$ are irreducible in $\F_7[x]$ and give an explicit isomorphism
    $\F_7[x]/(f)\xrightarrow{\sim}\F_7[x]/(g)$.
\end{ex}
\begin{sol}
    Check that every element of $\F_7$ is not a root.

    We need to find $c+d\beta\in\F_7[x]/(g)$ such that
    $$(c+d\beta)^2+2(c+d\beta)+2=0.$$
    Using that $\beta^2=6\beta+4$ we get
    \begin{align*}
        0 &= c^2+2cd\beta+d^2\beta^2+2c+2d\beta+2\\
        &=(2cd+6d^2+2d)\beta+(c^2+4d^2+2c+2).
    \end{align*}
    which is equivalent to
    $$\begin{cases}
        d(2c+6d+2)=0\\
        c^2+4d^2+2c+2=0\\
    \end{cases}.$$
    Since $d\neq0$ it follows that $c=4d+6$. Hence $d^2=1$ and so $d=1,6$.
    So $c=3,2$ and either pair would give us an isomorphism. 
\end{sol}

\begin{ex}{22.8}
    Calculate the orders $1-\sqrt{2}, 2-\sqrt{2}$ and $3-\sqrt{2}$ in $\F_5(\sqrt{2})^2$.
\end{ex}
\begin{sol}
    Note that $(1-\sqrt{2})^3=2$ and the order of 2 is 4. Hence the order of $(1-\sqrt{2})$ divides 12.
    Since $(1-\sqrt{2})^2=3-2\sqrt{2}, (1-\sqrt{2})^3=2, (1-\sqrt{2})^4=4$ and $(1-\sqrt{2})^6=(1-\sqrt{2})^2(1-\sqrt{2})^4=8=3$
    it follows that the order of $1-\sqrt{2}$ in $\F_5(\sqrt{2})$ is 12.
\end{sol}

\begin{ex}{22.11}
    Let $p$ be a prime. Show that $\F_p(x)/(x^2+x+1)$ is a field if and only if $p\equiv 2\mod 3$.
\end{ex}
\begin{proof}
    $(\Rightarrow)$ Suppose $\F_p(x)/(x^2+x+1)$ is a field. So $f(x)=x^2+x+1$ is irreducible in $\F_p[x]$. 
    Therefore $\F_p$ does not contain a non-trivial cube root of unity and so 3 doesn't divide $\abs{\F_p^*}=p-1$. 
    Since $f(x)=(x+2)^2$ in $\F_3[x]$ it can't be the case that $p$ is congruent to $0\mod 3$ and so $p\equiv 2\mod 3$. 
    
    $(\Leftarrow)$ Suppose $\F_p(x)/(x^2+x+1)$ is not a field. Then $f(x)=x^2+x+1$ is reducible in $\F_p[x]$ and so $f$ has a root $\alpha\in\F_p$.
    Then $\alpha\neq 0$ and $\alpha^3=1$. Therefore 3 divides $\abs{\F_p^*}=p-1$ and so $p\equiv 1\mod 3$.

\end{proof}

\begin{ex}{22.12}
    Let $q$ be a prime power.
    \begin{enumerate}
        \item For what $q$ is the quadratic extension $\F_{q^2}$ of $\F_q$ of the form $\F_q(\sqrt{x})$ with $x\in\F_q$?
        \item For what $q$ is the cubic extension $\F_{q^3}$ of $\F_q$ of the form $\F_q(\sqrt[3]{x})$ with $x\in\F_q$?
    \end{enumerate}
\end{ex}
\begin{sol}
    ${}$
    \begin{enumerate}
        \item Let $\varphi:\F_q\to\F_q$ be given by $\varphi(a)=a^2$. If $q=p^n$ is even, then $p=2$ and $\varphi$ is a field isomorphism.
            Therefore $\F_q(\sqrt{b})=\F_q$ for all $x\in\F_q$. If $q$ is odd, then $(-1)^2=1=1^2$ so the map is not injective and so it's not surjective.
            Therefore there exists $b\in\F_q$ such that $\sqrt{b}\not\in\F_q$. Then $x^2-b$ is the minimal polynomial of $b$ and so $\F_q(\sqrt{b})$ is a quadratic extension.
            Hence $\F_q(\sqrt{b})=\F_{q^2}$.
        \item Let $\varphi:\F_q\to\F_q$ be given by $\varphi(a)=a^2$. 
            If $q\equiv 0\mod 3$ then $\varphi$ is a field isomorphism and so $\F_q(\sqrt[3]{b})=\F_q$ for all $b\in\F_q$.
            If $q\equiv 1\mod 3$ then $\abs{\F_q^*}\equiv 0\mod 3$. 
            Hence there exists an element $a\in\F_q$ of order three so $\varphi(a)=1=\varphi(1)$ and so $\varphi$ is not injective.
            It follows that there exists $b\in\F_q$ such that $\sqrt[3]{b}\not\in\F_q$. Then $F_q(\sqrt[3]{b})$ is a cubic extension and so it is equal to $F_{q^3}$.
            Lastly if $q\equiv 2\mod 3$ then $\abs{\F_q^*}\equiv 1\mod 3$ and so there is no element of order three in $\F_q$.
            Therefore $\varphi$ is injective hence surjective and so every element has a cube root. It follows that $\F_q(\sqrt[3]{b})=\F_q, \forall b\in\F_q$. 
    \end{enumerate} 
\end{sol}

\begin{ex}{22.13}
    Let $p$ be an odd prime. 
    \begin{enumerate}
        \item Show that $\F_{p^2}$ contains a primitive eighth root of unity $\zeta$ and that $\alpha=\zeta+\zeta^{-1}$ satisfies $\alpha^2=2$.
        \item Prove: $\alpha\in\F_p\iff p\equiv\pm 1\mod 8$. Conclude that 2 is a square modulo $p$ if and only if $p\equiv\pm 1\mod 8$.
    \end{enumerate}
\end{ex}
\begin{proof}
    ${}$
    \begin{enumerate}
        \item Since $p^2-1=(p-1)(p+1)$ is a product of two consecutive even numbers it follows that $8\mid p^2-1$ and so there exists an element $\zeta\in\F_{p^2}^*$ of order 8.
            Let $\alpha=\zeta+\zeta^{-1}$. Then $\alpha^2=\zeta^2+2+\zeta^{-2}$. Noting that $\zeta^4=-1$ we have
            \begin{align*}
                \left(\zeta^2+\zeta^{-2}\right)^2&=\zeta^4+2+\zeta^{-4}\\
                &=2\zeta^4+2&&(\text{Since }\zeta^{-4}=\zeta^4)\\
                &=0
            \end{align*}
            and so $\alpha^2=2$. 
        \item Suppose $p\equiv\pm 1\mod 8$. Then 
            \begin{align*}
                \alpha^p &= \zeta^p+\zeta^{-p}\\
                &=\zeta^{8k\pm 1}+\zeta^{-8k\mp 1}\\
                &=\zeta^{\pm 1}+\zeta^{\mp 1}\\
                &=\alpha
            \end{align*}
            and so $\alpha\in\F_p$ as $\F_p=\{a\in\F_{p^2}\mid a^p = a\}$. 
            This proves that $p\equiv\mod 8\implies 2$ is a a square modulo $p$. 

            Conversely suppose that $p\not\equiv\pm 1\mod 8$. 
            Since $p$ can't be congruent to 0,2,4 or 6 modulo 8 it follows that $p\equiv 3\mod 8$ or $p\equiv 5\mod 8$.
            Equivalently, $p\equiv\pm 3\mod 8$. So we have
            \begin{align*}
                \alpha^p&=\zeta^p+\zeta^{-p}\\
                &=\zeta^{8k\pm 3}+\zeta^{-8k\mp 3}\\
                &=\zeta^{\pm 3}+\zeta^{\mp 3}\\
                &=\zeta^3+\zeta^{-3}\\
                &=\zeta^4\left(\zeta^{-1}+\zeta^{-7}\right)\\
                &=\zeta^4\left(\zeta^{-1}+\zeta^1\right)\\
                &=-\alpha.x
            \end{align*}
            So $\alpha\not\in\F_p$. 
            Since $\pm\alpha$ are the roots of 2 in $\F_{p^2}$, it follows that $2$ is not a square in $\F_p$.
    \end{enumerate}
\end{proof}

\begin{ex}{22.15}
    Determine all the primes for which $\F_p[x]/(x^4+1)$ is a field. 
\end{ex}
\begin{sol}
    If $p=2$ then $x^4+1=(x^2+1)^2=(x+1)^4$ so $p$ is odd. 
    Then $x^4+1$ divides $x^8-1$ in $\F_p[x]$. 
    Since $p^2-1=(p+1)(p-1)$ is a product of two consecutive even numbers it follows that $8\mid p^2-1$.
    Hence $x^8-1$ splits completely in $\F_{p^2}[x]$. So
    $$x^8-1=(x-1)(x-\beta)\cdots(x-\beta^7)=(x^4+1)(x^4-1)$$
    for some $\beta\in\F_{p^2}$. 
    If $x^4+1$ is irreducible in $\F_p[x]$, then for any root $\alpha$ of $x^4+1$, $\F_p(\alpha)$ is an extension of degree four.
    But $x^4+1$ splits completely in a quadratic extension and so it is reducible in $\F_p[x]$.

    \noindent\textbf{Alternative solution}
    
    If $-1$ is a square in $\F_p$ then $a^2=-1$ for some $a\in\F_p$. So
    $$x^4+1=x^4-a^2=(x^2+a)(x^2-a).$$
    If $2$ is a square in $\F_p$ then $b^2=2$ for some $b\in\F_p$ and so
    $$x^4+1=(x^2+1)^2-(bx)^2=(x^2+1+bx)(x^2+1-bx).$$
    Lastly, if neither $-1$ nor $2$ are squares in $\F_p$, then $p$ is odd (since $-1=1=1^2$ in $\Z/2\Z$).
    Then $\F_p^*=\{1,\alpha, \alpha^2,\dots,\alpha^{p-1}\}$ is cyclic subgroup of even order. 
    Since $-1$ and $2$ are odd powers of $\alpha$, it follows that their product $-2$ is an even power of $\alpha$ and so it is a square.
    So let $c\in\F_p$ such that $c^2=-2$. Then
    $$x^4+1=(x^2-1)-(cx)^2=(x^2-1-cx)(x^2-1+cx).$$
    Therefore $x^4+1$ is reducible modulo every prime.
\end{sol}