\documentclass{article}

\usepackage[utf8]{inputenc}
\usepackage{csquotes}
\usepackage[english]{babel}
\usepackage{amsmath,amssymb,amsthm,textcomp}
\usepackage{mathtools}
\usepackage{biblatex}
\usepackage{tikz}
\usepackage{graphics, setspace}
\usepackage{listings}
\usepackage{lipsum}
\usepackage{bookmark}
\usepackage{hyperref}
\hypersetup{
    colorlinks,
    citecolor=black,
    filecolor=black,
    linkcolor=black,
    urlcolor=black
}

\DeclareMathAlphabet{\pazocal}{OMS}{zplm}{m}{n}
\DeclareMathOperator{\Ima}{Im}
\newcommand{\Ba}{\mathcal{B}}
\newcommand{\Ta}{\mathcal{T}}
\newcommand{\Aa}{\mathcal{A}}
\newcommand{\R}{\mathbb{R}}
\newcommand{\Cc}{\mathbb{C}}
\newcommand{\Z}{\mathbb{Z}}
\newcommand{\N}{\mathbb{N}}
\newcommand{\Q}{\mathbb{Q}}
\newcommand{\p}{\mathbb{P}}
\newcommand{\Ss}{\mathcal{S}} % Schwartz space
\newcommand{\Rf}{\mathcal{R}} % reflection
\newcommand{\E}{\mathbb{E}}
\newcommand{\F}{\mathbb{F}}
\newcommand{\al}{\alpha}
\newcommand{\be}{\beta}
\newcommand{\de}{\delta}
\newcommand{\e}{\varepsilon}
\newcommand{\aand}{\quad\text{and}\quad}

\DeclarePairedDelimiter\abs{\lvert}{\rvert}%
\DeclarePairedDelimiter\norm{\lVert}{\rVert}%
% Swap the definition of \abs* and \norm*, so that \abs
% and \norm resizes the size of the brackets, and the 
% starred version does not.
\makeatletter
\let\oldabs\abs
\def\abs{\@ifstar{\oldabs}{\oldabs*}}
%
\let\oldnorm\norm
\def\norm{\@ifstar{\oldnorm}{\oldnorm*}}
\makeatother

\newtheorem{theorem}{Theorem}[section]
\newtheorem{corollary}{Corollary}[theorem]
\newtheorem{lemma}[theorem]{Lemma}
\newtheorem*{definition}{Definition}
\newtheorem*{remark}{Remark}

\theoremstyle{remark}
\newtheorem*{sol}{Solution}

\newenvironment{ex}[1]
    {\noindent\textbf{Exercise #1}\normalsize\newline}
    {\vspace{0.5 em}}

\newsavebox{\selvestebox}
\newenvironment{colbox}
    {\newcommand\colboxcolor{B6D0DE}%
    \begin{lrbox}{\selvestebox}%
    \begin{minipage}{\dimexpr\columnwidth-2\fboxsep\relax}}
    {\end{minipage}\end{lrbox}%
    \begin{center}
    \colorbox[HTML]{\colboxcolor}{\usebox{\selvestebox}}
    \end{center}}

\setcounter{secnumdepth}{1}

\title{Galois Theory - 5122GALO6Y}
\author{Yoav Eshel}
\date{\today}

\begin{document}
    \maketitle
    \tableofcontents
    \newpage
    \section{Summary}
    % write down summary of important definitions/theorems as well as useful techniques to solving exercises
    \section{Introduction}
    Galois theory is about studying Polynomials with coefficients in a field ($\Q,\R,\Cc$ etc.).
    Let
    $$
        f(T) = T^n+\cdots+a_1T+a_0\in\Q[T].
    $$
    Then $f(T)$ splits completely in $\Cc[T]$ as
    $$
        f(T)=(T-\alpha_1)\cdots(T-\alpha_n)
    $$
    with $\alpha_1,\dots \alpha_n\in\Cc$ are the roots of $f$. Galois theory studies permutation of the the roots that preserve algebraic relations between these roots.
    The allowed permutation of the roots give rise to a group denoted $\text{Gal}(f)$.
    The following definition of a Galois group does not require any background knowledge but is not very useful in practice.

    \begin{definition}
        Let $\sigma:\Cc\to\Cc$ be a field automorphism and $\alpha\in\Cc$ a root of $F(T)\in\Q[T]$. Since $\sigma(1)=1$ it follows that $\sigma(n)=n$ for all integers and so $\sigma(a/b)=\sigma(a)/\sigma(b)=a/b$ is the identity on $\Q$.
        Then
        \begin{align*}
            f(\sigma(\alpha))&=\sigma(\alpha)^n+\cdots+a_1\sigma(\alpha)+a_0\\
            &=\sigma(f(\alpha))\\
            &=0.
        \end{align*}
        Then each automorphism $\sigma$ is a permutation of the roots which is precisely the Galois group of the polynomial $\text{Gal}(f)\subset S_n$.
        In other words we have a group action 
        $$
        \text{Aut}(\Cc)\times\{\alpha_1,\dots,\alpha_n\}\to\{\alpha_1,\dots,\alpha_n\}
        $$
        Then $\text{Gal}(f):=\text{Im}(\phi)$ where $\phi:\text{Aut}(\Cc)\to S_n$ mapping $\sigma\mapsto(\alpha_i\mapsto \sigma(\alpha_i))$
    \end{definition}

    $\text{Gal}(f)\subset S_n$ is transitive subgroup (i.e. if its action on the set of roots is transitive) if and only if $f$ is irreducible.
    
    \section{Symmetric Polynomials}
    A symmetric polynomial is a polynomial $F(X_1,X_2,\dots, X_n)$ the is invariant under permutations of its variables.
    In other words
    $$ P(X_1, X_2,\dots, X_n)=P(X_{\sigma(1)}, X_{\sigma(2)},\dots,X_{\sigma(n)})$$
    for all $\sigma\in S_n$.
    Symmetric polynomials arise naturally in the study of the relation between the roots of a polynomial in one variable and its coefficients, 
    since the coefficients can be given by polynomial expressions in the roots, and all roots play a similar role in this setting.
    Let $f\in K(T)$ be a monic polynomial of degree $n$ that splits completely in $K$.
    Then
    $$ f(T)=(T-X_1)(T-X_2)\cdots(T-X_n)$$
    where $X_i$ are the roots of $f$. Then
    $$ f(T)=T^n+s_1 T^{n-1}+\cdots+(-1)^n s_n$$
    where
    \begin{align*}
        s_1&=X_1+X_2+\cdots+X_n\\
        s_2&=X_1X_2+X_1X_3+\cdots+X_{n-1}X_n\\
        \vdots&\\
        s_n&=X_1X_2\cdots X_n
    \end{align*}
    are called the \textit{elementary symmetric polynomials} in $X_1,X_2,\dot, X_n$.
    Then the fundamental theorem of symmetric polynomials states that every symmetric polynomial can be written as a polynomial expression in the elementary symmetric polynomials.

    To actually write a symmetric polynomial in terms of elementary symmetric polynomials we introduce some useful notation.
    We say a polynomial is ordered \textit{lexicographically} if the monomial $T_1^{e_1}T_2^{e_2}\cdots T_n^{e_n}$ with the highest $e_1$ is in front.
    If two monomials have the same $e_1$, then we compare their $e_2$ and so on. Like a dictionary.
    If $P$ is a symmetric polynomial in $n$ variables, choose a single representative proceeded by the symbol $\sum_n$ to denote the sum over the monomials in the $S_n$ orbit of the representative.
    Then for example
    \begin{align*}
        s_1&=\sum_n T_1\\
        s_2&=\sum_n T_1T_2\\
        \vdots&\\
        s_n&=\sum_n T_1T_2\cdots T_n=T_1T_2\cdots T_n.
    \end{align*}
    Now suppose $P$ is a symmetric polynomial. To find its representation in terms of symmetric polynomials:
    \begin{enumerate}
        \item Let $a\cdot T_1^{e_1}T_2^{e_2}\cdots T_n^{e_n}$ be the the first term in $P$, lexicographically.
        \item Form the monomial $$M = s_1^{e_1-e_2}s_2^{e_2-e_1}\cdots s_{n-1}^{e_{n-1}-e_n}s_n^{e_n}$$
        \item Let $P_i = P -cM$.
        \item Repeat steps (1)-(3) until $\deg P_i = 0$.
        \item The we can solve for $P$ and write it as a polynomial in the elementary symmetric polynomials.
    \end{enumerate}
    The representation obtained through the algorithm above is unique.

    The following theorem is useful when applying the algorithm above.
    \begin{theorem}[Orbit Stabilizer Theorem]\label{th:orbit_stabilizer}
        Let $G$ be a group acting on set $S$. For any $x\in S$ let $G_x=\{g\in G\mid g\cdot x= x\}$ denote the stabilizer of $x$, and let $G\cdot x=\{g\cdot x\mid g\in G\}$ denote the orbit of $x$. Then
        $$
            \abs{G}=\abs{G\cdot x}\abs{G_x}
        $$
    \end{theorem}
    Wondering how it might be useful? Consider 
    $$s_1^4=\left(\sum_n T_1\right)^4=(T_1+\cdots+T_n)(T_1+\cdots+T_n)(T_1+\cdots+T_n)(T_1+\cdots+T_n).$$ 
    After some thinking you might conclude that there are five possible representatives:
    $$T_1^4,\quad T_1^3T_2,\quad T_1^2T_2^2,\quad T_1^2T_2T_3\quad\text{and}\quad T_1T_2T_3T_4$$
    (note the the degrees always add up to four). But what are the coefficients? That's when the orbit-stabilizer theorem comes to the rescue.
    Let the permutation group $S_4$ act on the set of indices by permuting them. Then the coefficients in front of $\sum_n T_1^4$ is the size of the orbit of $(1,1,1,1)$.
    Since every permutation in $S_4$ return the same sequence, the size of the orbit is $\frac{4!}{4!}=1$. 
    Then the coefficients in front of $\sum_n T_1^2T_2^2$ is the size of the orbit of $(1,1,2,2)$.
    The permutations that fix it are $(1), (12), (34)$ and $(12)(34)$. So the size of the stabilizer is 4 and the size of the orbit is $\frac{4!}{4}=6$
    Similarly, the coefficients in front of $\sum_n T_1^2T_2T_3$ is the size of the orbit of $(1,1,2,3)$.
    Since the stabilizer contains only the permutations that switches the 1s and fixes the other two elements (namely $(1)$ and $(12)$) the size of the orbit is $ \frac{4!}{2}=12$.
    Lastly, the size of the orbit of $\sum_n T_1T_2T_3T_4$ is the $4!$ since there is no permutations (except the identity of course) that stabilizes it. We conclude that
    $$s_1^4=\sum_n T_1^4+6\sum_n T_1^2T_2^2+12\sum_n T_1^2T_2T_3+24\sum_n T_1T_2T_3T_4$$


    
    \section{Field Extensions}
    

    \begin{colbox}
        \begin{theorem}
            Let $K\subset L\subset M$ be a tower of fields, $X$ a $K$-basis for $L$ and $Y$ an $L$-basis for $M$.
            Then the set of elements $xy$ with $x\in X$ and $y\in Y$ forms a $K$-basis for $M$ and we have
            $$[M:K]=[M:L][L:K].$$
            In particular, $K\subset M$ is finite if and only if $K\subset L$ and $L\subset M$ are finite
        \end{theorem}
    \end{colbox}
    
    \begin{proof}
        Let $c\in M$. Then $c$ can be written uniquely as $c=\sum_{y\in Y} b_y\cdot y$ with coefficients $b_y\in L$ that are almost all 0.
        The elements $b_y\in L$ each have a unique representation as $b_y=\sum_{x\in X}a_{xy}\cdot x$ with coefficients $a_{xy}\in K$ that are almost all 0.
        Hence
        $$c=\sum_{y\in Y}\sum_{x\in X}a_{xy}\cdot x\cdot y$$
        and so $c$ is a finite $K$-linear combination of the elements $xy$. Therefore $(x,y)\in X\times Y$ forms a basis for $M$ over $K$.

        Since $\abs{X\times Y}=\abs{X}\cdot\abs{Y}$ it follows that
        $$[M:K]=[M:L]\cdot [L:K].$$
        Moreover, it follows that $X\times Y$ is finite if and only if $X$ and $Y$ are finite, since $X$ and $Y$ are non-empty.
    \end{proof}

    \subsection{Prime Fields}
    \begin{definition}
        Let $K$ be a field. Then the \textbf{prime field} in $K$ is the intersection over all subfields of $K$
    \end{definition}
    \begin{lemma}
        Let $K$ be a field of characteristic $k$. Then the prime field of $K$ is $\mathbb{F}_p$ if $k=p$ and $\Q$ if $k=0$.
    \end{lemma}
    \subsection{Algebraic and Transcendental Extensions}
    Let $L/K$ be a field extensions. Then we say that $\alpha\in L $ is \textit{algebraic} over $K$ if there exists an $f\in K[x], f\neq 0$, such that $f(\alpha)=0$.
    We say that $\alpha$ is \textit{transcendental} over $K$ if there exists no such $f$.
    The number of algebraic elements over $\Q$ in $\Cc$ is countable, so in fact $\Cc$ is mostly transcendental elements.

    \begin{colbox}
        \begin{theorem}
            Let $L/K$ be a field extension and take $\alpha\in L$. Then
            \begin{enumerate}
                \item If $\alpha$ is transcendental over k, then $K[\alpha]\simeq K[X]$ and $K(\alpha)\simeq K(x)$.
                \item If $\alpha$ is algebraic over $K$ then there exists a unique monic irreducible polynomial $f\in K[x]$ that has $\alpha$ has a zero.
                    In this case there is a field isomorphism
                    \begin{align*}
                        K[X]/(f)&\simeq K[\alpha]=K(\alpha)\\
                        g\mod (f)\mapsto g(\alpha)
                    \end{align*}
                    and the degree of $K(\alpha)$ over $K$ is the degree of $f$.
            \end{enumerate}
        \end{theorem}
    \end{colbox}

    \begin{definition}
        We say that an extension $L/K$ is \textbf{algebraic} if $\forall \alpha\in L$, $\alpha$ is algebraic over $K$.
    \end{definition}
    \begin{lemma}
        If a field extension is finite then it is algebraic.
    \end{lemma}
    The converse of this lemma does not hold.

    \newpage
    \section{Exercises}
    
\subsection{Symmetric Polynomial}
    \begin{ex}{14.10}
        Express the symmetric polynomials $\sum_n T_1^2T_2$ and $\sum_{n} T_1^3T_2$ in the elementary symmetric polynomials.
    \end{ex}
    \begin{sol}
        To get the polynomial $\sum_n T_1^2T_2$ we start with
        $$
            s_1s_2=\sum_n T_1\sum_n T_1T_2 = \sum_n T_1^2T_2+3\sum_n T_1T_2T_3 = \sum_n T_1^2T_2+3s_3
        $$
        Thus 
        $$
            \sum_n T_1^2T_2 = s_1s_2-3s_3
        $$

        Similarly, to transform the polynomial $\sum_{n} T_1^3T_2$ we start with
        \begin{align*}
            s_1^2s_2&=\left(\sum_nT_1\right)^2\sum_nT_1T_2\\
            &=\left(\sum_n T_1^2+2\sum_n T_1T_2\right)\sum_nT_1T_2\\
            &=\sum_nT_1^2\sum_n T_1T_2+2s_2^2\\
            &=\sum_nT_1^3T_2+\sum_n T_1^2T_2T_3+2s_2^2.
        \end{align*}
        And since
        $$
            s_1s_3=\sum_nT_1\sum_nT_1T_2T_3=\sum_nT_1^2T_2T_3+4\sum_n T_1T_2T_3T_4
        $$
        it follows that $\sum_n T_1^2T_2T_3=s_1s_3-4s_4$ and so
        $$
            \sum_{n} T_1^3T_2=s_1^2s_2-s_1s_3+4s_4-2s_2^2
        $$
    \end{sol}

    \begin{ex}{14.14}
        Prove: For $n\in\Z_{>0}$, we have $\Delta(X^n+a)=(-1)^{\frac12n(n-1)}n^na^{n-1}$.
    \end{ex}
    \begin{proof}
        Let $f(X)=X^n+a$ and let $\alpha_i$ be its roots. Then $f'(X)=nX^{n-1}$ and
        $$
            \Delta(f)=(-1)^{n(n-1)/2}R(f,f').
        $$
        Let $f_1(X)=a$ and then $f\equiv f_1\mod(f')$ since $f = f_1+f'\cdot\left(\frac1n X\right)$.
        Simplifying the resultant we get
        \begin{align*}
            R(f,f')&=R(f',f)&&(\text{Property }1)\\
            &=n^{n}R(f',f_1)&&(\text{Property }3)\\
            &=n^{n}\cdot\left(n^0\prod_{i=1}^{n-1}f_1(\alpha_i)\right)&&(\text{Property }2)\\
            &=n^n a^{n-1}
        \end{align*}
        and the result follows.
    \end{proof}

    \begin{ex}{14.15}
        Calculate the discriminant of the polynomial $f(X)=X^4+pX+q\in\Q(p,q)[X]$.
    \end{ex}
    \begin{sol}
        Then $f'(X)=4X^3+p$ and so 
        $$f_1(X)=f-f'\cdot h = X^4+pX+q+(4X^3+p)(\frac14 X) = \frac{3p}{4}X+q.$$
        Then the resultant is
        \begin{align*}
            R(f,f')&=R(f',f)&&(\text{Property } 1)\\
            &=4^{4-1}R(f', f_1)&&(\text{Property } 3)\\
            &=4^3\left((-1)^{3\cdot 1}R(f_1,f')\right)&&(\text{Property } 1)\\
            &=-4^3\left(\left(\frac{3p}{4}\right)^3\prod_{i=1}^{1}f'\left(\frac{-4q}{3p}\right)\right)&&(\text{Property } 2)\\
            &=-3^3p^3\left(4\left(\frac{-4q}{3p}\right)^3+p\right)\\
            &=4^4q^3-3^3p^4.
        \end{align*}
        Therefore the discriminant of $f$ is
        $$
            \Delta(f) = (-1)^{4\cdot 3/2}R(f,f') = R(f, f') = 4^4q^3-3^3p^4.
        $$
    \end{sol}

    \begin{ex}{14.16}
        For every $n>1$, determine an expression for the discriminant of the polynomial $f(X) = X^n+pX+q\in\Q(p,q)[X]$.
    \end{ex}
    \begin{sol}
        Let $f(X)=X^n+pX+q\in\Q(p,q)[X]$ for $n>1$. 
        Then $f'(X)=nX^{n-1}+p$ and $f\equiv f_1\mod(f')$ where
        $$f_1 = f-f'\cdot h = X^n+pX+q-\left(nX^{n-1}+p\right)\left(\frac1n X\right)=\frac{p(n-1)}{n}X+q.$$
        The resultant of $f$ and $f'$ is given by
        \begin{align*}
            R(f,f') &= R(f', f)&&(\text{Property } 1)\\
            &=n^{n-1}R(f', f_1)&&(\text{Property } 3)\\
            &=n^{n-1}\left((-1)^{n-1}R(f_1, f')\right)&&(\text{Property } 1)\\
            &=(-n)^{n-1}\left(\frac{p(n-1)}{n}\right)^{n-1}\prod_{i=1}^1 f'\left(-\frac{nq}{(n-1)p}\right)&&(\text{Property } 2)\\
            &=(-1)^{n-1}p^{n-1}(n-1)^{n-1}\left(\frac{(-1)^{n-1}n^nq^{n-1}}{(n-1)^{n-1}p^{n-1}}+p\right)\\
            &=n^nq^{n-1}+(-1)^{n-1}p^n(n-1)^{n-1}.
        \end{align*}
        Hence the discriminant of $f$ is
        $$
            \Delta(f)=(-1)^{n(n-1)/2}R(f,f')=(-1)^{n(n-1)/2}\left(n^nq^{n-1}+(-1)^{n-1}p^n(n-1)^{n-1}\right)
        $$
    \end{sol}

    \begin{ex}{14.17}
        Let $f\in\Z[X]$ be a monic polynomial. Prove that the following are equivalent
        \begin{enumerate}
            \item $\Delta(f)\neq 0$.
            \item The polynomial $f$ has no double zeroes in $\C$.
            \item The decomposition of $f$ in $\Q[X]$ has no multiple prime factors.
            \item The polynomial $f$ and its derivative $f'$ are relatively prime in $\Q[X]$.
            \item The polynomial $f\mod p $ and $f' \mod p$ are relatively prime in $\mathbb{F}_p[X]$ for almost all prime numbers $p$.
        \end{enumerate}
    \end{ex}
    \begin{proof}
        Let  $f\in\Z[X]$ be monic and $\{\alpha_1,\alpha_2,\dots,\alpha_n\}$ it roots in $\C$.

        $(1)\Rightarrow  (2)$. Suppose that $\alpha_i=\alpha_j$ for some $i\neq j$. Then 
        $$\Delta(f)=\prod_{1\leq i<j\leq n}(\alpha_i-\alpha_j)= 0,$$ 
        which is a contradiction.
        Therefore if $f$ has non-zero discriminant it has no double zeroes in $\C$. 

        $(2)\Rightarrow (3)$.
        
        $(3)\Rightarrow (4)$.

        $(4)\Rightarrow (5)$. If $f$ and $f'$ are relatively prime in $\Q[X]$ then 

        $(1)\Rightarrow (1)$. 
    \end{proof}

    \begin{ex}{14.19}
        Let $f\in\Q[X]$ be a monic polynomial with $n=\deg(f)$ distinct complex roots. Prove: the sign of $\Delta(f)$ is equal to $(-1)^s$ where $2s$ is the number of non-real zeroes of $f$.
    \end{ex}
    \begin{proof}
        Let $\{\alpha_1,\dots,\alpha_{n}\}$ be all the roots of $f$.
        Then each term $(\alpha_i-\alpha_j)^2$ in the discriminant falls into one of 3 cases
        \begin{enumerate}
            \item Both $\alpha_i$ and $\alpha_j$ are non-real. Then
            \begin{enumerate}
                \item If $\alpha_j=\overline{\alpha_i}$ then $\alpha_i-\alpha_j$ is purely complex and $(\alpha_i-\alpha_j)^2$ is negative.
                \item If $\alpha_j\neq\overline{\alpha_i}$ then $\overline{\alpha_i}$ and $\overline{\alpha_j}$ are also roots of $f$ and the term
                $$(\alpha_i-\alpha_j)^2(\overline{\alpha_i}-\overline{\alpha_j})^2=\left((\overline{\alpha_i-\alpha_j})(\alpha_i-\alpha_j)\right)^2=\abs{\alpha_i-\alpha_j}^2 $$
                is positive.
            \end{enumerate}
            \item $\alpha_i$ is non-real and $\alpha_j$ is real. Then $\overline{\alpha_i}$ is a root of $f$ and the term
            $$(\alpha_i-\alpha_j)^2(\overline{\alpha_i}-\alpha_j)^2=\abs{\alpha_i-\alpha_j}^2 $$
            is positive.
            \item Both $\alpha_i$ and $\alpha_j$ are real. Then $(\alpha_i-\alpha_j)^2$ is positive.
        \end{enumerate}
        Since the only negative terms are of the form $(\alpha_i-\overline{\alpha_i})^2$ and there are $2s$ non-real roots the sign of the determinant is $(-1)^s$.

    \end{proof}

    \begin{ex}{14.20}
        Prove: $f(X)=X^3+pX+q\in\R[X]$ has three (counted with multiplicity) real zeroes $\iff$ $4p^3+27q^\leq 0$.
    \end{ex}
    \begin{proof}
        By Ex. 16 we know that $\Delta(f)=(-1)^3\left(3^3q^2+2^2p^3\right)=-27q^2-4p^3$. 
        Let $a,b$ and $c$ be the roots of $f$. If $a,b,c\in\R$ then 
        $$
        -27q^2-4p^3=\Delta(f)=(a-b)^2(a-c)^2(b-c)^2\geq 0
        $$
        and so $4p^3+27q^\leq 0$.

        Now suppose that $a=x+yi$ and $b=x-yi$ are complex conjugates and $c$ is real. Then 
        \begin{align*}
            -27q^2-4p^3&=\Delta(f)\\
            &=(a-b)^2(a-c)^2(b-c)^2\\
            &=-4y^2\left((a-c)(\overline{a-c})\right)^2\\
            &=-4y^2\abs{a-c}^2\\
            &\leq 0.
        \end{align*}
        Hence $4p^3+27q^\geq 0$ and the result follows by contraposition.
    \end{proof}
        
    \begin{ex}{14.21}
        Express $p_4=\sum_nT_1^4$ in elementary symmetric polynomials
    \end{ex}
    \begin{sol}
        Let $n\geq 4$. Starting with
        \begin{align*}
            s_1^4 &= \left(\sum_nT_1\right)^4\\& = \sum_n T_1^4+4\sum_n T_1^3T_2+12\sum_n T_1^2T_2T_3+6\sum_nT_1^2T_2^2+24\sum_nT_1T_2T_3T_4.
        \end{align*}
        To understand how to coefficients of the sum are obtained, consider the number of ways the $T_i$ can be arranged. 
        For example, $T_1^4=T_1T_1T_1T_1$ can only be arranged in 1 way but $T_1^2T_2T_3=T_1T_1T_2T_3$ can be arrange in $\frac{4!}{2}=12$ ways (where we divided by 2 since the two $T_1$ can be swapped in any given arrangement).
        Then
        $$
            s_1^2s_2=\left(\sum_n T_1\right)^2s_2=\left(\sum_nT_1^2+2\sum_n T_1T_2\right)s_2 = \sum_n T_1^3T_2+\sum_nT_1^2T_2T_3+2s_2^2.
        $$
        So far we have
        \begin{align*}
            p_4 &= s_1^4-4\left(s_1^2s_2-2s_2^2-\sum_nT_1^2T_2T_3\right)-12\sum_n T_1^2T_2T_3-6\sum_nT_1^2T_2^2-24\sum_nT_1T_2T_3T_4\\
            &=s_1^4-4s_1^2s_2+8s_2^2-24s_4-6\sum_nT_1^2T_2^2-8\sum_n T_1^2T_2T_3.
        \end{align*}
        So continuing with $\sum_nT_1^2T_2^2$ we get
        $$
            s_2^2 = \left(\sum_n T_1T_2\right)^2=\sum_n T_1^2T_2^2+2\sum_n T_1^2 T_2T_3+6\sum_n T_1T_2T_3T_4.
        $$
        Finding the coefficients here is slightly trickier since $s_2$ contains pairs not all arrangements are allowed. 
        For example, $T_1^2T_2^2$ can only come from the pair $T_1T_2$. On the other hand $T_1T_2T_3T_4$ can come from $T_1T_2$ and $T_3T_4$ or $T_1T_4$ and $T_2T_3$ and so on.
        We choose the first pair (${4\choose 2}=6$ ways) which also fixes the second pair and so there are 6 ways to get $T_1T_2T_3T_4$.
        Hence
        \begin{align*}
            p_4 &= s_1^4-4s_1^2s_2+8s_2^2-24s_4-6\left(s_2^2-2\sum_nT_1^2T_2T_3-6s_4\right)-8\sum_n T_1^2T_2T_3\\
            &=s_1^4-4s_1^2s_2+2s_2^2+12s_4+4\sum_n T_1^2T_2T_3.
        \end{align*}
        Using Exercise 14.10 we get
        \begin{align*}
            p_4 &=s_1^4-4s_1^2s_2+2s_2^2+12s_4+4(s_1s_3-4s_4)\\
            &=s_1^4-4s_1^2s_2+2s_2^2-4s_4+4s_1s_3
        \end{align*}
    \end{sol}

    \begin{ex}{14.22}
        A rational function $f\in\Q[T_1,\dots,T_n]$ is called symmetric if it is invariant under all permutations of the variables $T_i$. Prove that every symmetric rational function is a rational function in the elementary symmetric functions.
    \end{ex}
    \begin{proof}
        Let $f\in\Q[T_1,\dots,T_n]$ be a symmetric rational function. 
        Then $f=g/h$ for $g,h$ polynomials. If $h$ is a symmetric polynomial then $g=fh$ is symmetric as well.
        By the fundamental theorem of symmetric polynomial both $g$ and $h$ can be written in terms of elementary symmetric polynomials and we're done.
        If $h$ is not symmetric, then let 
        $$\tilde{h}=\prod_{\sigma\in S_n\setminus\{e\}}\sigma(h)$$
        and then $h\tilde{h}$ is symmetric so $f=\frac{g\tilde{h}}{h\tilde{h}}$ which is again the case above.
    \end{proof}

    \begin{ex}{14.23}
        Write $\sum_{n}T_1^{-1}$ and $\sum_n T_1^{-2}$ as rational functions in $\Q[s_1,\dots,s_n]$
    \end{ex}
    \begin{sol}
        Starting with
        $$
            \sum_{n}T_1^{-1}=\frac{1}{T_1}+\cdots+\frac{1}{T_n}.
        $$
        We multiply by $1=\frac{s_n}{s_n}$ and simplify
        \begin{align*}
            \frac{s_n}{s_n}\sum_{n}T_1^{-1}&=\frac{T_1T_2\cdots T_n}{T_1T_2\cdots T_n}\left(\frac{1}{T_1}+\cdots+\frac{1}{T_n}\right)\\
            &=\frac{s_{n-1}}{s_n}
        \end{align*}

        For the second expression we present to approaches.
        \begin{enumerate}
            \item Observing that 
                $$\left(\sum_n T_1^{-1}\right)^2=\sum_{n} T_1^{-2}+2\sum_{n}T_1^{-1}T_2^{-1}$$
            we can write using the previous part
                $$ \sum_n T_1^{-2} = \frac{s_{n-1}^2}{s_n^2}-2\sum_{n}T_1^{-1}T_2^{-1}$$
            and multiplying by the second term by $\frac{s_{n}}{s_{n}}$ we get
                $$ \sum_n T_1^{-2} = \frac{s_{n-1}^2}{s_n^2} - 2\left(\frac{1}{T_1T_2}+\cdots+\frac{1}{T_{n-1}T_n}\right)\frac{T_1\cdots T_n}{T_1\cdots T_n}=\frac{s_{n-1}^2}{s_n^2} - 2\frac{s_{n-2}}{s_n}.$$
            Hence $\sum_n T_1^{-2}=\frac{s_{n-1}^2-2s_{n-2}s_n}{s_n^2}$.
            \item The second approach is slightly more involved. We start by multiplying by 1 in a clever (but different) way
                $$\left(\sum_n T_1^{-2}\right)\frac{s_n^2}{s_n^2}=\left(\frac{1}{T_1^2}+\cdots+\frac{1}{T_n^2}\right)\frac{T_1^2\cdots T_n^2}{T_1^2\cdots T_n^2}=\frac{\sum_n T_1^2\cdots T_{n-1}^2}{s_n^2}.$$
            Then $\sum_n T_1^2\cdots T_{n-1}^2$ is obviously (condescending much?) a symmetric polynomial and so we can use our trusty algorithm. Starting with
            \begin{align*}
                s_1^{2-2}s_2^{2-2}\cdots s_{n-1}^{2-0}&=s_{n-1}^2\\
                &=\left(\sum_n T_1\cdots T_{n-1}\right)^2\\
                &=\sum_n T_1^2\cdots T_{n-1}^2 + 2\sum_n T_1^2\cdots T_{n-2}^2T_{n-1}T_n.
            \end{align*}
            Moving to the second term
            \begin{align*}
                s_1^{2-2}\cdots s_{n-2}^{2-1}s_{n-1}^{1-1}s_n^1&=s_{n-2}s_n\\
                &=\left(\sum_n T_1\cdots T_{n-2}\right)T_1\cdots T_n\\
                &=\sum_n T_1^2\cdots T_{n-2}^2 T_{n-1}T_n
            \end{align*}
            and it follows that
            $$\sum_n T_1^2\cdots T_{n-1}^2 = s_{n-1}^2-2s_{n-2}s_n.$$
            So we conclude that
            $$ \sum_n T_1^{-2} = \frac{s_{n-1}^2-2s_{n-2}s_n}{s_n^2}$$
            which is reassuring.
        \end{enumerate}
        Note that in the first approach we stumbled upon something rather interesting:
        $$
            \sum_n T_1^{-1}\cdots T_k^{-1} = \frac{s_{n-k}}{s_n}
        $$
        the proof of which is left as an exercise to the reader.
    \end{sol}

    \begin{ex}{14.24}
        
    \end{ex}
\subsection{Field Extensions}
    \begin{ex}{21.18}
        Let $K\subset L$ be an algebraic extension. For $\alpha, \beta\in L$ prove that we have
        $$ \left[K(\alpha,\beta):K\right]\leq\left[K(\alpha):K\right]\cdot\left[K(\beta):K\right].$$

        Show that equality does not always hold. Does equality always hold if $[K(\alpha):K]$ and $[K(\beta):K]$ are relatively prime?
    \end{ex}
    \begin{proof}
        Let $f$ and $g$ be the minimal polynomials of $\al$ and $\be$ (respectively) in $K[x]$ and $f'$ be the minimal polynomial of $\alpha$ in $K(\beta)[x]$.
        If $\deg f'> \deg f$ then $f$ is a lower degree polynomial in $K(\beta)[x]$ with $f(\alpha)=0$ which is a contradiction. Hence $\deg f'\leq \deg f$ and so
        \begin{align*}
            \left[K(\alpha,\beta):K\right]&=\left[K(\al, \be):K(\be)\right]\cdot\left[K(\be):K\right]\\
            &=\deg f'\cdot \deg g\\
            &\leq\deg f\cdot \deg g\\
            &=\left[K(\al):K\right]\cdot \left[K(\be):K\right],   
        \end{align*}
        as desired.

        To show that equality does not always hold consider $\Q(\sqrt{2}, \sqrt[4]{2})$.
        Then $[\Q(\sqrt{2}):\Q]=2$ and $[\Q(\sqrt[4]{2}):\Q]=4$ but
        $$[\Q(\sqrt{2}, \sqrt[4]{2}):\Q]=4\cdot[\Q(\sqrt{2}, \sqrt[4]{2}):\Q(\sqrt[4]{2})]=4<8$$
        since $\left(\sqrt[4]{2}\right)^2=\sqrt{2}\in\Q(\sqrt[4]{2})$

        Lastly, suppose that $\deg f$ and $\deg g$ are relatively prime. Since
        \begin{align*}
            [K(\al,\be):K]&=[K(\al,\be), K(\al)]\cdot\deg f\\
            &=[K(\al,\be), K(\be)]\cdot\deg g
        \end{align*}
        it follows that $[K(\al,\be):K]$ is divisible by $\deg f$ and $\deg g$ and since they are relatively prime it is also divisible by $\deg f\cdot \deg g$.
        But we know that $[K(\al,\be):K]\leq\deg f\cdot\deg g$ and so $[K(\al,\be):K]=\deg f\cdot \deg g$.
    \end{proof}

    \begin{ex}{21.19}
        Let $K\subset K(\al)$ be an extension of odd degree. Prove that $K(\al^2)=K(\al)$.
    \end{ex}
    \begin{proof}
        Let $f$ be the minimal polynomial of $\al$ in $K[x]$. Then $\deg f=2n+1$ for some $n\in\Z_+$. 
        Since $\al^2\in K(\al)$ we get the tower $K(\al)/K\left(\al^2\right)/K$ and so\
        $$ \left[K(\al):K\right]=\left[K(\al):K\left(\al^2\right)\right]\cdot\left[K\left(\al^2\right):K\right].$$
        Let $g$ be the minimal polynomial of $\al$ in $K\left(\alpha^2\right)$. Then $\deg g\leq 2$ since $x^2-\al^2\in K\left(\alpha^2\right)$ is a polynomial with a root $\al$.
        Since $\left[K(\al):K\right]$ is odd, it is not divisible by two and so $\deg g = 1$. Hence $\left[K(\al):K\left(\al^2\right)\right]=1$ and it follows that $K(\al)=K\left(\al^2\right)$.
    \end{proof}

    \begin{ex}{21.23}
        Show that every quadratic extension of $\Q$ is of the form $\Q\left(\sqrt{d}\right)$ with $d\in\Z$.
        For what $d$ do we obtain the cyclotomic field $\Q(\zeta_3)$?
    \end{ex}
    \begin{proof}
        Let $K/\Q$ be a quadratic extension. Take $\al\in K\setminus\Q$. Then 
        $$ \Q\subset\Q(\al)\subset K $$
        and so
        $$2=\left[K:\Q\right]=\left[K:\Q(\al)\right]\left[\Q(\al):\Q\right].$$
        If $\left[\Q(\al):\Q\right]=1$ then $\Q(\al)=\Q$ and so $\al\in\Q$, which contradicts our assumption. 
        It follows that $\left[K:\Q(\al)\right]=1$ and so $K=\Q(\al)$. 
        Let 
        $$f(x)=x^2+a_1 x+a_0\in\Q[x]$$
        be the minimal polynomial of $\al$. 
        Let $d=\frac{a_1^2}{4}-a_0\in\Q$ and note that $a_0=-\al a_1-\al^2$. Then
        \begin{align*}
            \sqrt{d}&=\sqrt{\frac{a_1^2}{4}-a_0}\\
            &=\sqrt{\frac{a_1^2}{4}+a_1\al+\al^2}\\
            &=\frac{a_1+2\al}{2}.
        \end{align*}
        Hence $\sqrt{d}\in\Q(\al)$. By similar calculations we get $\al=\frac{2\sqrt{d}-a_1}{2}\in\Q(\sqrt{d})$.
        Hence $K=\Q(\al)=\Q(\sqrt{d})$. Of course, it is not yet the case the $d$ is an integer.
        Suppose that $d=\frac{p}{q}$. Since $\sqrt{d}=\frac{1}{q^2}\sqrt{qp}\in\Q(\sqrt{qp})$ we have
        $$K=\Q(\al)=\Q(\sqrt{d})=\Q(\sqrt{qp})$$
        with $qp\in\Z$ as desired.
    \end{proof}

    \begin{ex}{21.24}
        Is every cubic extension of $\Q$ of the form $\Q\left(\sqrt[3]{d}\right)$ for some $d\in\Q$?
    \end{ex}
    \begin{sol}
        No. Let $\alpha$ be a root of the monic irreducible polynomial $f(x)=x^3-3x+1\in\Q[x]$ (possible roots are $\pm 1$ and they both clearly don't work).
        There are three choices for $\alpha$ all in $\R$ (why? Using Exercise 14.16 the determinant is $4\cdot(-3)^3+27\cdot 1=-81<0$ and so by Exercise 14.20 $f$ has three real roots).
        Therefore there are three embeddings $\varphi:\Q(\alpha)\to\C$ and $\text{Im }\varphi\subset\R$.
        
        Assume for contradiction that there exists an isomorphism $\phi:\Q(\alpha)\to\Q(\sqrt[3]{d})$ for some $d\in\Q$.
        Since $\sqrt[3]{d}\not\in\Q$, $x^3-d$ is irreducible and so $f_\Q^{\sqrt[3]{d}}=x^3-d$.
        Since $f_\Q^{\sqrt[3]{d}}$ has one real and two non-real roots (again, using exercises 14.16 and 14.20 with the fact that $27\cdot(-d)^2>0$) there are three embeddings of $\Q(\sqrt[3]{d})$ into $\C$ to of which are not subsets of $\R$.
        
        Let $\Phi:\Q(\sqrt[3]{d})\to\C$ be one of the latter. 
        Then $\Phi\circ\phi:\Q(\alpha)\to\C$ is an imbedding of $\Q(\alpha)$ into $\C$ whose image is not a subset of $\R$.
        Therefore we conclude that $\phi$ doesn't exists.
    \end{sol}

    \begin{ex}{21.26}
        Let $M=\Q(\al)=\Q(1+\sqrt{2}+\sqrt{3})$. Show that $M$ is of degree 4 over $\Q$, determine the minimal polynomial and write $\sqrt{2}$ and $\sqrt{3}$ in the basis $\{1,\al, \al^2,\al^3\}$.
        Also prove that the group $G=\text{Aut}_\Q(M)$ is isomorphic to $V_4$ and that $f^\al_\Q=\prod_{\sigma\in G}X-\sigma(\al)\in\Q[X]$.
    \end{ex}
    \begin{sol}
        Let $\be = \al-1=\sqrt{2}+\sqrt{3}$. Then clearly $M=\Q(\al)=\Q(\be)$. Let
        \begin{align*}
            f(x)&=(x-\sqrt{2}-\sqrt{3})(x+\sqrt{2}-\sqrt{3})(x-\sqrt{2}+\sqrt{3})(x+\sqrt{2}+\sqrt{3})\\
            &=x^4-10x^2+1\in\Q[x]
        \end{align*}
        and so $f(\be)=0$ by construction. 
        
        Is $f$ the minimal polynomial of $\be$ in $\Q[x]$? It is if we can prove that $[M:\Q]=4$.
        From
        $$ (\sqrt{2}+\sqrt{3})(\sqrt{3}-\sqrt{2})=1 $$
        It follows that $\be^{-1}=\sqrt{3}-\sqrt{2}$. Therefore
        $$ \sqrt{2}=\frac12(\be-\be^{-1})\quad\text{and}\quad\sqrt{3}=\frac12(\be+\be^{-1})$$
        and so $M=\Q(\sqrt{2}+\sqrt{3})=\Q(\sqrt{2},\sqrt{3})$. 
        Hence we have the towers $M/\Q(\sqrt{2})/Q$ and $M/\Q(\sqrt{3})/Q$. 
        Let $g(x)=x^2-3$. Suppose it is not the minimal polynomial of $\sqrt{3}$ in $\Q(\sqrt{2})$.
        Then there exists $a+b\sqrt{2}\in\Q(\sqrt{2})$ such that
        $$ 0 = g(a+b\sqrt{2})=a^2+2b^2-3+2ab\sqrt{2}.$$
        But since
        \begin{equation*}
            \begin{cases}
                a^2+2b^2-3=0\\
                2ab=0
            \end{cases}
        \end{equation*}
        has no solutions it follows that no such element exists.
        Therefore $g$ is the minimal polynomial of $\sqrt{3}$ and $[M:\Q(\sqrt{2})]=\deg g=2$.
        Since $x^2-2$ is the minimal polynomial of $\sqrt{2}$ in $\Q$ we conclude that 
        $$[M:\Q]=[M:\Q(\sqrt{2})]\cdot[\Q(\sqrt{2}):\Q)]=4$$ 
        and therefore $f$ is the minimal polynomial of $\be$.

        Thus $f(x-1)$ is the minimal polynomial of $\al$ in $\Q$. 
        From $f(\be)=0$ it follows that $1=\beta(10\beta-\beta^3)$ and so $\be^{-1}=10\beta-\beta^3$.
        Hence
        $$\sqrt{2}=\frac12\left(\be-\be^{-1}\right)=\frac12\left(\be-10\be+\be^3\right)=\frac12\left(-9(\al-1)+(\al-1)^3\right)$$
        and
        $$\sqrt{3}=\frac12\left(\be+\be^{-1}\right)=\frac12\left(11(\al-1)-(\al-1)^3\right)$$

        Let $G=\text{Aut}(M)$ and take $\sigma\in G$. Then by definition $\sigma(1)=1$ and it follows by induction and the properties of isomorphism that $\sigma(a)=a$ for all $a\in\Z$.
        Since $1=\sigma(1)=\sigma(a\cdot a^{-1})=\sigma(a)\cdot\sigma(a)^{-1}=a\cdot a^{-1}$ it also follows that $\sigma\left(\frac{p}{q}\right)=\frac{p}{q}$. 
        Hence $\sigma$ restricted to $\Q$ is simply the identity map. 
        Therefore $\sigma$ is completely determined by $\sigma(\sqrt{2})$ and $\sigma(\sqrt{3})$.
        Since $0=\sigma(0)=\sigma(\sqrt{2}^2-2)=\sigma(\sqrt{2})^2-2$ the only options are $\sigma(\sqrt{2})=\pm\sqrt{2}$.
        Similarly we conclude that $\sigma(\sqrt{3})=\pm\sqrt{3}$. This gives four possible automorphism.
        Take $\sigma,\tau\in G$ such that $\sigma(\sqrt{2})=-\sqrt{2}, \sigma(x)=x$ $\forall x\in M\setminus\{\sqrt{2}\}$ and $\tau(\sqrt{3})=-\sqrt{3},\tau(x)=x$ $\forall x\in M\setminus\{\sqrt{3}\}$. 
        Since 
        $$\sigma\circ\sigma=\tau\circ\tau=\sigma\circ\tau\circ\sigma\circ\tau=e$$
        where $e$ is the identity map it follows that $G$ is isomorphic to $V_4$, the Klein four-group.

        Lastly, consider
        \begin{align*}
            \tilde{f}&=\prod_{\sigma\in G}x-\sigma(\al)\\
            &=(x-1-\sqrt{2}-\sqrt{3})(x-1+\sqrt{2}-\sqrt{3})(x-1-\sqrt{2}+\sqrt{3})\\&\qquad\qquad (x-1+\sqrt{2}+\sqrt{3}).
        \end{align*}
        Hence $\tilde{f}(x)=f(x-1)$ which we already proved is the minimal polynomial of $\al$ in $\Q[x]$.

    \end{sol}

    \begin{ex}{21.28}
        Prove $\Q(\sqrt{2},\sqrt[3]{3})=\Q(\sqrt{2}\sqrt[3]{3})=\Q(\sqrt{2}+\sqrt[3]{3})$.
        Determine the minimum polynomials of $\sqrt{2}\sqrt[3]{3}$ and $\sqrt{2}+\sqrt[3]{3}$ over $\Q$.
    \end{ex}
    \begin{proof}
        Clearly we have that $\Q(\sqrt{2}\sqrt[3]{3})\subset\Q(\sqrt{2},\sqrt[3]{3})$ 
        and $\Q(\sqrt{2}+\sqrt{3})\subset\Q(\sqrt{2},\sqrt[3]{3})$. 
        Since $x^2-2$ is irreducible (Eisenstein with $p=2$) and $x^3-3$ is irreducible (Eisenstein with $p=3$) and $(3,2)=1$ it follows that $[\Q(\sqrt{2},\sqrt[3]{3}):\Q]=6$.

        Now consider $f(x)=x^6-72$. Then $f(\sqrt{2}\sqrt[3]{3})=0$ and so $[\Q(\sqrt{2}\sqrt[3]{3}):\Q]\leq 6$.
        Suppose that $f(x)=a(x)b(x)$ in $\Q[x]$ for $a(x),b(x)$ non constant. Furthermore suppose without loss of generality that $\deg a\geq \deg b$.
        Reducing $f$ modulo $7$ we find that 
        $$\overline{f}(x)=x^6-2=x^6-9=(x^3-3)(x^3+3)=\overline{a}(x)\overline{b}(x)\in\mathbb{F}_7[x]$$
        Reducing $f$ modulo $5$ we get
        $$\overline{f}(x)=x^6-2=x^6+8=(x^4-2x^2+4)(x^2+2)=\overline{a}(x)\overline{b}(x)\in\mathbb{F}_5[x].$$
        Since $f$ modulo 5 has no cubic terms it follows that $\deg a = 6$ and $\deg b = 1$ and so $f$ is irreducible. Therefore $[\Q(\sqrt{2},\sqrt[3]{3}):\Q]=\deg f=6$
        and since $\Q(\sqrt{2}\sqrt[3]{3})\subset\Q(\sqrt{2},\sqrt[3]{3})$ it follows that $\Q(\sqrt{2}\sqrt[3]{3})=\Q(\sqrt{2},\sqrt[3]{3})$. 

        Let $\alpha=\sqrt{2},\beta=\sqrt[3]{3}$ and $\gamma=\alpha+\beta$. Let $L=\mathbb{Q}(\alpha, \beta)$, $K=\mathbb{Q}(\gamma)$ and suppose that $\alpha,\beta\not\in K$. Since if one of $\alpha,\beta$ is in $K$, we get the other one for free it follows that $L=K(\alpha)=K(\beta)$. Then the minimal polynomial of $\alpha$ in $K[X]$ is of degree 2 since $\alpha$ is a root of $X^2-2$ and we assumed $\alpha\not\in K$. Since $X^3-3$ has one real root and two non-real roots it follows that it is the minimal polynomial of $\beta$ in $K[X]$. Hence we conclude
        $$2=[K(\alpha):K]=[L:K]=[K(\beta):K]=3,$$
        clearly a contradiction. Therefore $K=L$ and so
        $\Q(\sqrt{2}\sqrt[3]{3})\subset\Q(\sqrt{2},\sqrt[3]{3}).$
    \end{proof}

    \begin{ex}{21.29}
        Take $K=\Q(\al)$ with $f^\al_\Q=x^3+2x^2+1$.
        \begin{enumerate}
            \item Determine the inverse of $\al+1$ in the basis $\{1,\al,\al^2\}$ of $K$ over $\Q$.
            \item Determine the minimal polynomial of $\al^2$ over $\Q$.
        \end{enumerate}
    \end{ex}
    \begin{sol}
        ${}$
        \begin{enumerate}
            \item Since
                \begin{align*}
                    0&=\al^3+2\al^2+1\\
                    &=(\al+1)(\al^2+\al-1)+2.
                \end{align*}
                It follows that $(\al+1)^{-1}=-\frac12(\al^2+\al-1)$.
            \item From $\al^3+2\al^2+1=0$ it follows that $\al^3=-2\al^2-1$.
                Squaring both sides we get that $\al^6=4\al^4+4\al^2+1$ or alternatively
                $$\left(\al^2\right)^3-4\left(\al^2\right)^2-4\left(\al^2\right)-1=0.$$
                By Ex. 19 we know that $\Q(\al)=\Q(\al^2)$.
                Therefore the minimal polynomial of $\al^2$ over $\Q$ has degree 3 and it follows that
                $$ f^{\al^2}_\Q(x)=x^3-4x^2-4x-1. $$
        \end{enumerate}
    \end{sol}

    \begin{ex}{21.30}
        Define the cyclotomic field $\Q(\zeta_5)$ and let $\al=\zeta_5^2+\zeta_5^3$.
        \begin{enumerate}
            \item Show that $\Q(\al)$ is a quadratic extension of $\Q$ and determine $f^\al_\Q$.
            \item Prove: $\Q(\al)=\Q(\sqrt{5})$
            \item Prove: $\cos(2\pi/5)=\frac{\sqrt{5}-1}{4}$ and $\sin(2\pi/5)=\sqrt{\frac{5+\sqrt{5}}{8}}$
        \end{enumerate}
    \end{ex}
    \begin{proof}
        ${}$ 
        \begin{enumerate}
            \item The degree of the 5th cyclotomic polynomial
            $$\Phi_5(x)=\prod_{\substack{1\leq k\leq 5\\(k,5)=1}}\left(x-e^{\frac{2\pi k}{5}i}\right)=x^4+x^3+x^2+x+1$$
            is 4 and since $\Phi_5=f^{\zeta_5}_\Q$ it follows that $[\Q(\zeta_5):\Q]=4$.
            Thus 
            $$[\Q(\zeta_5):\Q(\alpha)]\mid4.$$ 
            Note that $\zeta_5^3=\frac{1}{\zeta_5^2}=\overline{\zeta_5^2}$. 
            Hence $\al=\zeta_5^2+\zeta_5^3=\zeta_5^2+\overline{\zeta_5^2}\in\R$ and so $\Q(\al)\subsetneq\Q(\zeta_5)$.
            Together with the fact that $\zeta_5$ is a root of $x^3+x^2-\al\in\Q(\al)$ it follows that
            $$1<[\Q(\zeta_5):\Q(\al)]\leq 3\implies [\Q(\zeta_5):\Q(\al)]=2.$$
            Finally, since $\Q(\zeta_5)/\Q(\al)/\Q$ is a tower of fields and
            $$[\Q(\al):\Q]=\frac{[\Q(\zeta_5):\Q]}{[\Q(\zeta_5):\Q(\al)]}=2$$
            it follows that $\Q(\al)$ is a quadratic extension.

            Let $w=\zeta_5^2$. Then $\al=w+\frac1w$ and $\Phi_5(w)=0$ by definition of $\Phi_5$. Since $w\neq0$ it follows that
            \begin{align*}
                0&=1+w+w^2+w^3+w^4\\
                0&=\frac{1}{w^2}+\frac1w+1+w+w^2\\
                0&=\left(w+\frac1w\right)^2+w+\dfrac1w-1\\
                0&=\al^2+\al-1.
            \end{align*}
            Since $x^2+x-1$ is monic polynomial of degree 2 we conclude that $f^\al_\Q=x^2+x-1$.
        \item By construction $\al$ is a root of $x^2+x-1$ and so
            $$\al\in\left\{\frac{-1\pm\sqrt{5}}{2}\right\}.$$
            Since we can write $\al$ as polynomial in $\sqrt{5}$ and vice versa it follows that $\Q(\al)=\Q(\sqrt{5})$.
        \item Let $\zeta_5=e^{\frac{6\pi}{5}i}$. Then $w=\zeta_5^2=e^{\frac{2\pi}{5}i}$ and 
            $$\cos\frac{2\pi}{5}=\frac{w+\overline{w}}{2}=\frac{\al}{2}.$$
            Thus $2\cos\frac{2\pi}{5}$ is a root of $f^\al_Q$. Since $\frac{2\pi}{5}$ is in the first quadrant, $\cos\frac{2\pi}{5}$ is positive and so
            $$\cos\frac{2\pi}{5}=\frac{-1+\sqrt{5}}{4}.$$
            Therefore we also have
            \begin{align*}
                \sin\frac{2\pi}{5}&=\sqrt{1-\cos^2\frac{2\pi}{5}}\\
                &=\sqrt{\frac{5+\sqrt{5}}{8}}.
            \end{align*}
        \end{enumerate}
    \end{proof}

    \begin{ex}{21.31}
        Let $\overline{K}$ be an algebraic closure of $K$ and $L\subset\overline{K}$ a field that contains $K$. Prove that $\overline{K}$ is an algebraic closure of $L$.
    \end{ex}
    \begin{proof}
        Let $\overline{L}=\{\alpha\in\overline{K}\mid \alpha\text{ algebraic over }\}$ be the algebraic closure of $L$.
        By definition we have that $\overline{L}\subset\overline{K}$ so it is left to show the other inclusion. Let $\alpha\in\overline{K}$.
        Then $\alpha$ is algebraic over $K$ by definition, and so there exists $f\in K[x]$ such that $f(\alpha)=0$. Then $f\in L[x]$ since $K\subset L$ and so $\alpha$ is algebraic over $L$.
        Therefore $\alpha\in\overline{L}$ and so $\overline{L}=\overline{K}$.
    \end{proof}

    \begin{ex}{21.32}
        Let $K\subset L$ be a field extension and $\overline{K}$ the algebraic closure of $K$ in $L$. 
        Prove that every $\alpha\in L\setminus\overline{K}$ is transcendental over $\overline{K}$.
    \end{ex}
    \begin{proof}
        Suppose there exists $\alpha\in L\setminus\overline{K}$ that is algebraic over $\overline{K}$.
        Let
        $$f(x)=x^n+a_{n-1}x^{n-1}+\cdots+a_0$$
        be the minimal polynomial of $\alpha$ in $\overline{K}[x]$. Let 
        $$K_1=K(a_0,\dots,a_{n-1})\aand K_2=K_1(\alpha).$$
        Then $K_1/K$ is an algebraic extension since $a_0,\dots,a_n\in\overline{K}$ and $K_2/K_1$ is algebraic since $f\in K_1[x]$.
        So we have the tower of fields $K_2/K_1/K$  and it follows that $K_2/K$ is an algebraic extension and so $\alpha$ is algebraic over $K$.
        By definition of algebraic closure, $\alpha\in\overline{K}$ which contradicts our assumption. 
        Therefore $\alpha$ must be transcendental over $\overline{K}$.
    \end{proof}

    \begin{ex}{21.35}
        Let $f\in K[x]$ be a polynomial of degree $n\geq 1$. Prove: $[\Omega_K^f:K]$ divides $n!$. 
    \end{ex}
    \begin{proof}
        If $n=1$, the $K$ is splitting field of $f$ and $[K:K]=1$ divides $n!=1$.
        Suppose the statement holds for some $n\geq 1$. There are two cases
        \begin{enumerate}
            \item Suppose $f$ is irreducible, $\deg f = n+1$ and $\alpha$ is a root of $f$.
                Then $K(\alpha)$ is an extension of degree $n+1$ and $f(x)=(x-\alpha)g(x)\in K(\alpha)[x]$.
                Let $M$ be the splitting field of $g$ over $K(\alpha)$. 
                Then $[M:K(\alpha)]$ divides $n!$ by the induction hypothesis.
                But $M$ is also the splitting field of $f$ over $K$ and so
                $$[M:K]=[M:K(\alpha)][K(\alpha):K]=[M:K(\alpha)](n+1)$$
                which divides $(n+1)!$.
            \item Suppose $f$ is a reducible polynomial of degree $n+1$. Let $f(x)=h(x)g(x)$ with $h$ irreducible. 
                Construct the tower of fields $M/L/K$ such that $L$ is the splitting field of $h$ over $K$ and $M$ is the splitting field of $g$ over $L$.
                Then $[L:K]$ divides $\deg h!$ and $[M:L]$ divides $\deg g!$ by induction hypothesis. 
                Hence 
                $$[M:K]=[M:L][L:K]$$ 
                divides $\deg h!\cdot\deg g!$.
                And since 
                $${n+1\choose \deg h}=\frac{(n+1)!}{\deg h!(n+1-\deg h)!}=\frac{(n+1)!}{\deg h!\deg g!}$$
                is an integer it follows that $[M:K]$ divides $(n+1)!$.
        \end{enumerate}
    \end{proof}

    \begin{ex}{21.36}
        Let $d\in\Z$ be an integer that is not a third power in $\Z$. Prove that the splitting field $\Omega_\Q^{x^3-d}$ has degree 6 over $\Q$.
        What is the degree if $d$ is a third power?
    \end{ex}
    \begin{proof}
        Let $f(x)=x^3-d$. Suppose $f$ has a root in $r/s\in\Q$. Then $s\mid 1$ and $r\mid d$ so $f(r)=r^3-d=0\implies d=r^3$ which contradicts our assumption.
        Therefore $f(x)$ has no roots in $\Q$ and since $\deg f=3$ it follows that $f$ is irreducible. Then $\Q[X]/(f)$ is a field and $\alpha\equiv x\mod\left(x^3-d\right)$ is a zero of $f$.
        Therefore $f$ splits in $\Q(\alpha)$ as
        $$x^3-d=(x-\alpha)(x^2+\alpha x+\alpha^2)$$
        and $$[\Q(\alpha):\Q]=3$$. 
        Let $h(x)=x^2+x+1$. Then $h(x+1)=x^2+3x+3$ is irreducible in $\Q[x]$ (Eisenstein with $p=3$) and so $h(x)$ is irreducible in $\Q[x]$.
        Since $\Q[x]/(h)$ is a quadratic extension, it cannot be a subfield of the cubic extension $\Q[x]/(f)$ and so $h(x)$ has no zeros in $\Q(a)$.
        Hence it is irreducible in $\Q(a)[x]$. 
        It follows that  $\Q(\alpha)[x]/(h)\cong\Q(\alpha)(\beta)$ is a quadratic extension for $\beta\equiv x\mod\left(x^2+x+1\right)$.
        Then 
        $$f(x)=(x-\alpha)(x-\alpha\beta)(x+\alpha\beta+\alpha)$$
        and so $\Q(\alpha, \beta)$ is the splitting of $f$.
        Moreover
        $$[\Q(\alpha, \beta):\Q]=[\Q(\alpha,\beta):\Q(\alpha)]\cdot[\Q(\alpha):\Q]=2\cdot 3= 6$$
        as desired.

        If $d=r^3$ for some $r\in\Z$ then $f(x)$ is reducible since
        $$f(x)=(x-r)(x^2+rx+r^2)\in\Q[x].$$
        Then $r\beta$ is a root of $X^2+rx+r^2$ and so the quadratic extension $\Q(\beta)\cong\Q[x]/(x^2+x+1)$ is the splitting field of $f$.
    \end{proof}

    \begin{ex}{21.37}
        Determine the degree of the splitting field of $x^4-2$ over $\Q$.
    \end{ex}
    \begin{sol}
        Since
        $$x^4-2=(x-\sqrt[4]{2})(x+\sqrt[4]{2})(x-\sqrt[4]{2}i)(x+\sqrt[4]{2}i),$$
        the splitting field of $x^4-2$ is $\Q(\sqrt{2}, i)$.
        We know that $[\Q(\sqrt{2}):\Q]=2$ and $[\Q(i):\Q]=2$. Therefore the degree of $\Q(\sqrt{2},i)$ over $\Q(\sqrt{2})$ is less than 2.
        It can't be 1 since $\Q(\sqrt{2})\subset\R$ and $i\not\in\R$ and so $[\Q(\sqrt{2},i):\Q(\sqrt{2})]=2$. Therefore
        $$[\Q(\sqrt{2},i):\Q]=[\Q(\sqrt{2},i):\Q(\sqrt{2}][\Q(\sqrt{2}):\Q]=4.$$ 
    \end{sol}
    
    \begin{ex}{21.38}
        Determine the degree of the splitting field of $x^4-4$ and $x^4+4$. Explain why the notation $\Q(\sqrt[4]{4})$ and $\Q(\sqrt[4]{-4})$  is not used for the fields obtained through the adjunction of a zero of, respectively, $x^4-4$ and $x^4+4$ to $\Q$.
    \end{ex}
    \begin{sol}
        Note that
        $$ x^4-4=\left(x+\sqrt{2}\right)\left(x-\sqrt{2}\right)\left(x+\sqrt{2}i\right)\left(x-\sqrt{2}i\right).$$
        Since $i=\left(\sqrt{2}\right)^{-1}\sqrt{2}i$ the splitting field of $x^4-4$ is $\Q\left(\sqrt{2},\sqrt{2}i\right)=\Q\left(\sqrt{2},i\right)$. 
        Similarly
        $$x^4+4=\left(x-1-i\right)\left(x-1+i\right)\left(x+1-i\right)\left(x+1+i\right),$$
        and so the splitting filed of $x^4+4$ is $\Q(i)$. To compute the degree of the splitting fields note that:
        \begin{enumerate}
            \item $(x+1)^2+1$ is irreducible in $\Q[x]$ (Eisenstein with $p=2$) hence $x^2+1$ is irreducible in $\Q[x]$ and so $x^2+1$ is the minimal polynomial of $i$ over $\Q$.
            \item $x^2-2$ is the minimal polynomial of $\sqrt{2}$ over $\Q$ (Eisenstein with $p=2$)
            \item $\left[\Q\left(\sqrt{2},i\right):\Q\left(\sqrt{2}\right)\right]\leq 2$ since the minimal polynomial of $i$ over $\Q$ is of degree two by (1). 
            However $\Q\left(\sqrt{2}\right)\subset\R$ and $i\not\in\R$ so the degree cannot be one. Therefore $\left[\Q\left(\sqrt{2},i\right):\Q\left(\sqrt{2}\right)\right]=2$.
        \end{enumerate}
        It follows that
        $$[\Q(i):\Q]=2$$
        and
        $$\left[\Q\left(\sqrt{2}, i\right):\Q\right]=\left[\Q\left(\sqrt{2}, i\right):\Q\left(\sqrt{2}\right)\right]\left[\Q\left(\sqrt{2}\right):\Q\right]=4.$$   

        Outside the fact that the notation $\Q\left(\sqrt[4]{-4}\right)$ is ambiguous (which of the four roots does it stand for?), it is also misleading.
        It might seem like an extension of degree four, but as shown above it is of degree 2, regardless of which of the roots you assign to $\sqrt[4]{-4}$.
        Similarly the degree of $\Q\left(\sqrt[4]{4}\right)$ is two and not four since $\sqrt[4]{4}=\sqrt{2}$. 
        Therefore it is clearer and to simply write $\Q(\sqrt{2})$ and $\Q(1+i)$ for the adjunction of a zero of, respectively, $x^4-4$ and $x^4+4$ to $\Q$.
    \end{sol}
\subsection{Finite Fields}
\begin{ex}{22.6}
    Give an explicit isomorphism $\F_5[x]/(x^2+x+1)\xrightarrow\sim \F_5(\sqrt{2})$
\end{ex}
\begin{sol}
    Let $\varphi:\F_5[x]/(x^2+x+1)\xrightarrow \F_5(\sqrt{2})$ be an isomorphism and $\alpha$ the equivalence class of $x$ in $\F_5[x]/(x^2+x+1)$. 
    Since $\varphi$ is identity on $\F_5$, we only need to find where $\alpha$ is mapped to.
    Suppose
    $$\varphi(\alpha)=c+d\sqrt{2}.$$
    Then $\left(c+d\sqrt{2}\right)^2+c+d\sqrt{2}+1=0$. Hence 
    \begin{equation*}
        \begin{cases}
            d(2c+1)=0\\
            c^2+2d^2+c+1=0\\
        \end{cases}.
    \end{equation*}
     Since $d=0$ would be a contradiction it follows from the first equation that $c=2$.
     Substituting into the second we get $4+2d^2+3=2d^2+2$ and so $2d^2=3$. Therefore $d=2,3$ and either value will give us an isomorphism.
     So let
     $$\varphi(a+b\alpha)=a+b(2+2\sqrt{2})= 2a+2b\sqrt{2}.$$
\end{sol}

\begin{ex}{22.7}
    
\end{ex}

\begin{ex}{22.8}
    
\end{ex}

\begin{ex}{22.11}
    Let $p$ be a prime. Show that $\F_p(x)/(x^2+x+1)$ is a field if and only if $p\equiv 2\mod 3$.
\end{ex}
\begin{proof}
    $(\Rightarrow)$ Suppose $\F_p(x)/(x^2+x+1)$ is a field. So $f(x)=x^2+x+1$ is irreducible in $\F_p[x]$. 
    Therefore $\F_p$ does not contain a non-trivial cube root of unity and so 3 doesn't divide $\abs{\F_p^*}=p-1$. 
    Since $f(x)=(x+2)^2$ in $\F_3[x]$ it can't be the case that $p$ is congruent to $0\mod 3$ and so $p\equiv 2\mod 3$. 
    
    $(\Leftarrow)$ Suppose $\F_p(x)/(x^2+x+1)$ is not a field. Then $f(x)=x^2+x+1$ is reducible in $\F_p[x]$ and so $f$ has a root $\alpha\in\F_p$.
    Then $\alpha\neq 0$ and $\alpha^3=1$. Therefore 3 divides $\abs{\F_p^*}=p-1$ and so $p\equiv 1\mod 3$.

\end{proof}

\begin{ex}{22.12}
    Let $q$ be a prime power.
    \begin{enumerate}
        \item For what $q$ is the quadratic extension $\F_{q^2}$ of $\F_q$ of the form $\F_q(\sqrt{x})$ with $x\in\F_q$?
        \item For what $q$ is the cubic extension $\F_{q^3}$ of $\F_q$ of the form $\F_q(\sqrt[3]{x})$ with $x\in\F_q$?
    \end{enumerate}
\end{ex}
\begin{sol}
    ${}$
    \begin{enumerate}
        \item Let $\varphi:\F_q\to\F_q$ be given by $\varphi(a)=a^2$. If $q=p^n$ is even, then $p=2$ and $\varphi$ is a field isomorphism.
            Therefore $\F_q(\sqrt{b})=\F_q$ for all $x\in\F_q$. If $q$ is odd, then $(-1)^2=1=1^2$ so the map is not injective and so it's not surjective.
            Therefore there exists $b\in\F_q$ such that $\sqrt{b}\not\in\F_q$. Then $x^2-b$ is the minimal polynomial of $b$ and so $\F_q(\sqrt{b})$ is a quadratic extension.
            Hence $\F_q(\sqrt{b})=\F_{q^2}$.
        \item Let $\varphi:\F_q\to\F_q$ be given by $\varphi(a)=a^2$. 
            If $q\equiv 0\mod 3$ then $\varphi$ is a field isomorphism and so $\F_q(\sqrt[3]{b})=\F_q$ for all $b\in\F_q$.
            If $q\equiv 1\mod 3$ then $\abs{\F_q^*}\equiv 0\mod 3$. 
            Hence there exists an element $a\in\F_q$ of order three so $\varphi(a)=1=\varphi(1)$ and so $\varphi$ is not injective.
            It follows that there exists $b\in\F_q$ such that $\sqrt[3]{b}\not\in\F_q$. Then $F_q(\sqrt[3]{b})$ is a cubic extension and so it is equal to $F_{q^3}$.
            Lastly if $q\equiv 2\mod 3$ then $\abs{\F_q^*}\equiv 1\mod 3$ and so there is no element of order three in $\F_q$.
            Therefore $\varphi$ is injective hence surjective and so every element has a cube root. It follows that $\F_q(\sqrt[3]{b})=\F_q, \forall b\in\F_q$. 
    \end{enumerate} 
\end{sol}

\begin{ex}{22.13}
    
\end{ex}

\begin{ex}{22.15}
    Determine all the primes for which $\F_p[x]/(x^4+1)$ is a field. 
\end{ex}
\begin{sol}
    If $p=2$ then $x^4+1=(x^2+1)^2=(x+1)^4$ so $p$ is odd. 
    Then $x^4+1$ divides $x^8-1$ in $\F_p[x]$. 
    Since $p^2-1=(p+1)(p-1)$ is a product of two consecutive even numbers it follows that $8\mid p^2-1$.
    Hence $x^8-1$ splits completely in $\F_{p^2}[x]$. So
    $$x^8-1=(x-1)(x-\beta)\cdots(x-\beta^7)=(x^4+1)(x^4-1)$$
    for some $\beta\in\F_{p^2}$. 
    If $x^4+1$ is irreducible in $\F_p[x]$, then for any root $\alpha$ of $x^4+1$, $\F_p(\alpha)$ is an extension of degree four.
    But $x^4+1$ splits completely in a quadratic extension and so it is reducible in $\F_p[x]$.

    \noindent\textbf{Alternative solution}
    
    If $-1$ is a square in $\F_p$ then $a^2=-1$ for some $a\in\F_p$. So
    $$x^4+1=x^4-a^2=(x^2+a)(x^2-a).$$
    If $2$ is a square in $\F_p$ then $b^2=2$ for some $b\in\F_p$ and so
    $$x^4+1=(x^2+1)^2-(bx)^2=(x^2+1+bx)(x^2+1-bx).$$
    Lastly, if neither $-1$ nor $2$ are squares in $\F_p$, then $p$ is odd (since $-1=1=1^2$ in $\Z/2\Z$).
    Then $\F_p^*=\{1,\alpha, \alpha^2,\dots,\alpha^{p-1}\}$ is cyclic subgroup of even order. 
    Since $-1$ and $2$ are odd powers of $\alpha$, it follows that their product $-2$ is an even power of $\alpha$ and so it is a square.
    So let $c\in\F_p$ such that $c^2=-2$. Then
    $$x^4+1=(x^2-1)-(cx)^2=(x^2-1-cx)(x^2-1+cx).$$
    Therefore $x^4+1$ is reducible modulo every prime.
\end{sol}
\subsection{Separable and Normal Extensions}
\end{document}

