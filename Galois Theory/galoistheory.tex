\documentclass{article}

\usepackage[utf8]{inputenc}
\usepackage{csquotes}
\usepackage[english]{babel}
\usepackage{amsmath,amssymb,amsthm,textcomp}
\usepackage{mathtools}
\usepackage{biblatex}
\usepackage{tikz}
\usepackage{graphics, setspace}
\usepackage{listings}
\usepackage{lipsum}
\usepackage{bookmark}
\usepackage{hyperref}
\hypersetup{
    colorlinks,
    citecolor=black,
    filecolor=black,
    linkcolor=black,
    urlcolor=black
}
\theoremstyle{plain}

\DeclareMathAlphabet{\pazocal}{OMS}{zplm}{m}{n}
\DeclareMathOperator{\Ima}{Im}
\newcommand{\Ba}{\mathcal{B}}
\newcommand{\Ta}{\mathcal{T}}
\newcommand{\Aa}{\mathcal{A}}
\newcommand{\R}{\mathbb{R}}
\newcommand{\Cc}{\mathbb{C}}
\newcommand{\Z}{\mathbb{Z}}
\newcommand{\N}{\mathbb{N}}
\newcommand{\Q}{\mathbb{Q}}
\newcommand{\p}{\mathbb{P}}
\newcommand{\Ss}{\mathcal{S}} % Schwartz space
\newcommand{\Rf}{\mathcal{R}} % reflection
\newcommand{\E}{\mathbb{E}}
\newcommand{\F}{\mathbb{F}}
\newcommand{\al}{\alpha}
\newcommand{\be}{\beta}
\newcommand{\de}{\delta}
\newcommand{\e}{\varepsilon}
\newcommand{\aand}{\quad\text{and}\quad}

\DeclarePairedDelimiter\abs{\lvert}{\rvert}%
\DeclarePairedDelimiter\norm{\lVert}{\rVert}%
% Swap the definition of \abs* and \norm*, so that \abs
% and \norm resizes the size of the brackets, and the 
% starred version does not.
\makeatletter
\let\oldabs\abs
\def\abs{\@ifstar{\oldabs}{\oldabs*}}
%
\let\oldnorm\norm
\def\norm{\@ifstar{\oldnorm}{\oldnorm*}}
\makeatother


\newtheorem{theorem}{Theorem}
\newtheorem*{theorem*}{Theorem}
\newtheorem{corollary}{Corollary}[theorem]
\newtheorem{lemma}[theorem]{Lemma}
\newtheorem*{definition}{Definition}
\newtheorem*{remark}{Remark}

\theoremstyle{remark}
\newtheorem*{sol}{Solution}

\newenvironment{ex}[1]
    {\noindent\textbf{Exercise #1}\normalsize\newline}
    {\vspace{0.5 em}}

\newsavebox{\selvestebox}
\newenvironment{colbox}[1]
    {\newcommand\colboxcolor{B6D0DE}%
    \begin{lrbox}{\selvestebox}%
    \begin{minipage}{\dimexpr\columnwidth-2\fboxsep\relax}\textbf{#1}}
    {\vspace{0.5em}\end{minipage}\end{lrbox}%
    \begin{center}
    \colorbox[HTML]{\colboxcolor}{\usebox{\selvestebox}}
    \end{center}}

\setcounter{secnumdepth}{1}

\title{Galois Theory - 5122GALO6Y}
\author{Yoav Eshel}
\date{\today}

\begin{document}
    \maketitle
    \tableofcontents
    \newpage
    \section{Summary}
    \begin{colbox}{21.3}
        \begin{theorem*}
            Let $K\subset L\subset M$ be a tower of fields, $X$ a $K$-basis for $L$ and $Y$ an $L$-basis for $M$.
            Then the set of elements $xy$ with $x\in X$ and $y\in Y$ forms a $K$-basis for $M$ and we have
            $$[M:K]=[M:L][L:K].$$
            In particular, $K\subset M$ is finite if and only if $K\subset L$ and $L\subset M$ are finite
        \end{theorem*}
    \end{colbox}
    \begin{proof}
        Let $c\in M$. Then $c$ can be written uniquely as $c=\sum_{y\in Y} b_y\cdot y$ with coefficients $b_y\in L$ that are almost all 0.
        The elements $b_y\in L$ each have a unique representation as $b_y=\sum_{x\in X}a_{xy}\cdot x$ with coefficients $a_{xy}\in K$ that are almost all 0.
        Hence
        $$c=\sum_{y\in Y}\sum_{x\in X}a_{xy}\cdot x\cdot y$$
        and so $c$ is a finite unique $K$-linear combination of the elements $xy$. Therefore $(x,y)\in X\times Y$ forms a basis for $M$ over $K$.

        Since $\abs{X\times Y}=\abs{X}\cdot\abs{Y}$ it follows that
        $$[M:K]=[M:L]\cdot [L:K].$$
        Moreover, it follows that $X\times Y$ is finite if and only if $X$ and $Y$ are finite, since $X$ and $Y$ are non-empty.
    \end{proof}

    \begin{colbox}{21.5}
        \begin{theorem*}
            Let $L/K$ be a field extension and take $\alpha\in L$. Then
            \begin{enumerate}
                \item If $\alpha$ is transcendental over k, then $K[\alpha]\simeq K[X]$ and $K(\alpha)\simeq K(x)$.
                \item If $\alpha$ is algebraic over $K$ then there exists a unique monic irreducible polynomial $f=f_K^\alpha\in K[x]$ that has $\alpha$ has a zero.
                    In this case there is a field isomorphism
                    \begin{align*}
                        K[X]/(f)&\simeq K(\alpha)\\
                        g\mod (f)&\mapsto g(\alpha)
                    \end{align*}
                    and the degree of $K(\alpha)$ over $K$ is the degree of $f$.
            \end{enumerate}
        \end{theorem*}
    \end{colbox}
    \begin{proof}
        Consider the ring homomorphism $\phi:K[x]\to L$ given by $f\mapsto f(\alpha)$.
        The image of $\phi$ is equal to $K[\alpha]$, and we have two possibilities.

        If $\alpha$ is transcendental over $K$, then $\phi$ is injective, and we obtain an isomorphism $K[x]\xrightarrow{\sim}K[\alpha]$.
        The field of fractions $K(\alpha)$ is then isomorphic to $K(x)$.

        If $\alpha$ is algebraic over $K$, then $\ker\phi$ is a non-trivial ideal of $K[x]$.
        Since $K[x]$ is a PID, there is a unique monic generator $f=f_K^\alpha\in K[x]$ of $\ker\phi$.
        This is the "smallest" monic polynomial in $K[x]$ that has $\alpha$ as a zero. 
        The isomorphism theorem then gives an isomorphism $K[x]/(f_k^\alpha)\xrightarrow{\sim}K[\alpha]\subset L$ of integral domains.
        Hence $(f_K^\alpha)$ is a prime ideal in $K[x]$ and $f_K^\alpha$ is irreducible. 
        Since a prime ideal $(f_K^\alpha)\neq 0$ is a PID is maximal, we have that $K[x]/(f_k^\alpha)\cong K[\alpha]$ is a field and therefore equal to $K(\alpha)$.
        Since every polynomial modulo $f_K^\alpha$ has a unique representative $g$; obtained through division with remainder by $f_K^\alpha$. 
        If $\deg f_K^\alpha=n$ then the residue classes of $\{1,x,x^2,\dots, x^{n-1}\}$ is a basis for $K[x]/(f_K^\alpha)$ over $K$. 
        Therefore $[K(\alpha):K]=n=\deg f_K^\alpha$.
    \end{proof}

    \begin{colbox}{21.9}
        \begin{theorem*}
            For a tower $M/L/K$ of fields, we have
            $$ K\subset M\text{ is algebraic}\iff K\subset L\text{ and } L\subset M\text{ are algebraic}.$$
        \end{theorem*}
    \end{colbox}
    \begin{proof}
        If $K\subset M$ is algebraic, then for any $\alpha\in L\subset M$ there exists a polynomial $f\in K[x]$ with $f(\alpha)=0$ so $L/K$ is algebraic.
        Similarly, for any $\alpha\in M$ there exists a polynomial in $f\in K[x]\subset L[x]$ with $f(\alpha)=0$ so $M/L$ is algebraic as well.

        Suppose that $K\subset L$ and $L\subset M$ are algebraic extensions and let $c\in M$. 
        Then $c$ has a minimum polynomial $f_L^c=\sum_{i=0}^n b_ix^i\in L[x]$. 
        Each $b_i\in L$ is algebraic over $K$, so $L_0=K(b_0,\dots, b_n)$ is a finite extension of $K$. 
        Because $c$ is also algebraic over $L_0$, the extension $L_0(c)/L_0$ is finite. Since
        $$[L_0(c):K]=[L_0(c):L_0]\cdot [L_0: K]$$
        the extension $L_0(c)/K$ is also finite and so algebraic. In particular, $c$ is algebraic over $K$ and so $M/K$ is algebraic.
    \end{proof}

    \begin{colbox}{22.1}
        \begin{theorem*}
            Let $F$ be a finite field and $\F_p$ the prime field of $F$. Then $F$ is an extension of $\F_p$ of finite degree $n$ , and $F$ has $p^n$ elements.

            Conversely, for any prime power $q=p^n$, there exists, up to isomorphism, a unique field $\F_q$ with $q$ elements; it is the splitting field of $x^q-x$ over $\F_p$.
        \end{theorem*}
    \end{colbox}
    \begin{proof}
        If $F$ is finite, then $F$ is of finite degree over its prime field $\F_p$.
        If this degree is equal to $n$, then $F$, as an $n$-dimensional vector space over $\F_p$ has exactly $p^n$ elements. 
        THe group of units $F^*$ then has order $p-1$ and it follows that the elements of $F^*$ are exactly the $p^n-1$ zeros of the polynomial $x^{p^n-1}-1\in F[x]$.
        In particular we have 
        $$\prod_{\alpha\in F}(x-\alpha)=x^{p^n}-x\in\F_p[x].$$
        It follows that $F$ is a splitting field of $x^{p^n}-x$ over $\F_p$ and since splitting fields are unique (up to isomorphism), it follows there can exists at most one field with $p^n$ elements.

        Conversely, for every prime power $q=p^n>1$, a splitting field of $f(x)=x^q-x\in\F_p[x]$ over $\F_p$ is a field with $q$ elements. 
        Since $f'=-1$ has no zeros, $f$ has no double roots in an algebraic closure $\overline{\F_p}$ of $\F_p$.
        The zero set
        $$\F_q=\{a\in\overline{\F_p}\mid a^{p^n}=\alpha\}$$
        of $f$ therefore has $q=p^n$ elements. By Fermat's little theorem, we have $\F_p\subset\F_q$.
        It is clear the $\F_q$ is closed under multiplication and division by non-zero elements. The additivity of taking $p$th power implies
        $$(\alpha+\beta)^{p^n}=\alpha^{p^n}+\beta^{p^n}=\alpha+\beta\in\F_q$$
        and so $\F_q$ is closed under addition. It follows that $F_q$ is a subfield of $\overline{\F_p}$ and therefore a splitting field of $f$ over $\F_p$. 
    \end{proof}

    \begin{colbox}{22.10}
        \begin{theorem*}
            Let $\F_q$ and $\F_r$ be subfields of $\overline{\F_p}$ with, respectively, $q=p^i$ and $r=p^j$ elements.
            The following are equivalent:
            \begin{enumerate}
                \item $\F_q$ is a subfield of $\F_r$.
                \item $r$ is a power of $q$.
                \item $i$ is divisor of $j$.
            \end{enumerate}
        \end{theorem*}
    \end{colbox}
    \begin{proof}
        If $\F_r$ is an extension field of $\F_q$ of degree $d$, then we have $r=q^d$ and therefore $j=di$.
        This proves $1\implies 2\implies 3$. Finally, if $i$ is a divisor of $j$, then for $\alpha\in\overline{\F_p}$ we have the implication $\text{Frob}_p^i(\alpha)=\alpha\implies\text{Frob}_p^j(\alpha)=\alpha$.
        This is equivalent to $\F_q\subset\F_r$.
    \end{proof}

    \begin{colbox}{23.1}
        \begin{theorem*}
            Let $K\subset L_1=K(\alpha)$ be a simple algebraic field extension, $K\subset L_2$ an arbitrary field extension and $S$ the set of zeros of $f_K^\alpha$ in $L_1$.
            Then there is a bijection $\text{Hom}_K(L_1,L_2)\xrightarrow{\sim}S$ given by $\sigma\mapsto\sigma(\alpha)$.
        \end{theorem*}
    \end{colbox}
    \begin{proof}
        A homomorphism $\sigma:K(\alpha)\to L_2$ that is the identity on $K$ is fixed by the choice of the elements $\sigma(\alpha)\in L_2$.
        To the $\sigma(\alpha)$ is a zero of $f=f_K^\alpha$ in $L_2$, we write $f=\sum_{i=0}^n a_ix^i\in K[x]$.
        Then 
        $$f(\sigma(\alpha))=\sum_{i=0}^n a_i\sigma(\alpha)^i=\sigma\left(\sum_{i=0}^n a_i\alpha^i\right)=\sigma(f(\alpha))=0$$
        because $\sigma$ is the identity on $a_i\in K$. This proves $\sigma(\alpha)\in S$. 

        Conversely, for every zero $s\in S$ of $f$, the map $L_1\to L_2$ defined by $\sum_{i}c_i\alpha^i\mapsto c_is^i$ is a $K$-homomorphism $L_1\to L_2$
    \end{proof}

    \begin{colbox}{23.5}
        \begin{theorem*}
            For a finite extension $K\subset L$, the following are equivalent
            \begin{enumerate}
                \item The extension $K\subset L$ is separable
                \item We have $L=K(\alpha_1,\dots,\alpha_t)$ for elements $\alpha_1,\dots,\alpha_t\in L$ that are separable over $K$
                \item $[L:K]_s=[L:K]$
            \end{enumerate}
        \end{theorem*}
    \end{colbox}
    \begin{proof}
        $(1\implies 2)$ Since $L$ is a separable extension, every element in $L$ is separable over $K$ by definition.
        Moreover, since $L$ is finite, it is algebraic, and so we can find $\alpha_1,\dots,\alpha_t\in L$ that do the trick.

        $(2\implies 3)$ We know that for a simple extension $K(\alpha_i)$ it holds that
        $$\abs{[K(\alpha_i):K]_s}=\abs{\{\alpha\in L\mid f_K^{\alpha_i}(\alpha)=0\}}$$
        and since $\alpha_i$ is separable $f_K^{\alpha_i}$ has no repeated roots and so
        $$\abs{[K(\alpha_i):K]_s}=\abs{\{\alpha\in L\mid f_K^{\alpha_i}(\alpha)=0\}}=\deg f_K^{\alpha_i}=[K(\alpha_i):K].$$
        Let $E=K(\alpha_1,\dots,\alpha_k)$ for some $k\geq 1$ and suppose that
        $$[E:K]_s=[E:K].$$
        Then $\alpha_{k+1}$ is separable over $E$ since $f_E^{\alpha_{k+1}}\mid f_K^{\alpha_{k+1}}$ in $\overline{K}[x]$. Therefore
        $$[E(\alpha_{k+1}):K]_s=[E(\alpha_{k+1}):K]$$
        which completes the induction.

        $(3\implies 1)$ For every $\alpha\in L$ we have a tower $K\subset K(\alpha)\subset L$. 
        Since the separable degree is bounded by the degree, it follows from the equality
        $$[L:K(\alpha)]_s\cdot [K(\alpha):K]_s=[L:K]_s=[L:K]=[L:K(\alpha)]\cdot[K(\alpha):K]$$
        that $[K(\alpha):K]_s=[K(\alpha):K]$. Therefore $f_K^\alpha$ has exactly $\deg f_K^\alpha$ distinct roots in $\overline{K}$, so $\alpha$ is separable over $K$.
    \end{proof}

    \begin{colbox}{23.6}
        \begin{theorem*}
            Let $f\in K[x]$ be an irreducible polynomial, and suppose that $f$ is inseparable. Then we have $p=\text{char }K>0$ and $f=g(x^p)$ for some $g\in K[x]$.
            Moreover, not all coefficients of $f$ are $p$th powers in $K$.
        \end{theorem*}
    \end{colbox}
    \begin{proof}
        If $f$ has a double zero $\alpha\in\overline{K}$, then $\alpha$ is also a zero of the derivative of $f'$ of $f$. 
        Because (up to multiplication by a unit) $f$ is the minimum polynomial of $\alpha$ over $K$, the assumption $f'(\alpha)=0$ implies $f\mid f'$.
        Since $f'$ has a lower degree than $f$, this is only possible if $f'$ is the zero polynomial in $K[x]$. 
        
        For $K$ of characteristic 0, we find that $f$ is constant, which contradicts the assumption that $f$ is irreducible.
        We therefore have $\text{char }K=p>0$, and by explicitly taking derivatives we see that we obtain $f'=0$ for the polynomials in $K[x]$ of the form
        $f=\sum_i a_i x^{ip}\in K[x]$. For $g=\sum_i a_i x^i$ we have $f=g(x^p)$.

        If all coefficients of $f$ are $p$th powers in $K$, say $a_i=c_i^p\in K$, then we have 
        $$f=\sum_i a_ix^{ip}=\sum_i c_i^p x^{ip}=\left(\sum_i c_i x^i\right)^p$$
        which is a contradiction since $f$ is irreducible. 
    \end{proof}

    \begin{colbox}{23.14}
        \begin{theorem*}
            For a finite extension $K\subset L$ with fundamental set $X(L/K)$, the following are equivalent:
            \begin{enumerate}
                \item The extension $K\subset L$ is normal
                \item The filed $L$ is the splitting field $\Omega_K^f$ of a polynomial $f\in K[x]$.
                \item For all $\tau,\sigma\in X(L/K)$, we have $\sigma[L]=\tau[L]$.
            \end{enumerate}
        \end{theorem*}
    \end{colbox}
    \begin{proof}
        $(1\implies 2)$. Write $L=K(\beta_1,\dots, \beta_t)$. Then because of the normality of $K\subset L$,
        all the zeros of 
        $$f=f_K^{\beta_1}\cdot f_K^{\beta_2}\cdots f_K^{\beta_t}\in K[x]$$
        are in $L$. Since they generate $L$ over $K$, we have $L=\Omega_K^f$.
        
        $(2\implies 3)$. Let $L=\Omega_K^f$ and suppose that in $\overline{K}[x]$, the polynomial $f$ decomposes as $f=\prod_{i=1}^n (x-\alpha_i)$. 
        This gives an inclusion $\sigma:L=K(\alpha_1,\dots, \alpha_n)\subset\overline{K}$. For $\tau\in X(L/K)$ arbitrary, we then have $\prod_{i=1}^n(x-\tau(\alpha_i))=f$
        because $\tau$ is the identity on the coefficients of $f$. We find the $\tau$ permutes the zeros of $f$, and that given the desired equality $\tau[L]=K(\tau(\alpha_1),\dots,\tau(\alpha_n))=K(\alpha_1,\dots,\alpha_n)=\sigma[L]\subset\overline{K}$.

        $(3\implies 1)$. Choose an element $\sigma:L\to\overline{K}$ in $X(L/K)$ and take $\alpha\in L$ arbitrary. 
        Suppose that $f_K^\alpha$ decomposes as $f_K^\alpha=\prod_{i=1}^n(x-\alpha_i)\in\overline{K}[x]$. Then we have a $K$-isomorphism $\sigma_i:K(\alpha)\to K(\alpha_i)\subset\overline{K}$.
        Then each of the isomorphism $\sigma_i$ has an extension to an isomorphism $\sigma_i':\overline{L}\to\overline{K}$ where $\overline{L}$ is an algebraic closure of $L$ (and therefore of $K(\alpha)$).
        This isomorphism maps $\alpha$ to $\alpha_i\in\sigma_i'[L]$, and by assumption 3, we have $\sigma_i'[L]=\sigma[L]$ for all $\sigma_i'$. It follows that $f_K^\alpha$ decomposes into linear factors over $\sigma[L][x]$ and therefore over $L[x]$.
    \end{proof}
    \section{Introduction}
    Galois theory is about studying Polynomials with coefficients in a field ($\Q,\R,\Cc$ etc.).
    Let
    $$
        f(T) = T^n+\cdots+a_1T+a_0\in\Q[T].
    $$
    Then $f(T)$ splits completely in $\Cc[T]$ as
    $$
        f(T)=(T-\alpha_1)\cdots(T-\alpha_n)
    $$
    with $\alpha_1,\dots \alpha_n\in\Cc$ are the roots of $f$. Galois theory studies permutation of the the roots that preserve algebraic relations between these roots.
    The allowed permutation of the roots give rise to a group denoted $\text{Gal}(f)$.
    The following definition of a Galois group does not require any background knowledge but is not very useful in practice.

    \begin{definition}
        Let $\sigma:\Cc\to\Cc$ be a field automorphism and $\alpha\in\Cc$ a root of $F(T)\in\Q[T]$. Since $\sigma(1)=1$ it follows that $\sigma(n)=n$ for all integers and so $\sigma(a/b)=\sigma(a)/\sigma(b)=a/b$ is the identity on $\Q$.
        Then
        \begin{align*}
            f(\sigma(\alpha))&=\sigma(\alpha)^n+\cdots+a_1\sigma(\alpha)+a_0\\
            &=\sigma(f(\alpha))\\
            &=0.
        \end{align*}
        Then each automorphism $\sigma$ is a permutation of the roots which is precisely the Galois group of the polynomial $\text{Gal}(f)\subset S_n$.
        In other words we have a group action 
        $$
        \text{Aut}(\Cc)\times\{\alpha_1,\dots,\alpha_n\}\to\{\alpha_1,\dots,\alpha_n\}
        $$
        Then $\text{Gal}(f):=\text{Im}(\phi)$ where $\phi:\text{Aut}(\Cc)\to S_n$ mapping $\sigma\mapsto(\alpha_i\mapsto \sigma(\alpha_i))$
    \end{definition}

    $\text{Gal}(f)\subset S_n$ is transitive subgroup (i.e. if its action on the set of roots is transitive) if and only if $f$ is irreducible.
    
    \section{Symmetric Polynomials}
    A symmetric polynomial is a polynomial $F(X_1,X_2,\dots, X_n)$ the is invariant under permutations of its variables.
    In other words
    $$ P(X_1, X_2,\dots, X_n)=P(X_{\sigma(1)}, X_{\sigma(2)},\dots,X_{\sigma(n)})$$
    for all $\sigma\in S_n$.
    Symmetric polynomials arise naturally in the study of the relation between the roots of a polynomial in one variable and its coefficients, 
    since the coefficients can be given by polynomial expressions in the roots, and all roots play a similar role in this setting.
    Let $f\in K(T)$ be a monic polynomial of degree $n$ that splits completely in $K$.
    Then
    $$ f(T)=(T-X_1)(T-X_2)\cdots(T-X_n)$$
    where $X_i$ are the roots of $f$. Then
    $$ f(T)=T^n+s_1 T^{n-1}+\cdots+(-1)^n s_n$$
    where
    \begin{align*}
        s_1&=X_1+X_2+\cdots+X_n\\
        s_2&=X_1X_2+X_1X_3+\cdots+X_{n-1}X_n\\
        \vdots&\\
        s_n&=X_1X_2\cdots X_n
    \end{align*}
    are called the \textit{elementary symmetric polynomials} in $X_1,X_2,\dot, X_n$.
    Then the fundamental theorem of symmetric polynomials states that every symmetric polynomial can be written as a polynomial expression in the elementary symmetric polynomials.

    To actually write a symmetric polynomial in terms of elementary symmetric polynomials we introduce some useful notation.
    We say a polynomial is ordered \textit{lexicographically} if the monomial $T_1^{e_1}T_2^{e_2}\cdots T_n^{e_n}$ with the highest $e_1$ is in front.
    If two monomials have the same $e_1$, then we compare their $e_2$ and so on. Like a dictionary.
    If $P$ is a symmetric polynomial in $n$ variables, choose a single representative proceeded by the symbol $\sum_n$ to denote the sum over the monomials in the $S_n$ orbit of the representative.
    Then for example
    \begin{align*}
        s_1&=\sum_n T_1\\
        s_2&=\sum_n T_1T_2\\
        \vdots&\\
        s_n&=\sum_n T_1T_2\cdots T_n=T_1T_2\cdots T_n.
    \end{align*}
    Now suppose $P$ is a symmetric polynomial. To find its representation in terms of symmetric polynomials:
    \begin{enumerate}
        \item Let $a\cdot T_1^{e_1}T_2^{e_2}\cdots T_n^{e_n}$ be the the first term in $P$, lexicographically.
        \item Form the monomial $$M = s_1^{e_1-e_2}s_2^{e_2-e_1}\cdots s_{n-1}^{e_{n-1}-e_n}s_n^{e_n}$$
        \item Let $P_i = P -cM$.
        \item Repeat steps (1)-(3) until $\deg P_i = 0$.
        \item The we can solve for $P$ and write it as a polynomial in the elementary symmetric polynomials.
    \end{enumerate}
    The representation obtained through the algorithm above is unique.

    The following theorem is useful when applying the algorithm above.
    \begin{theorem}[Orbit Stabilizer Theorem]\label{th:orbit_stabilizer}
        Let $G$ be a group acting on set $S$. For any $x\in S$ let $G_x=\{g\in G\mid g\cdot x= x\}$ denote the stabilizer of $x$, and let $G\cdot x=\{g\cdot x\mid g\in G\}$ denote the orbit of $x$. Then
        $$
            \abs{G}=\abs{G\cdot x}\abs{G_x}
        $$
    \end{theorem}
    Wondering how it might be useful? Consider 
    $$s_1^4=\left(\sum_n T_1\right)^4=(T_1+\cdots+T_n)(T_1+\cdots+T_n)(T_1+\cdots+T_n)(T_1+\cdots+T_n).$$ 
    After some thinking you might conclude that there are five possible representatives:
    $$T_1^4,\quad T_1^3T_2,\quad T_1^2T_2^2,\quad T_1^2T_2T_3\quad\text{and}\quad T_1T_2T_3T_4$$
    (note the the degrees always add up to four). But what are the coefficients? That's when the orbit-stabilizer theorem comes to the rescue.
    Let the permutation group $S_4$ act on the set of indices by permuting them. Then the coefficients in front of $\sum_n T_1^4$ is the size of the orbit of $(1,1,1,1)$.
    Since every permutation in $S_4$ return the same sequence, the size of the orbit is $\frac{4!}{4!}=1$. 
    Then the coefficients in front of $\sum_n T_1^2T_2^2$ is the size of the orbit of $(1,1,2,2)$.
    The permutations that fix it are $(1), (12), (34)$ and $(12)(34)$. So the size of the stabilizer is 4 and the size of the orbit is $\frac{4!}{4}=6$
    Similarly, the coefficients in front of $\sum_n T_1^2T_2T_3$ is the size of the orbit of $(1,1,2,3)$.
    Since the stabilizer contains only the permutations that switches the 1s and fixes the other two elements (namely $(1)$ and $(12)$) the size of the orbit is $ \frac{4!}{2}=12$.
    Lastly, the size of the orbit of $\sum_n T_1T_2T_3T_4$ is the $4!$ since there is no permutations (except the identity of course) that stabilizes it. We conclude that
    $$s_1^4=\sum_n T_1^4+6\sum_n T_1^2T_2^2+12\sum_n T_1^2T_2T_3+24\sum_n T_1T_2T_3T_4$$


    
    \section{Field Extensions}
    \subsection{Prime Fields}
    \begin{definition}
        Let $K$ be a field. Then the \textbf{prime field} in $K$ is the intersection over all subfields of $K$
    \end{definition}
    \begin{lemma}
        Let $K$ be a field of characteristic $k$. Then the prime field of $K$ is $\mathbb{F}_p$ if $k=p$ and $\Q$ if $k=0$.
    \end{lemma}
    \subsection{Algebraic and Transcendental Extensions}
    Let $L/K$ be a field extensions. Then we say that $\alpha\in L $ is \textit{algebraic} over $K$ if there exists an $f\in K[x], f\neq 0$, such that $f(\alpha)=0$.
    We say that $\alpha$ is \textit{transcendental} over $K$ if there exists no such $f$.
    The number of algebraic elements over $\Q$ in $\Cc$ is countable, so in fact $\Cc$ is mostly transcendental elements.

    \begin{definition}
        We say that an extension $L/K$ is \textbf{algebraic} if $\forall \alpha\in L$, $\alpha$ is algebraic over $K$.
    \end{definition}
    \begin{lemma}
        If a field extension is finite then it is algebraic.
    \end{lemma}
    The converse of this lemma does not hold.

    \newpage
    \section{Exercises}
    \subsection{Symmetric Polynomial}
    \begin{ex}{14.10}
        Express the symmetric polynomials $\sum_n T_1^2T_2$ and $\sum_{n} T_1^3T_2$ in the elementary symmetric polynomials.
    \end{ex}
    \begin{sol}
        To get the polynomial $\sum_n T_1^2T_2$ we start with
        $$
            s_1s_2=\sum_n T_1\sum_n T_1T_2 = \sum_n T_1^2T_2+3\sum_n T_1T_2T_3 = \sum_n T_1^2T_2+3s_3
        $$
        Thus 
        $$
            \sum_n T_1^2T_2 = s_1s_2-3s_3
        $$

        Similarly, to transform the polynomial $\sum_{n} T_1^3T_2$ we start with
        \begin{align*}
            s_1^2s_2&=\left(\sum_nT_1\right)^2\sum_nT_1T_2\\
            &=\left(\sum_n T_1^2+2\sum_n T_1T_2\right)\sum_nT_1T_2\\
            &=\sum_nT_1^2\sum_n T_1T_2+2s_2^2\\
            &=\sum_nT_1^3T_2+\sum_n T_1^2T_2T_3+2s_2^2.
        \end{align*}
        And since
        $$
            s_1s_3=\sum_nT_1\sum_nT_1T_2T_3=\sum_nT_1^2T_2T_3+4\sum_n T_1T_2T_3T_4
        $$
        it follows that $\sum_n T_1^2T_2T_3=s_1s_3-4s_4$ and so
        $$
            \sum_{n} T_1^3T_2=s_1^2s_2-s_1s_3+4s_4-2s_2^2
        $$
    \end{sol}
        
    \begin{ex}{14.21}
        Express $p_4=\sum_nT_1^4$ in elementary symmetric polynomials
    \end{ex}
    \begin{sol}
        Let $n\geq 4$. Starting with
        \begin{align*}
            s_1^4 &= \left(\sum_nT_1\right)^4\\& = \sum_n T_1^4+4\sum_n T_1^3T_2+12\sum_n T_1^2T_2T_3+6\sum_nT_1^2T_2^2+24\sum_nT_1T_2T_3T_4.
        \end{align*}
        To understand how to coefficients of the sum are obtained, consider the number of ways the $T_i$ can be arranged. 
        For example, $T_1^4=T_1T_1T_1T_1$ can only be arranged in 1 way but $T_1^2T_2T_3=T_1T_1T_2T_3$ can be arrange in $\frac{4!}{2}=12$ ways (where we divided by 2 since the two $T_1$ can be swapped in any given arrangement).
        Then
        $$
            s_1^2s_2=\left(\sum_n T_1\right)^2s_2=\left(\sum_nT_1^2+2\sum_n T_1T_2\right)s_2 = \sum_n T_1^3T_2+\sum_nT_1^2T_2T_3+2s_2^2.
        $$
        So far we have
        \begin{align*}
            p_4 &= s_1^4-4\left(s_1^2s_2-2s_2^2-\sum_nT_1^2T_2T_3\right)-12\sum_n T_1^2T_2T_3-6\sum_nT_1^2T_2^2-24\sum_nT_1T_2T_3T_4\\
            &=s_1^4-4s_1^2s_2+8s_2^2-24s_4-6\sum_nT_1^2T_2^2-8\sum_n T_1^2T_2T_3.
        \end{align*}
        So continuing with $\sum_nT_1^2T_2^2$ we get
        $$
            s_2^2 = \left(\sum_n T_1T_2\right)^2=\sum_n T_1^2T_2^2+2\sum_n T_1^2 T_2T_3+6\sum_n T_1T_2T_3T_4.
        $$
        Finding the coefficients here is slightly trickier since $s_2$ contains pairs not all arrangements are allowed. 
        For example, $T_1^2T_2^2$ can only come from the pair $T_1T_2$. On the other hand $T_1T_2T_3T_4$ can come from $T_1T_2$ and $T_3T_4$ or $T_1T_4$ and $T_2T_3$ and so on.
        We choose the first pair (${4\choose 2}=6$ ways) which also fixes the second pair and so there are 6 ways to get $T_1T_2T_3T_4$.
        Hence
        \begin{align*}
            p_4 &= s_1^4-4s_1^2s_2+8s_2^2-24s_4-6\left(s_2^2-2\sum_nT_1^2T_2T_3-6s_4\right)-8\sum_n T_1^2T_2T_3\\
            &=s_1^4-4s_1^2s_2+2s_2^2+12s_4+4\sum_n T_1^2T_2T_3.
        \end{align*}
        Using Exercise 14.10 we get
        \begin{align*}
            p_4 &=s_1^4-4s_1^2s_2+2s_2^2+12s_4+4(s_1s_3-4s_4)\\
            &=s_1^4-4s_1^2s_2+2s_2^2-4s_4+4s_1s_3
        \end{align*}
    \end{sol}

    \begin{ex}{14.22}
        A rational function $f\in\Q[T_1,\dots,T_n]$ is called symmetric if it is invariant under all permutations of the variables $T_i$. Prove that every symmetric rational function is a rational function in the elementary symmetric functions.
    \end{ex}
    \begin{proof}
        Let $f\in\Q[T_1,\dots,T_n]$ be a symmetric rational function. 
        Then $f=g/h$ for $g,h$ polynomials. If $h$ is a symmetric polynomial then $g=fh$ is symmetric as well.
        By the fundamental theorem of symmetric polynomial both $g$ and $h$ can be written in terms of elementary symmetric polynomials and we're done.
        If $h$ is not symmetric, then let 
        $$\tilde{h}=\prod_{\sigma\in S_n\setminus\{e\}}\sigma(h)$$
        and then $h\tilde{h}$ is symmetric so $f=\frac{g\tilde{h}}{h\tilde{h}}$ which is again the case above.
    \end{proof}

    \begin{ex}{14.23}
        Write $\sum_{n}T_1^{-1}$ and $\sum_n T_1^{-2}$ as rational functions in $\Q[s_1,\dots,s_n]$
    \end{ex}
    \begin{sol}
        Starting with
        $$
            \sum_{n}T_1^{-1}=\frac{1}{T_1}+\cdots+\frac{1}{T_n}.
        $$
        We multiply by $1=\frac{s_n}{s_n}$ and simplify
        \begin{align*}
            \frac{s_n}{s_n}\sum_{n}T_1^{-1}&=\frac{T_1T_2\cdots T_n}{T_1T_2\cdots T_n}\left(\frac{1}{T_1}+\cdots+\frac{1}{T_n}\right)\\
            &=\frac{s_{n-1}}{s_n}
        \end{align*}

        For the second expression we present to approaches.
        \begin{enumerate}
            \item Observing that 
                $$\left(\sum_n T_1^{-1}\right)^2=\sum_{n} T_1^{-2}+2\sum_{n}T_1^{-1}T_2^{-1}$$
            we can write using the previous part
                $$ \sum_n T_1^{-2} = \frac{s_{n-1}^2}{s_n^2}-2\sum_{n}T_1^{-1}T_2^{-1}$$
            and multiplying by the second term by $\frac{s_{n}}{s_{n}}$ we get
                $$ \sum_n T_1^{-2} = \frac{s_{n-1}^2}{s_n^2} - 2\left(\frac{1}{T_1T_2}+\cdots+\frac{1}{T_{n-1}T_n}\right)\frac{T_1\cdots T_n}{T_1\cdots T_n}=\frac{s_{n-1}^2}{s_n^2} - 2\frac{s_{n-2}}{s_n}.$$
            Hence $\sum_n T_1^{-2}=\frac{s_{n-1}^2-2s_{n-2}s_n}{s_n^2}$.
            \item The second approach is slightly more involved. We start by multiplying by 1 in a clever (but different) way
                $$\left(\sum_n T_1^{-2}\right)\frac{s_n^2}{s_n^2}=\left(\frac{1}{T_1^2}+\cdots+\frac{1}{T_n^2}\right)\frac{T_1^2\cdots T_n^2}{T_1^2\cdots T_n^2}=\frac{\sum_n T_1^2\cdots T_{n-1}^2}{s_n^2}.$$
            Then $\sum_n T_1^2\cdots T_{n-1}^2$ is obviously (condescending much?) a symmetric polynomial and so we can use our trusty algorithm. Starting with
            \begin{align*}
                s_1^{2-2}s_2^{2-2}\cdots s_{n-1}^{2-0}&=s_{n-1}^2\\
                &=\left(\sum_n T_1\cdots T_{n-1}\right)^2\\
                &=\sum_n T_1^2\cdots T_{n-1}^2 + 2\sum_n T_1^2\cdots T_{n-2}^2T_{n-1}T_n.
            \end{align*}
            Moving to the second term
            \begin{align*}
                s_1^{2-2}\cdots s_{n-2}^{2-1}s_{n-1}^{1-1}s_n^1&=s_{n-2}s_n\\
                &=\left(\sum_n T_1\cdots T_{n-2}\right)T_1\cdots T_n\\
                &=\sum_n T_1^2\cdots T_{n-2}^2 T_{n-1}T_n
            \end{align*}
            and it follows that
            $$\sum_n T_1^2\cdots T_{n-1}^2 = s_{n-1}^2-2s_{n-2}s_n.$$
            So we conclude that
            $$ \sum_n T_1^{-2} = \frac{s_{n-1}^2-2s_{n-2}s_n}{s_n^2}$$
            which is reassuring.
        \end{enumerate}
        Note that in the first approach we stumbled upon something rather interesting:
        $$
            \sum_n T_1^{-1}\cdots T_k^{-1} = \frac{s_{n-k}}{s_n}
        $$
        the proof of which is left as an exercise to the reader.
    \end{sol}

\subsection{Field Extensions}

\subsection{Finite Fields}

\subsection{Separable and Normal Extensions}
\end{document}

