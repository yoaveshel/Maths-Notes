\documentclass{article}

\usepackage[utf8]{inputenc}
\usepackage{csquotes}
\usepackage[english]{babel}
\usepackage{amsmath,amssymb,amsthm,textcomp}
\usepackage{mathtools}
\usepackage{biblatex}
\usepackage{tikz}
\usepackage{graphics, setspace}
\usepackage{listings}
\usepackage{lipsum}
\usepackage{bookmark}
\usepackage{hyperref}
\hypersetup{
    colorlinks,
    citecolor=black,
    filecolor=black,
    linkcolor=black,
    urlcolor=black
}

\DeclareMathAlphabet{\pazocal}{OMS}{zplm}{m}{n}
\DeclareMathOperator{\Ima}{Im}
\newcommand{\Ba}{\mathcal{B}}
\newcommand{\Ta}{\mathcal{T}}
\newcommand{\Aa}{\mathcal{A}}
\newcommand{\R}{\mathbb{R}}
\newcommand{\C}{\mathbb{C}}
\newcommand{\Z}{\mathbb{Z}}
\newcommand{\N}{\mathbb{N}}
\newcommand{\Q}{\mathbb{Q}}
\newcommand{\p}{\mathbb{P}}
\newcommand{\Ss}{\mathcal{S}} % Schwartz space
\newcommand{\F}{\mathcal{F}} % Fourier Transform
\newcommand{\Rf}{\mathcal{R}} % reflection
\newcommand{\E}{\mathbb{E}}

\DeclarePairedDelimiter\abs{\lvert}{\rvert}%
\DeclarePairedDelimiter\norm{\lVert}{\rVert}%
% Swap the definition of \abs* and \norm*, so that \abs
% and \norm resizes the size of the brackets, and the 
% starred version does not.
\makeatletter
\let\oldabs\abs
\def\abs{\@ifstar{\oldabs}{\oldabs*}}
%
\let\oldnorm\norm
\def\norm{\@ifstar{\oldnorm}{\oldnorm*}}
\makeatother

\newtheorem{theorem}{Theorem}[section]
\newtheorem{corollary}{Corollary}[theorem]
\newtheorem{lemma}[theorem]{Lemma}
\newtheorem*{definition}{Definition}
\newtheorem*{remark}{Remark}

\theoremstyle{remark}
\newtheorem*{sol}{Solution}

\newenvironment{ex}[1]
    {\noindent\textbf{Exercise #1}\normalsize\newline}
    {\vspace{0.5 em}}

\setcounter{secnumdepth}{1}

\title{Galois Theory - 5122GALO6Y}
\author{Yoav Eshel}
\date{\today}

\begin{document}
    \maketitle
    \tableofcontents
    \newpage
    \section{Introduction}
    Galois theory is about studying Polynomials with coefficients in a field ($\Q,\R,\C$ etc.).
    Let
    $$
        f(T) = T^n+\cdots+a_1T+a_0\in\Q[T].
    $$
    Then $f(T)$ splits completely in $\C[T]$ as
    $$
        f(T)=(T-\alpha_1)\cdots(T-\alpha_n)
    $$
    with $\alpha_1,\dots \alpha_n\in\C$ are the roots of $f$. Galois theory studies permutation of the the roots that preserve algebraic relations between these roots.
    The allowed permutation of the roots give rise to a group denoted $\text{Gal}(f)$.
    The following definition of a Galois group does not require any background knowledge but is not very useful in practice.

    \begin{definition}
        Let $\sigma:\C\to\C$ be a field automorphism and $\alpha\in\C$ a root of $F(T)\in\Q[T]$. Since $\sigma(1)=1$ it follows that $\sigma(n)=n$ for all integers and so $\sigma(a/b)=\sigma(a)/\sigma(b)=a/b$ is the identity on $\Q$.
        Then
        \begin{align*}
            f(\sigma(\alpha))&=\sigma(\alpha)^n+\cdots+a_1\sigma(\alpha)+a_0\\
            &=\sigma(f(\alpha))\\
            &=0.
        \end{align*}
        Then each automorphism $\sigma$ is a permutation of the roots which is precisely the Galois group of the polynomial $\text{Gal}(f)\subset S_n$.
        In other words we have a group action 
        $$
        \text{Aut}(\C)\times\{\alpha_1,\dots,\alpha_n\}\to\{\alpha_1,\dots,\alpha_n\}
        $$
        Then $\text{Gal}(f):=\text{Im}(\phi)$ where $\phi:\text{Aut}(\C)\to S_n$ mapping $\sigma\mapsto(\alpha_i\mapsto \sigma(\alpha_i))$
    \end{definition}

    $\text{Gal}(f)\subset S_n$ is transitive subgroup (i.e. if its action on the set of roots is transitive) if and only if $f$ is irreducible.
    
    \section{Symmetric Polynomials}
    A symmetric polynomial is a polynomial $F(X_1,X_2,\dots, X_n)$ the is invariant under permutations of its variables.
    In other words
    $$ P(X_1, X_2,\dots, X_n)=P(X_{\sigma(1)}, X_{\sigma(2)},\dots,X_{\sigma(n)})$$
    for all $\sigma\in S_n$.
    Symmetric polynomials arise naturally in the study of the relation between the roots of a polynomial in one variable and its coefficients, 
    since the coefficients can be given by polynomial expressions in the roots, and all roots play a similar role in this setting.
    Let $f\in K(T)$ be a monic polynomial of degree $n$ that splits completely in $K$.
    Then
    $$ f(T)=(T-X_1)(T-X_2)\cdots(T-X_n)$$
    where $X_i$ are the roots of $f$. Then
    $$ f(T)=T^n+s_1 T^{n-1}+\cdots+(-1)^n s_n$$
    where
    \begin{align*}
        s_1&=X_1+X_2+\cdots+X_n\\
        s_2&=X_1X_2+X_1X_3+\cdots+X_{n-1}X_n\\
        \vdots&\\
        s_n&=X_1X_2\cdots X_n
    \end{align*}
    are called the \textit{elementary symmetric polynomials} in $X_1,X_2,\dot, X_n$.
    Then the fundamental theorem of symmetric polynomials states that every symmetric polynomial can be written as a polynomial expression in the elementary symmetric polynomials.

    To actually write a symmetric polynomial in terms of elementary symmetric polynomials we introduce some useful notation.
    We say a polynomial is ordered \textit{lexicographically} if the monomial $T_1^{e_1}T_2^{e_2}\cdots T_n^{e_n}$ with the highest $e_1$ is in front.
    If two monomials have the same $e_1$, then we compare their $e_2$ and so on. Like a dictionary.
    If $P$ is a symmetric polynomial in $n$ variables, choose a single representative proceeded by the symbol $\sum_n$ to denote the sum over the monomials in the $S_n$ orbit of the representative.
    Then for example
    \begin{align*}
        s_1&=\sum_n T_1\\
        s_2&=\sum_n T_1T_2\\
        \vdots&\\
        s_n&=\sum_n T_1T_2\cdots T_n=T_1T_2\cdots T_n.
    \end{align*}
    Now suppose $P$ is a symmetric polynomial. To find its representation in terms of symmetric polynomials:
    \begin{enumerate}
        \item Let $a\cdot T_1^{e_1}T_2^{e_2}\cdots T_n^{e_n}$ be the the first term in $P$, lexicographically.
        \item Form the monomial $$M = s_1^{e_1-e_2}s_2^{e_2-e_1}\cdots s_{n-1}^{e_{n-1}-e_n}s_n^{e_n}$$
        \item Let $P_i = P -cM$.
        \item Repeat steps (1)-(3) until $\deg P_i = 0$.
        \item The we can solve for $P$ and write it as a polynomial in the elementary symmetric polynomials.
    \end{enumerate}
    The representation obtained through the algorithm above is unique.

    
    When expanding 
    The following theorem is useful when applying the algorithm above.
    \begin{theorem}[Orbit Stabilizer Theorem]\label{th:orbit_stabilizer}
        Let $G$ be a group acting on set $S$. For any $x\in S$ let $G_x=\{g\in G\mid g\cdot x= x\}$ denote the stabilizer of $x$, and let $G\cdot x=\{g\cdot x\mid g\in G\}$ denote the orbit of $x$. Then
        $$
            \abs{G}=\abs{G\cdot x}\abs{G_x}
        $$
    \end{theorem}
    Since $S_n$ is acting on the set $\{T_1,\dots,T_2\}$ we can find the number of elements in a given sum. Since $\abs{S_n}=n!$ the orbit of an elementary is given by
    $$\frac{n!}{\text{size of stabilizer}}$$
    
    \section{Field Extensions}
    \subsection{Prime Fields}
    \begin{definition}
        Let $k$ be a field. Then the \textbf{prime field} in $K$ is the intersection over all subfields of $K$
    \end{definition}
    \begin{lemma}
        Let $K$ be a field of characteristic $k$. Then the prime field of $K$ is $\mathbb{F}_p$ if $k=p$ and $\Q$ if $k=0$.
    \end{lemma}
    \subsection{Algebraic and Transcendental Extensions}
    Let $L/K$ be a field extensions. Then we say that $\alpha\in L $ is \textit{algebraic} over $K$ if there exists an $f\in K[x], f\neq 0$, such that $f(\alpha)=0$.
    We say that $\alpha$ is \textit{transcendental} over $K$ if there exists no such $f$.
    The number of algebraic elements over $\Q$ in $\C$ is countable, so in fact $\C$ is mostly transcendental elements.

    \begin{theorem}
        Let $L/K$ be a field extension and take $\alpha\in L$. Then
        \begin{enumerate}
            \item If $\alpha$ is transcendental over k, then $K[\alpha]\simeq K[X]$
            \item If $\alpha$ is algebraic over $K$ then there exists $f\in K[X]$ monic and irreducible and
                $$ K[X]/f\simeq K[\alpha]=K(\alpha)$$
                and the degree of $L$ over $K$ is the degree of $f$.
        \end{enumerate}
    \end{theorem}

    \begin{definition}
        We say that an extension $L/K$ is \textbf{algebraic} if $\forall \alpha\in L$, $\alpha$ is algebraic over $K$.
    \end{definition}
    \begin{lemma}
        If a field extension is finite then it is algebraic.
    \end{lemma}
    The converse of this lemma does not hold.

    \newpage
    \section{Exercises}
    \subsection{Symmetric Polynomial}
    \begin{ex}{14.10}
        Express the symmetric polynomials $\sum_n T_1^2T_2$ and $\sum_{n} T_1^3T_2$ in the elementary symmetric polynomials.
    \end{ex}
    \begin{sol}
        To get the polynomial $\sum_n T_1^2T_2$ we start with
        $$
            s_1s_2=\sum_n T_1\sum_n T_1T_2 = \sum_n T_1^2T_2+3\sum_n T_1T_2T_3 = \sum_n T_1^2T_2+3s_3
        $$
        Thus 
        $$
            \sum_n T_1^2T_2 = s_1s_2-3s_3
        $$

        Similarly, to transform the polynomial $\sum_{n} T_1^3T_2$ we start with
        \begin{align*}
            s_1^2s_2&=\left(\sum_nT_1\right)^2\sum_nT_1T_2\\
            &=\left(\sum_n T_1^2+2\sum_n T_1T_2\right)\sum_nT_1T_2\\
            &=\sum_nT_1^2\sum_n T_1T_2+2s_2^2\\
            &=\sum_nT_1^3T_2+\sum_n T_1^2T_2T_3+2s_2^2.
        \end{align*}
        And since
        $$
            s_1s_3=\sum_nT_1\sum_nT_1T_2T_3=\sum_nT_1^2T_2T_3+4\sum_n T_1T_2T_3T_4
        $$
        it follows that $\sum_n T_1^2T_2T_3=s_1s_3-4s_4$ and so
        $$
            \sum_{n} T_1^3T_2=s_1^2s_2-s_1s_3+4s_4-2s_2^2
        $$
    \end{sol}
        
    \begin{ex}{14.21}
        Express $p_4=\sum_nT_1^4$ in elementary symmetric polynomials
    \end{ex}
    \begin{sol}
        Let $n\geq 4$. Starting with
        \begin{align*}
            s_1^4 &= \left(\sum_nT_1\right)^4\\& = \sum_n T_1^4+4\sum_n T_1^3T_2+12\sum_n T_1^2T_2T_3+6\sum_nT_1^2T_2^2+24\sum_nT_1T_2T_3T_4.
        \end{align*}
        To understand how to coefficients of the sum are obtained, consider the number of ways the $T_i$ can be arranged. 
        For example, $T_1^4=T_1T_1T_1T_1$ can only be arranged in 1 way but $T_1^2T_2T_3=T_1T_1T_2T_3$ can be arrange in $\frac{4!}{2}=12$ ways (where we divided by 2 since the two $T_1$ can be swapped in any given arrangement).
        Then
        $$
            s_1^2s_2=\left(\sum_n T_1\right)^2s_2=\left(\sum_nT_1^2+2\sum_n T_1T_2\right)s_2 = \sum_n T_1^3T_2+\sum_nT_1^2T_2T_3+2s_2^2.
        $$
        So far we have
        \begin{align*}
            p_4 &= s_1^4-4\left(s_1^2s_2-2s_2^2-\sum_nT_1^2T_2T_3\right)-12\sum_n T_1^2T_2T_3-6\sum_nT_1^2T_2^2-24\sum_nT_1T_2T_3T_4\\
            &=s_1^4-4s_1^2s_2+8s_2^2-24s_4-6\sum_nT_1^2T_2^2-8\sum_n T_1^2T_2T_3.
        \end{align*}
        So continuing with $\sum_nT_1^2T_2^2$ we get
        $$
            s_2^2 = \left(\sum_n T_1T_2\right)^2=\sum_n T_1^2T_2^2+2\sum_n T_1^2 T_2T_3+6\sum_n T_1T_2T_3T_4.
        $$
        Finding the coefficients here is slightly trickier since $s_2$ contains pairs not all arrangements are allowed. 
        For example, $T_1^2T_2^2$ can only come from the pair $T_1T_2$. On the other hand $T_1T_2T_3T_4$ can come from $T_1T_2$ and $T_3T_4$ or $T_1T_4$ and $T_2T_3$ and so on.
        We choose the first pair (${4\choose 2}=6$ ways) which also fixes the second pair and so there are 6 ways to get $T_1T_2T_3T_4$.
        Hence
        \begin{align*}
            p_4 &= s_1^4-4s_1^2s_2+8s_2^2-24s_4-6\left(s_2^2-2\sum_nT_1^2T_2T_3-6s_4\right)-8\sum_n T_1^2T_2T_3\\
            &=s_1^4-4s_1^2s_2+2s_2^2+12s_4+4\sum_n T_1^2T_2T_3.
        \end{align*}
        Using Exercise 14.10 we get
        \begin{align*}
            p_4 &=s_1^4-4s_1^2s_2+2s_2^2+12s_4+4(s_1s_3-4s_4)\\
            &=s_1^4-4s_1^2s_2+2s_2^2-4s_4+4s_1s_3
        \end{align*}
    \end{sol}

    \begin{ex}{14.22}
        A rational function $f\in\Q[T_1,\dots,T_n]$ is called symmetric if it is invariant under all permutations of the variables $T_i$. Prove that every symmetric rational function is a rational function in the elementary symmetric functions.
    \end{ex}
    \begin{proof}
        Let $f\in\Q[T_1,\dots,T_n]$ be a symmetric rational function. 
        Then $f=g/h$ for $g,h$ polynomials. If $h$ is a symmetric polynomial then $g=fh$ is symmetric as well.
        By the fundamental theorem of symmetric polynomial both $g$ and $h$ can be written in terms of elementary symmetric polynomials and we're done.
        If $h$ is not symmetric, then let 
        $$\tilde{h}=\prod_{\sigma\in S_n\setminus\{e\}}\sigma(h)$$
        and then $h\tilde{h}$ is symmetric so $f=\frac{g\tilde{h}}{h\tilde{h}}$ which is again the case above.
    \end{proof}

    \begin{ex}{14.23}
        Write $\sum_{n}T_1^{-1}$ and $\sum_n T_1^{-2}$ as rational functions in $\Q[s_1,\dots,s_n]$
    \end{ex}
    \begin{sol}
        Starting with
        $$
            \sum_{n}T_1^{-1}=\frac{1}{T_1}+\cdots+\frac{1}{T_n}.
        $$
        We multiply by $1=\frac{s_n}{s_n}$ and simplify
        \begin{align*}
            \frac{s_n}{s_n}\sum_{n}T_1^{-1}&=\frac{T_1T_2\cdots T_n}{T_1T_2\cdots T_n}\left(\frac{1}{T_1}+\cdots+\frac{1}{T_n}\right)\\
            &=\frac{s_{n-1}}{s_n}
        \end{align*}

        For the second expression we present to approaches.
        \begin{enumerate}
            \item Observing that 
                $$\left(\sum_n T_1^{-1}\right)^2=\sum_{n} T_1^{-2}+2\sum_{n}T_1^{-1}T_2^{-1}$$
            we can write using the previous part
                $$ \sum_n T_1^{-2} = \frac{s_{n-1}^2}{s_n^2}-2\sum_{n}T_1^{-1}T_2^{-1}$$
            and multiplying by the second term by $\frac{s_{n}}{s_{n}}$ we get
                $$ \sum_n T_1^{-2} = \frac{s_{n-1}^2}{s_n^2} - 2\left(\frac{1}{T_1T_2}+\cdots+\frac{1}{T_{n-1}T_n}\right)\frac{T_1\cdots T_n}{T_1\cdots T_n}=\frac{s_{n-1}^2}{s_n^2} - 2\frac{s_{n-2}}{s_n}.$$
            Hence $\sum_n T_1^{-2}=\frac{s_{n-1}^2-2s_{n-2}s_n}{s_n^2}$.
            \item The second approach is slightly more involved. We start by multiplying by 1 in a clever (but different) way
                $$\left(\sum_n T_1^{-2}\right)\frac{s_n^2}{s_n^2}=\left(\frac{1}{T_1^2}+\cdots+\frac{1}{T_n^2}\right)\frac{T_1^2\cdots T_n^2}{T_1^2\cdots T_n^2}=\frac{\sum_n T_1^2\cdots T_{n-1}^2}{s_n^2}.$$
            Then $\sum_n T_1^2\cdots T_{n-1}^2$ is obviously (condescending much?) a symmetric polynomial and so we can use our trusty algorithm. Starting with
            \begin{align*}
                s_1^{2-2}s_2^{2-2}\cdots s_{n-1}^{2-0}&=s_{n-1}^2\\
                &=\left(\sum_n T_1\cdots T_{n-1}\right)^2\\
                &=\sum_n T_1^2\cdots T_{n-1}^2 + 2\sum_n T_1^2\cdots T_{n-2}^2T_{n-1}T_n.
            \end{align*}
            Moving to the second term
            \begin{align*}
                s_1^{2-2}\cdots s_{n-2}^{2-1}s_{n-1}^{1-1}s_n^1&=s_{n-2}s_n\\
                &=\left(\sum_n T_1\cdots T_{n-2}\right)T_1\cdots T_n\\
                &=\sum_n T_1^2\cdots T_{n-2}^2 T_{n-1}T_n
            \end{align*}
            and it follows that
            $$\sum_n T_1^2\cdots T_{n-1}^2 = s_{n-1}^2-2s_{n-2}s_n.$$
            So we conclude that
            $$ \sum_n T_1^{-2} = \frac{s_{n-1}^2-2s_{n-2}s_n}{s_n^2}$$
            which is reassuring.
        \end{enumerate}
        Note that in the first approach we stumbled upon something rather interesting:
        $$
            \sum_n T_1^{-1}\cdots T_k^{-1} = \frac{s_{n-k}}{s_n}
        $$
        the proof of which is left as an exercise to the reader.
    \end{sol}

\subsection{Field Extensions}

\subsection{Finite Fields}

\subsection{Separable and Normal Extensions}
\end{document}

