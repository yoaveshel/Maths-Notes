
\subsection{Symmetric Polynomial}
    \begin{ex}{14.10}
        Express the symmetric polynomials $\sum_n T_1^2T_2$ and $\sum_{n} T_1^3T_2$ in the elementary symmetric polynomials.
    \end{ex}
    \begin{sol}
        To get the polynomial $\sum_n T_1^2T_2$ we start with
        $$
            s_1s_2=\sum_n T_1\sum_n T_1T_2 = \sum_n T_1^2T_2+3\sum_n T_1T_2T_3 = \sum_n T_1^2T_2+3s_3
        $$
        Thus 
        $$
            \sum_n T_1^2T_2 = s_1s_2-3s_3
        $$

        Similarly, to transform the polynomial $\sum_{n} T_1^3T_2$ we start with
        \begin{align*}
            s_1^2s_2&=\left(\sum_nT_1\right)^2\sum_nT_1T_2\\
            &=\left(\sum_n T_1^2+2\sum_n T_1T_2\right)\sum_nT_1T_2\\
            &=\sum_nT_1^2\sum_n T_1T_2+2s_2^2\\
            &=\sum_nT_1^3T_2+\sum_n T_1^2T_2T_3+2s_2^2.
        \end{align*}
        And since
        $$
            s_1s_3=\sum_nT_1\sum_nT_1T_2T_3=\sum_nT_1^2T_2T_3+4\sum_n T_1T_2T_3T_4
        $$
        it follows that $\sum_n T_1^2T_2T_3=s_1s_3-4s_4$ and so
        $$
            \sum_{n} T_1^3T_2=s_1^2s_2-s_1s_3+4s_4-2s_2^2
        $$
    \end{sol}

    \begin{ex}{14.14}
        Prove: For $n\in\Z_{>0}$, we have $\Delta(X^n+a)=(-1)^{\frac12n(n-1)}n^na^{n-1}$.
    \end{ex}
    \begin{proof}
        Let $f(X)=X^n+a$ and let $\alpha_i$ be its roots. Then $f'(X)=nX^{n-1}$ and
        $$
            \Delta(f)=(-1)^{n(n-1)/2}R(f,f').
        $$
        Let $f_1(X)=a$ and then $f\equiv f_1\mod(f')$ since $f = f_1+f'\cdot\left(\frac1n X\right)$.
        Simplifying the resultant we get
        \begin{align*}
            R(f,f')&=R(f',f)&&(\text{Property }1)\\
            &=n^{n}R(f',f_1)&&(\text{Property }3)\\
            &=n^{n}\cdot\left(n^0\prod_{i=1}^{n-1}f_1(\alpha_i)\right)&&(\text{Property }2)\\
            &=n^n a^{n-1}
        \end{align*}
        and the result follows.
    \end{proof}

    \begin{ex}{14.15}
        Calculate the discriminant of the polynomial $f(X)=X^4+pX+q\in\Q(p,q)[X]$.
    \end{ex}
    \begin{sol}
        Then $f'(X)=4X^3+p$ and so 
        $$f_1(X)=f-f'\cdot h = X^4+pX+q+(4X^3+p)(\frac14 X) = \frac{3p}{4}X+q.$$
        Then the resultant is
        \begin{align*}
            R(f,f')&=R(f',f)&&(\text{Property } 1)\\
            &=4^{4-1}R(f', f_1)&&(\text{Property } 3)\\
            &=4^3\left((-1)^{3\cdot 1}R(f_1,f')\right)&&(\text{Property } 1)\\
            &=-4^3\left(\left(\frac{3p}{4}\right)^3\prod_{i=1}^{1}f'\left(\frac{-4q}{3p}\right)\right)&&(\text{Property } 2)\\
            &=-3^3p^3\left(4\left(\frac{-4q}{3p}\right)^3+p\right)\\
            &=4^4q^3-3^3p^4.
        \end{align*}
        Therefore the discriminant of $f$ is
        $$
            \Delta(f) = (-1)^{4\cdot 3/2}R(f,f') = R(f, f') = 4^4q^3-3^3p^4.
        $$
    \end{sol}

    \begin{ex}{14.16}
        For every $n>1$, determine an expression for the discriminant of the polynomial $f(X) = X^n+pX+q\in\Q(p,q)[X]$.
    \end{ex}
    \begin{sol}
        Let $f(X)=X^n+pX+q\in\Q(p,q)[X]$ for $n>1$. 
        Then $f'(X)=nX^{n-1}+p$ and $f\equiv f_1\mod(f')$ where
        $$f_1 = f-f'\cdot h = X^n+pX+q-\left(nX^{n-1}+p\right)\left(\frac1n X\right)=\frac{p(n-1)}{n}X+q.$$
        The resultant of $f$ and $f'$ is given by
        \begin{align*}
            R(f,f') &= R(f', f)&&(\text{Property } 1)\\
            &=n^{n-1}R(f', f_1)&&(\text{Property } 3)\\
            &=n^{n-1}\left((-1)^{n-1}R(f_1, f')\right)&&(\text{Property } 1)\\
            &=(-n)^{n-1}\left(\frac{p(n-1)}{n}\right)^{n-1}\prod_{i=1}^1 f'\left(-\frac{nq}{(n-1)p}\right)&&(\text{Property } 2)\\
            &=(-1)^{n-1}p^{n-1}(n-1)^{n-1}\left(\frac{(-1)^{n-1}n^nq^{n-1}}{(n-1)^{n-1}p^{n-1}}+p\right)\\
            &=n^nq^{n-1}+(-1)^{n-1}p^n(n-1)^{n-1}.
        \end{align*}
        Hence the discriminant of $f$ is
        $$
            \Delta(f)=(-1)^{n(n-1)/2}R(f,f')=(-1)^{n(n-1)/2}\left(n^nq^{n-1}+(-1)^{n-1}p^n(n-1)^{n-1}\right)
        $$
    \end{sol}

    \begin{ex}{14.17}
        Let $f\in\Z[X]$ be a monic polynomial. Prove that the following are equivalent
        \begin{enumerate}
            \item $\Delta(f)\neq 0$.
            \item The polynomial $f$ has no double zeroes in $\C$.
            \item The decomposition of $f$ in $\Q[X]$ has no multiple prime factors.
            \item The polynomial $f$ and its derivative $f'$ are relatively prime in $\Q[X]$.
            \item The polynomial $f\mod p $ and $f' \mod p$ are relatively prime in $\mathbb{F}_p[X]$ for almost all prime numbers $p$.
        \end{enumerate}
    \end{ex}
    \begin{proof}
        Let  $f\in\Z[X]$ be monic and $\{\alpha_1,\alpha_2,\dots,\alpha_n\}$ it roots in $\C$.

        $(1)\Rightarrow  (2)$. Suppose that $\alpha_i=\alpha_j$ for some $i\neq j$. Then 
        $$\Delta(f)=\prod_{1\leq i<j\leq n}(\alpha_i-\alpha_j)= 0,$$ 
        which is a contradiction.
        Therefore if $f$ has non-zero discriminant it has no double zeroes in $\C$. 

        $(2)\Rightarrow (3)$.
        
        $(3)\Rightarrow (4)$.

        $(4)\Rightarrow (5)$. If $f$ and $f'$ are relatively prime in $\Q[X]$ then 

        $(1)\Rightarrow (1)$. 
    \end{proof}

    \begin{ex}{14.19}
        Let $f\in\Q[X]$ be a monic polynomial with $n=\deg(f)$ distinct complex roots. Prove: the sign of $\Delta(f)$ is equal to $(-1)^s$ where $2s$ is the number of non-real zeroes of $f$.
    \end{ex}
    \begin{proof}
        Let $\{\alpha_1,\dots,\alpha_{n}\}$ be all the roots of $f$.
        Then each term $(\alpha_i-\alpha_j)^2$ in the discriminant falls into one of 3 cases
        \begin{enumerate}
            \item Both $\alpha_i$ and $\alpha_j$ are non-real. Then
            \begin{enumerate}
                \item If $\alpha_j=\overline{\alpha_i}$ then $\alpha_i-\alpha_j$ is purely complex and $(\alpha_i-\alpha_j)^2$ is negative.
                \item If $\alpha_j\neq\overline{\alpha_i}$ then $\overline{\alpha_i}$ and $\overline{\alpha_j}$ are also roots of $f$ and the term
                $$(\alpha_i-\alpha_j)^2(\overline{\alpha_i}-\overline{\alpha_j})^2=\left((\overline{\alpha_i-\alpha_j})(\alpha_i-\alpha_j)\right)^2=\abs{\alpha_i-\alpha_j}^2 $$
                is positive.
            \end{enumerate}
            \item $\alpha_i$ is non-real and $\alpha_j$ is real. Then $\overline{\alpha_i}$ is a root of $f$ and the term
            $$(\alpha_i-\alpha_j)^2(\overline{\alpha_i}-\alpha_j)^2=\abs{\alpha_i-\alpha_j}^2 $$
            is positive.
            \item Both $\alpha_i$ and $\alpha_j$ are real. Then $(\alpha_i-\alpha_j)^2$ is positive.
        \end{enumerate}
        Since the only negative terms are of the form $(\alpha_i-\overline{\alpha_i})^2$ and there are $2s$ non-real roots the sign of the determinant is $(-1)^s$.

    \end{proof}

    \begin{ex}{14.20}
        Prove: $f(X)=X^3+pX+q\in\R[X]$ has three (counted with multiplicity) real zeroes $\iff$ $4p^3+27q^\leq 0$.
    \end{ex}
    \begin{proof}
        By Ex. 16 we know that $\Delta(f)=(-1)^3\left(3^3q^2+2^2p^3\right)=-27q^2-4p^3$. 
        Let $a,b$ and $c$ be the roots of $f$. If $a,b,c\in\R$ then 
        $$
        -27q^2-4p^3=\Delta(f)=(a-b)^2(a-c)^2(b-c)^2\geq 0
        $$
        and so $4p^3+27q^\leq 0$.

        Now suppose that $a=x+yi$ and $b=x-yi$ are complex conjugates and $c$ is real. Then 
        \begin{align*}
            -27q^2-4p^3&=\Delta(f)\\
            &=(a-b)^2(a-c)^2(b-c)^2\\
            &=-4y^2\left((a-c)(\overline{a-c})\right)^2\\
            &=-4y^2\abs{a-c}^2\\
            &\leq 0.
        \end{align*}
        Hence $4p^3+27q^\geq 0$ and the result follows by contraposition.
    \end{proof}
        
    \begin{ex}{14.21}
        Express $p_4=\sum_nT_1^4$ in elementary symmetric polynomials
    \end{ex}
    \begin{sol}
        Let $n\geq 4$. Starting with
        \begin{align*}
            s_1^4 &= \left(\sum_nT_1\right)^4\\& = \sum_n T_1^4+4\sum_n T_1^3T_2+12\sum_n T_1^2T_2T_3+6\sum_nT_1^2T_2^2+24\sum_nT_1T_2T_3T_4.
        \end{align*}
        To understand how to coefficients of the sum are obtained, consider the number of ways the $T_i$ can be arranged. 
        For example, $T_1^4=T_1T_1T_1T_1$ can only be arranged in 1 way but $T_1^2T_2T_3=T_1T_1T_2T_3$ can be arrange in $\frac{4!}{2}=12$ ways (where we divided by 2 since the two $T_1$ can be swapped in any given arrangement).
        Then
        $$
            s_1^2s_2=\left(\sum_n T_1\right)^2s_2=\left(\sum_nT_1^2+2\sum_n T_1T_2\right)s_2 = \sum_n T_1^3T_2+\sum_nT_1^2T_2T_3+2s_2^2.
        $$
        So far we have
        \begin{align*}
            p_4 &= s_1^4-4\left(s_1^2s_2-2s_2^2-\sum_nT_1^2T_2T_3\right)-12\sum_n T_1^2T_2T_3-6\sum_nT_1^2T_2^2-24\sum_nT_1T_2T_3T_4\\
            &=s_1^4-4s_1^2s_2+8s_2^2-24s_4-6\sum_nT_1^2T_2^2-8\sum_n T_1^2T_2T_3.
        \end{align*}
        So continuing with $\sum_nT_1^2T_2^2$ we get
        $$
            s_2^2 = \left(\sum_n T_1T_2\right)^2=\sum_n T_1^2T_2^2+2\sum_n T_1^2 T_2T_3+6\sum_n T_1T_2T_3T_4.
        $$
        Finding the coefficients here is slightly trickier since $s_2$ contains pairs not all arrangements are allowed. 
        For example, $T_1^2T_2^2$ can only come from the pair $T_1T_2$. On the other hand $T_1T_2T_3T_4$ can come from $T_1T_2$ and $T_3T_4$ or $T_1T_4$ and $T_2T_3$ and so on.
        We choose the first pair (${4\choose 2}=6$ ways) which also fixes the second pair and so there are 6 ways to get $T_1T_2T_3T_4$.
        Hence
        \begin{align*}
            p_4 &= s_1^4-4s_1^2s_2+8s_2^2-24s_4-6\left(s_2^2-2\sum_nT_1^2T_2T_3-6s_4\right)-8\sum_n T_1^2T_2T_3\\
            &=s_1^4-4s_1^2s_2+2s_2^2+12s_4+4\sum_n T_1^2T_2T_3.
        \end{align*}
        Using Exercise 14.10 we get
        \begin{align*}
            p_4 &=s_1^4-4s_1^2s_2+2s_2^2+12s_4+4(s_1s_3-4s_4)\\
            &=s_1^4-4s_1^2s_2+2s_2^2-4s_4+4s_1s_3
        \end{align*}
    \end{sol}

    \begin{ex}{14.22}
        A rational function $f\in\Q[T_1,\dots,T_n]$ is called symmetric if it is invariant under all permutations of the variables $T_i$. Prove that every symmetric rational function is a rational function in the elementary symmetric functions.
    \end{ex}
    \begin{proof}
        Let $f\in\Q[T_1,\dots,T_n]$ be a symmetric rational function. 
        Then $f=g/h$ for $g,h$ polynomials. If $h$ is a symmetric polynomial then $g=fh$ is symmetric as well.
        By the fundamental theorem of symmetric polynomial both $g$ and $h$ can be written in terms of elementary symmetric polynomials and we're done.
        If $h$ is not symmetric, then let 
        $$\tilde{h}=\prod_{\sigma\in S_n\setminus\{e\}}\sigma(h)$$
        and then $h\tilde{h}$ is symmetric so $f=\frac{g\tilde{h}}{h\tilde{h}}$ which is again the case above.
    \end{proof}

    \begin{ex}{14.23}
        Write $\sum_{n}T_1^{-1}$ and $\sum_n T_1^{-2}$ as rational functions in $\Q[s_1,\dots,s_n]$
    \end{ex}
    \begin{sol}
        Starting with
        $$
            \sum_{n}T_1^{-1}=\frac{1}{T_1}+\cdots+\frac{1}{T_n}.
        $$
        We multiply by $1=\frac{s_n}{s_n}$ and simplify
        \begin{align*}
            \frac{s_n}{s_n}\sum_{n}T_1^{-1}&=\frac{T_1T_2\cdots T_n}{T_1T_2\cdots T_n}\left(\frac{1}{T_1}+\cdots+\frac{1}{T_n}\right)\\
            &=\frac{s_{n-1}}{s_n}
        \end{align*}

        For the second expression we present to approaches.
        \begin{enumerate}
            \item Observing that 
                $$\left(\sum_n T_1^{-1}\right)^2=\sum_{n} T_1^{-2}+2\sum_{n}T_1^{-1}T_2^{-1}$$
            we can write using the previous part
                $$ \sum_n T_1^{-2} = \frac{s_{n-1}^2}{s_n^2}-2\sum_{n}T_1^{-1}T_2^{-1}$$
            and multiplying by the second term by $\frac{s_{n}}{s_{n}}$ we get
                $$ \sum_n T_1^{-2} = \frac{s_{n-1}^2}{s_n^2} - 2\left(\frac{1}{T_1T_2}+\cdots+\frac{1}{T_{n-1}T_n}\right)\frac{T_1\cdots T_n}{T_1\cdots T_n}=\frac{s_{n-1}^2}{s_n^2} - 2\frac{s_{n-2}}{s_n}.$$
            Hence $\sum_n T_1^{-2}=\frac{s_{n-1}^2-2s_{n-2}s_n}{s_n^2}$.
            \item The second approach is slightly more involved. We start by multiplying by 1 in a clever (but different) way
                $$\left(\sum_n T_1^{-2}\right)\frac{s_n^2}{s_n^2}=\left(\frac{1}{T_1^2}+\cdots+\frac{1}{T_n^2}\right)\frac{T_1^2\cdots T_n^2}{T_1^2\cdots T_n^2}=\frac{\sum_n T_1^2\cdots T_{n-1}^2}{s_n^2}.$$
            Then $\sum_n T_1^2\cdots T_{n-1}^2$ is obviously (condescending much?) a symmetric polynomial and so we can use our trusty algorithm. Starting with
            \begin{align*}
                s_1^{2-2}s_2^{2-2}\cdots s_{n-1}^{2-0}&=s_{n-1}^2\\
                &=\left(\sum_n T_1\cdots T_{n-1}\right)^2\\
                &=\sum_n T_1^2\cdots T_{n-1}^2 + 2\sum_n T_1^2\cdots T_{n-2}^2T_{n-1}T_n.
            \end{align*}
            Moving to the second term
            \begin{align*}
                s_1^{2-2}\cdots s_{n-2}^{2-1}s_{n-1}^{1-1}s_n^1&=s_{n-2}s_n\\
                &=\left(\sum_n T_1\cdots T_{n-2}\right)T_1\cdots T_n\\
                &=\sum_n T_1^2\cdots T_{n-2}^2 T_{n-1}T_n
            \end{align*}
            and it follows that
            $$\sum_n T_1^2\cdots T_{n-1}^2 = s_{n-1}^2-2s_{n-2}s_n.$$
            So we conclude that
            $$ \sum_n T_1^{-2} = \frac{s_{n-1}^2-2s_{n-2}s_n}{s_n^2}$$
            which is reassuring.
        \end{enumerate}
        Note that in the first approach we stumbled upon something rather interesting:
        $$
            \sum_n T_1^{-1}\cdots T_k^{-1} = \frac{s_{n-k}}{s_n}
        $$
        the proof of which is left as an exercise to the reader.
    \end{sol}

    \begin{ex}{14.24}
        
    \end{ex}

\subsection{Field Extensions}
    \begin{ex}{21.18}
        Let $K\subset L$ be an algebraic extension. For $\alpha, \beta\in L$ prove that we have
        $$ \left[K(\alpha,\beta):K\right]\leq\left[K(\alpha):K\right]\cdot\left[K(\beta):K\right].$$

        Show that equality does not always hold. Does equality always hold if $[K(\alpha):K]$ and $[K(\beta):K]$ are relatively prime?
    \end{ex}
    \begin{proof}
        Let $f$ and $g$ be the minimal polynomials of $\al$ and $\be$ (respectively) in $K[x]$ and $f'$ be the minimal polynomial of $\alpha$ in $K(\beta)[x]$.
        If $\deg f'> \deg f$ then $f$ is a lower degree polynomial in $K(\beta)[x]$ with $f(\alpha)=0$ which is a contradiction. Hence $\deg f'\leq \deg f$ and so
        \begin{align*}
            \left[K(\alpha,\beta):K\right]&=\left[K(\al, \be):K(\be)\right]\cdot\left[K(\be):K\right]\\
            &=\deg f'\cdot \deg g\\
            &\leq\deg f\cdot \deg g\\
            &=\left[K(\al):K\right]\cdot \left[K(\be):K\right],   
        \end{align*}
        as desired.

        To show that equality does not always hold consider $\Q(\sqrt{2}, \sqrt[4]{2})$.
        Then $[\Q(\sqrt{2}):\Q]=2$ and $[\Q(\sqrt[4]{2}):\Q]=4$ but
        $$[\Q(\sqrt{2}, \sqrt[4]{2}):\Q]=4\cdot[\Q(\sqrt{2}, \sqrt[4]{2}):\Q(\sqrt[4]{2})]=4<8$$
        since $\left(\sqrt[4]{2}\right)^2=\sqrt{2}\in\Q(\sqrt[4]{2})$

        Lastly, suppose that $\deg f$ and $\deg g$ are relatively prime. Since
        \begin{align*}
            [K(\al,\be):K]&=[K(\al,\be), K(\al)]\cdot\deg f\\
            &=[K(\al,\be), K(\be)]\cdot\deg g
        \end{align*}
        it follows that $[K(\al,\be):K]$ is divisible by $\deg f$ and $\deg g$ and since they are relatively prime it is also divisible by $\deg f\cdot \deg g$.
        But we know that $[K(\al,\be):K]\leq\deg f\cdot\deg g$ and so $[K(\al,\be):K]=\deg f\cdot \deg g$.
    \end{proof}

    \begin{ex}{21.19}
        Let $K\subset K(\al)$ be an extension of odd degree. Prove that $K(\al^2)=K(\al)$.
    \end{ex}
    \begin{proof}
        Let $f$ be the minimal polynomial of $\al$ in $K[x]$. Then $\deg f=2n+1$ for some $n\in\Z_+$. 
        Since $\al^2\in K(\al)$ we get the tower $K(\al)/K\left(\al^2\right)/K$ and so\
        $$ \left[K(\al):K\right]=\left[K(\al):K\left(\al^2\right)\right]\cdot\left[K\left(\al^2\right):K\right].$$
        Let $g$ be the minimal polynomial of $\al$ in $K\left(\alpha^2\right)$. Then $\deg g\leq 2$ since $x^2-\al^2\in K\left(\alpha^2\right)$ is a polynomial with a root $\al$.
        Since $\left[K(\al):K\right]$ is odd, it is not divisible by two and so $\deg g = 1$. Hence $\left[K(\al):K\left(\al^2\right)\right]=1$ and it follows that $K(\al)=K\left(\al^2\right)$.
    \end{proof}

    \begin{ex}{21.23}
        Show that every quadratic extension of $\Q$ is of the form $\Q\left(\sqrt{d}\right)$ with $d\in\Z$.
        For what $d$ do we obtain the cyclotomic field $\Q(\zeta_3)$?
    \end{ex}
    \begin{proof}
        Let $K/\Q$ be a quadratic extension. Take $\al\in K\setminus\Q$. Then 
        $$ \Q\subset\Q(\al)\subset K $$
        and so
        $$2=\left[K:\Q\right]=\left[K:\Q(\al)\right]\left[\Q(\al):\Q\right].$$
        If $\left[\Q(\al):\Q\right]=1$ then $\Q(\al)=\Q$ and so $\al\in\Q$, which contradicts our assumption. 
        It follows that $\left[K:\Q(\al)\right]=1$ and so $K=\Q(\al)$. 
        Let 
        $$f(x)=x^2+a_1 x+a_0\in\Q[x]$$
        be the minimal polynomial of $\al$. 
        Let $d=\frac{a_1^2}{4}-a_0\in\Q$ and note that $a_0=-\al a_1-\al^2$. Then
        \begin{align*}
            \sqrt{d}&=\sqrt{\frac{a_1^2}{4}-a_0}\\
            &=\sqrt{\frac{a_1^2}{4}+a_1\al+\al^2}\\
            &=\frac{a_1+2\al}{2}.
        \end{align*}
        Hence $\sqrt{d}\in\Q(\al)$. By similar calculations we get $\al=\frac{2\sqrt{d}-a_1}{2}\in\Q(\sqrt{d})$.
        Hence $K=\Q(\al)=\Q(\sqrt{d})$. Of course, it is not yet the case the $d$ is an integer.
        Suppose that $d=\frac{p}{q}$. Since $\sqrt{d}=\frac{1}{q^2}\sqrt{qp}\in\Q(\sqrt{qp})$ we have
        $$K=\Q(\al)=\Q(\sqrt{d})=\Q(\sqrt{qp})$$
        with $qp\in\Z$ as desired.
    \end{proof}

    \begin{ex}{21.24}
        Is every cubic extension of $\Q$ of the form $\Q\left(\sqrt[3]{d}\right)$ for some $d\in\Q$?
    \end{ex}
    \begin{sol}
        No.
    \end{sol}

    \begin{ex}{21.26}
        Let $M=\Q(\al)=\Q(1+\sqrt{2}+\sqrt{3})$. Show that $M$ is of degree 4 over $\Q$, determine the minimal polynomial and write $\sqrt{2}$ and $\sqrt{3}$ in the basis $\{1,\al, \al^2,\al^3\}$.
        Also prove that the group $G=\text{Aut}_\Q(M)$ is isomorphic to $V_4$ and that $f^\al_\Q=\prod_{\sigma\in G}X-\sigma(\al)\in\Q[X]$.
    \end{ex}
    \begin{sol}
        Let $\be = \al-1=\sqrt{2}+\sqrt{3}$. Then clearly $M=\Q(\al)=\Q(\be)$. Let
        \begin{align*}
            f(x)&=(x-\sqrt{2}-\sqrt{3})(x+\sqrt{2}-\sqrt{3})(x-\sqrt{2}+\sqrt{3})(x+\sqrt{2}+\sqrt{3})\\
            &=x^4-10x^2+1\in\Q[x]
        \end{align*}
        and so $f(\be)=0$ by construction. 
        
        Is $f$ the minimal polynomial of $\be$ in $\Q[x]$? It is if we can prove that $[M:\Q]=4$.
        From
        $$ (\sqrt{2}+\sqrt{3})(\sqrt{3}-\sqrt{2})=1 $$
        It follows that $\be^{-1}=\sqrt{3}-\sqrt{2}$. Therefore
        $$ \sqrt{2}=\frac12(\be-\be^{-1})\quad\text{and}\quad\sqrt{3}=\frac12(\be+\be^{-1})$$
        and so $M=\Q(\sqrt{2}+\sqrt{3})=\Q(\sqrt{2},\sqrt{3})$. 
        Hence we have the towers $M/\Q(\sqrt{2})/Q$ and $M/\Q(\sqrt{3})/Q$. 
        Let $g(x)=x^2-3$. Suppose it is not the minimal polynomial of $\sqrt{3}$ in $\Q(\sqrt{2})$.
        Then there exists $a+b\sqrt{2}\in\Q(\sqrt{2})$ such that
        $$ 0 = g(a+b\sqrt{2})=a^2+2b^2-3+2ab\sqrt{2}.$$
        But since
        \begin{equation*}
            \begin{cases}
                a^2+2b^2-3=0\\
                2ab=0
            \end{cases}
        \end{equation*}
        has no solutions it follows that no such element exists.
        Therefore $g$ is the minimal polynomial of $\sqrt{3}$ and $[M:\Q(\sqrt{2})]=\deg g=2$.
        Since $x^2-2$ is the minimal polynomial of $\sqrt{2}$ in $\Q$ we conclude that 
        $$[M:\Q]=[M:\Q(\sqrt{2})]\cdot[\Q(\sqrt{2}):\Q)]=4$$ 
        and therefore $f$ is the minimal polynomial of $\be$.

        Thus $f(x-1)$ is the minimal polynomial of $\al$ in $\Q$. 
        From $f(\be)=0$ it follows that $1=\beta(10\beta-\beta^3)$ and so $\be^{-1}=10\beta-\beta^3$.
        Hence
        $$\sqrt{2}=\frac12\left(\be-\be^{-1}\right)=\frac12\left(\be-10\be+\be^3\right)=\frac12\left(-9(\al-1)+(\al-1)^3\right)$$
        and
        $$\sqrt{3}=\frac12\left(\be+\be^{-1}\right)=\frac12\left(11(\al-1)-(\al-1)^3\right)$$

        Let $G=\text{Aut}(M)$ and take $\sigma\in G$. Then by definition $\sigma(1)=1$ and it follows by induction and the properties of isomorphism that $\sigma(a)=a$ for all $a\in\Z$.
        Since $1=\sigma(1)=\sigma(a\cdot a^{-1})=\sigma(a)\cdot\sigma(a)^{-1}=a\cdot a^{-1}$ it also follows that $\sigma\left(\frac{p}{q}\right)=\frac{p}{q}$. 
        Hence $\sigma$ restricted to $\Q$ is simply the identity map. 
        Therefore $\sigma$ is completely determined by $\sigma(\sqrt{2})$ and $\sigma(\sqrt{3})$.
        Since $0=\sigma(0)=\sigma(\sqrt{2}^2-2)=\sigma(\sqrt{2})^2-2$ the only options are $\sigma(\sqrt{2})=\pm\sqrt{2}$.
        Similarly we conclude that $\sigma(\sqrt{3})=\pm\sqrt{3}$. This gives four possible automorphism.
        Take $\sigma,\tau\in G$ such that $\sigma(\sqrt{2})=-\sqrt{2}, \sigma(x)=x$ $\forall x\in M\setminus\{\sqrt{2}\}$ and $\tau(\sqrt{3})=-\sqrt{3},\tau(x)=x$ $\forall x\in M\setminus\{\sqrt{3}\}$. 
        Since 
        $$\sigma\circ\sigma=\tau\circ\tau=\sigma\circ\tau\circ\sigma\circ\tau=e$$
        where $e$ is the identity map it follows that $G$ is isomorphic to $V_4$, the Klein four-group.

        Lastly, consider
        \begin{align*}
            \tilde{f}&=\prod_{\sigma\in G}x-\sigma(\al)\\
            &=(x-1-\sqrt{2}-\sqrt{3})(x-1+\sqrt{2}-\sqrt{3})(x-1-\sqrt{2}+\sqrt{3})\\&\qquad\qquad (x-1+\sqrt{2}+\sqrt{3}).
        \end{align*}
        Hence $\tilde{f}(x)=f(x-1)$ which we already proved is the minimal polynomial of $\al$ in $\Q[x]$.

    \end{sol}

    \begin{ex}{21.29}
        Take $K=\Q(\al)$ with $f^\al_\Q=x^3+2x^2+1$.
        \begin{enumerate}
            \item Determine the inverse of $\al+1$ in the basis $\{1,\al,\al^2\}$ of $K$ over $\Q$.
            \item Determine the minimal polynomial of $\al^2$ over $\Q$.
        \end{enumerate}
    \end{ex}
    \begin{sol}
        ${}$
        \begin{enumerate}
            \item Since
                \begin{align*}
                    0&=\al^3+2\al^2+1\\
                    &=(\al+1)(\al^2+\al-1)+2.
                \end{align*}
                It follows that $(\al+1)^{-1}=-\frac12(\al^2+\al-1)$.
            \item From $\al^3+2\al^2+1=0$ it follows that $\al^3=-2\al^2-1$.
                Squaring both sides we get that $\al^6=4\al^4+4\al^2+1$ or alternatively
                $$\left(\al^2\right)^3-4\left(\al^2\right)^2-4\left(\al^2\right)-1=0.$$
                By Ex. 19 we know that $\Q(\al)=\Q(\al^2)$.
                Therefore the minimal polynomial of $\al^2$ over $\Q$ has degree 3 and it follows that
                $$ f^{\al^2}_\Q(x)=x^3-4x^2-4x-1. $$
        \end{enumerate}
    \end{sol}

    \begin{ex}{21.30}
        Define the cyclotomic field $\Q(\zeta_5)$ and let $\al=\zeta_5^2+\zeta_5^3$.
        \begin{enumerate}
            \item Show that $\Q(\al)$ is a quadratic extension of $\Q$ and determine $f^\al_\Q$.
            \item Prove: $\Q(\al)=\Q(\sqrt{5})$
            \item Prove: $\cos(2\pi/5)=\frac{\sqrt{5}-1}{4}$ and $\sin(2\pi/5)=\sqrt{\frac{5+\sqrt{5}}{8}}$
        \end{enumerate}
    \end{ex}
    \begin{proof}
        ${}$ 
        \begin{enumerate}
            \item The degree of the 5th cyclotomic polynomial
            $$\Phi_5(x)=\prod_{\substack{1\leq k\leq 5\\(k,5)=1}}\left(x-e^{\frac{2\pi k}{5}i}\right)=x^4+x^3+x^2+x+1$$
            is 4 and since $\Phi_5=f^{\zeta_5}_\Q$ it follows that $[\Q(\zeta_5):\Q]=4$.
            Thus 
            $$[\Q(\zeta_5):\Q(\alpha)]\mid4.$$ 
            Note that $\zeta_5^3=\frac{1}{\zeta_5^2}=\overline{\zeta_5^2}$. 
            Hence $\al=\zeta_5^2+\zeta_5^3=\zeta_5^2+\overline{\zeta_5^2}\in\R$ and so $\Q(\al)\subsetneq\Q(\zeta_5)$.
            Together with the fact that $\zeta_5$ is a root of $x^3+x^2-\al\in\Q(\al)$ it follows that
            $$1<[\Q(\zeta_5):\Q(\al)]\leq 3\implies [\Q(\zeta_5):\Q(\al)]=2.$$
            Finally, since $\Q(\zeta_5)/\Q(\al)/\Q$ is a tower of fields and
            $$[\Q(\al):\Q]=\frac{[\Q(\zeta_5):\Q]}{[\Q(\zeta_5):\Q(\al)]}=2$$
            it follows that $\Q(\al)$ is a quadratic extension.

            Let $w=\zeta_5^2$. Then $\al=w+\frac1w$ and $\Phi_5(w)=0$ by definition of $\Phi_5$. Since $w\neq0$ it follows that
            \begin{align*}
                0&=1+w+w^2+w^3+w^4\\
                0&=\frac{1}{w^2}+\frac1w+1+w+w^2\\
                0&=\left(w+\frac1w\right)^2+w+\dfrac1w-1\\
                0&=\al^2+\al-1.
            \end{align*}
            Since $x^2+x-1$ is monic polynomial of degree 2 we conclude that $f^\al_\Q=x^2+x-1$.
        \item By construction $\al$ is a root of $x^2+x-1$ and so
            $$\al\in\left\{\frac{-1\pm\sqrt{5}}{2}\right\}.$$
            Since we can write $\al$ as polynomial in $\sqrt{5}$ and vice versa it follows that $\Q(\al)=\Q(\sqrt{5})$.
        \item Let $\zeta_5=e^{\frac{6\pi}{5}i}$. Then $w=\zeta_5^2=e^{\frac{2\pi}{5}i}$ and 
            $$\cos\frac{2\pi}{5}=\frac{w+\overline{w}}{2}=\frac{\al}{2}.$$
            Thus $2\cos\frac{2\pi}{5}$ is a root of $f^\al_Q$. Since $\frac{2\pi}{5}$ is in the first quadrant, $\cos\frac{2\pi}{5}$ is positive and so
            $$\cos\frac{2\pi}{5}=\frac{-1+\sqrt{5}}{4}.$$
            Therefore we also have
            \begin{align*}
                \sin\frac{2\pi}{5}&=\sqrt{1-\cos^2\frac{2\pi}{5}}\\
                &=\sqrt{\frac{5+\sqrt{5}}{8}}.
            \end{align*}
        \end{enumerate}
    \end{proof}


    \begin{enumerate}
        \item The degree of the $p$th cyclotomic polynomial is $\varphi(p)=p-1$ and so
            $$4=\deg\Phi_5=\left[\Q(\zeta_5):\Q\right]=\left[\Q(\zeta_5):\Q(\al)\right]\cdot\left[\Q(\al):\Q\right]$$
            Since $\zeta_5$ is a root of $x^3+x^2-\al\in\Q(\al)[x]$ it follows that 
            $$\left[\Q(\zeta_5):\Q(\al)\right]\leq 3.$$
            Thus the minimal polynomial of $\al$ over $\Q$ is of degree 2 or 4.

            Let $f(x)=x^2+x-1\in\Q[x]$. Then
            \begin{align*}
                f(\al)&=\al^2+\al-1\\
                &=(\zeta_5^2+\zeta_5^3)^2+\zeta_5^2+\zeta_5^3-1\\
                &=\zeta_5^4+2+\zeta_5^6+\zeta_5^2+\zeta_5^3-1&&(\text{Since }\zeta_5^5=1)\\
                &=1+\zeta_5+\zeta_5^2+\zeta_5^3+\zeta_4^4.
            \end{align*}
            Since the 5th cyclotomic polynomial is given by
            $$\Phi_5(x)=\prod_{\substack{1\leq k\leq 5\\(k,5)=1}}\left(x-e^{\frac{2\pi k}{5}i}\right)=x^4+x^3+x^2+x+1$$
            and $\Phi_5(\zeta_5)=0$ by definition it follows that $f(\al)=0$.
            Since $f$ is monic and of degree $2$ it is the minimal polynomial of $\al$ over $\Q$.
            Hence $\Q(\al)$ is a quadratic extension.
        \item Since $\al$ is a root of $f(x)=x^2+x-1$ and the roots of $f$ are given by $\frac{-1\pm\sqrt{5}}{4}$
            it follows that $\Q(\al)=\Q(\sqrt{5})$.
        \item We know that
            $$\frac{-1\pm\sqrt{5}}{4}=\al=\zeta_5^2+\zeta_5^3=\cos\frac{4\pi k}{5}+\cos\frac{6\pi k}{5}+\left(\sin\frac{4\pi k}{5}+\sin\frac{6\pi k}{5}\right)i.$$
            and since
            $$\sin\frac{4\pi k}{5}+\sin\frac{6\pi k}{5}=2\cos\frac{2\pi k}{10}\sin\pi k= 0,\quad k\in\Z$$
            it follows that
            $$\frac{-1\pm\sqrt{5}}{4}=\cos\frac{4\pi k}{5}+\cos\frac{6\pi k}{5}.$$
    \end{enumerate}
\subsection{Finite Fields}

\subsection{Separable and Normal Extensions}