\subsection{Basis for a topology}
\begin{ex}{13.1}
    Let $X$ be a topological space; let $A\subset X$. Suppose that for each $x\in A$ there is an open set $U$ such that $x\in U\subset A$. Show that $A$ is open.
\end{ex}
\begin{proof}
    For every $x\in A$, let $U_x$ denote the open set containing $x$ such that $U_x\subset A$. Then $U = \bigcap_{x\in A} U_x\subset A$ since each $U_x$ is contained in $A$.
    For the other inclusion, take $x\in A$. Then $x\in U$ since $x$ is in $U_x$ by definition. Hence $A\subset U$ and it follows that $A=U$. 
    Since each $U_x$ and arbitrary unions of open sets are open it follows that $A$ is open.     
\end{proof}

\begin{ex}{13.4}
    ${}$
    \begin{enumerate}
        \item If $\{\Ta_\alpha\}$ is a family of topologies on $X$,how that $\bigcap\Ta_\alpha$ is a topology on $X$. Is $\bigcup\Ta_\alpha$ a topology on $X$?
        \item Let $\{\Ta_\alpha\}$ be a family of topologies on $X$. Show that there is a unique smallest topology containing all the all the collection of $\Ta_\alpha$ and a unique largest topology contained in all $\Ta_\alpha$
        \item If $X=\{a,b,c\}$ let
            $$
                \Ta_1 = \{\varnothing, X, \{a\}, \{a,b\}\}\quad\text{and}\quad\Ta_2=\{\varnothing, X,\{a\}, \{b,c\}\}.
            $$
            Find the smallest topology containing $\Ta_1$ and $\Ta_2$ and the largest topology contained in $\Ta_1$ and $\Ta_2$.
    \end{enumerate}
\end{ex}
\begin{sol}
    ${}$
    \begin{enumerate}
        \item (a) Since $\varnothing, X\in\Ta_\alpha$ for all $\alpha$ it follows that $\varnothing, X\in\bigcap\Ta_\alpha$. (b) If $U_\beta\in\bigcap\Ta_\alpha$, then $U_\beta\in\Ta_\alpha$ for all $\alpha$ and so $\bigcup U_\beta\in\Ta_\alpha$ for all $\alpha$ since $\Ta_\alpha$ is a topology. Hence $\bigcup U_\beta\in\bigcap\Ta_\alpha$. 
        (c) If $U_1,U_2\in\bigcap\Ta_\alpha$ then $U_1,U_2\in\Ta_\alpha$ for all $\alpha$ and so $U_1\cap U_2\in\Ta_\alpha$ for all $\alpha$. Therefore $U_1\cap U_2\in\bigcap \Ta_\alpha$. It follows by induction that $\bigcap\Ta_\alpha$ is closed under countable intersections. Hence an intersections of topologies is a topology.
        
        Let $X=\{a,b,c\}$. Then $\Ta_1=\{\varnothing, X, \{a\}\}$ and $\Ta_2=\{\varnothing, X, \{b\}\}$ are topologies on $X$. But $\Ta_1\cup\Ta_2=\{\varnothing, X, \{a\}, \{b\}\}$ is not a topology since $\{a\},\{b\}\in\Ta_1\cup\Ta_2$ but $\{a\}\cup\{b\}=\{a,b\}\not\in\Ta_1\cup\Ta_2$. Hence a union of topologies is, in general, not a topology. 
        
        \item Let $\Ss=\bigcup_\alpha \Ta_\alpha$. Then $X\in\Ss$ since $X$ is in each individual $\Ta_\alpha$ as they are all topologies. It follows that $X=\bigcup_{S\in\Ss} S$ and so $\Ss$ is a sub-basis. 
        Let $\Ba$ be the basis generated by $\Ss$ and $\Ta_s$ be the topology generated by $\Ba$. Fix some $\Ta_\alpha$ and take $U\in\Ta_\alpha$. Then $U\in\Ss\subset\Ba\subset\Ta_\Ss$ by construction.
        Hence $U\in\Ta_\Ss$ and it follows that $\Ta_\alpha\subset\Ta_\Ss$ for all $\alpha$. Is it the smallest topology with such property? Let $\Ta'$ be a topology on $X$ such that $\Ta_\alpha\subset\Ta'$ for all $\alpha$ and take $U\in\Ta_\Ss$.
        Then $U$ is an arbitrary union of finite intersections of elements of $\Ss=\bigcup\Ta_\alpha\subset\Ta'$. Since $\Ta'$ is a topology it is closed under arbitrary unions and finite intersections and so $U\in\Ta'$. Hence $\Ta_\Ss\subset\Ta'$ and it follows that $\Ta_\Ss$ is the smallest topology containing all $\Ta_\alpha$.

        From part one we know that $\bigcap\Ta_\alpha$ is a topology, and by definition it is contained in $\Ta_\alpha$ for all $\alpha$. 
        If $\Ta'\subset\Ta_\alpha,\,\forall\alpha$ is a topology, then for every $U\in\Ta'$, $U\in\bigcap\Ta_\alpha$ and so $\Ta'\subset\bigcap\Ta_\alpha$. Therefore $\bigcap\Ta_\alpha$ is the largest topology that is contained in all $\Ta_\alpha$.
        
        \item Apply part (2).
    \end{enumerate}
\end{sol}

\begin{ex}{13.5}
    Show that that topology $\Ta$ on $X$ generated by a basis $\Ba$ is equal to the intersections of all the topologies on $X$ that contain $\Ba$.
\end{ex}
\begin{proof}
    Let $T = \left\{\Ta_{\beta}\mid \Ba\subset\Ta_{\beta}\right\}$ be the collection of all topologies on $X$ that contain $\Ba$. Let $u\in\Ta$. Then $U$ can be written as a union of element in $\Ba$, i.e.
    $$
        U = \bigcup_{\alpha} B_\alpha,\quad B_\alpha\in\Ba
    $$
    Since $\Ta_\beta$ is a topology and $\Ba\subset\Ta_\beta$ for all $\Ta_\beta\in T$ it follows that $U= \bigcup_{\alpha} B_\alpha\in\Ta_\beta$ for all $\Ta_\beta\in T$ and so 
    $$\Ta\subset\bigcap_{\Ta_\beta\in T}\Ta_\beta.$$

    Since $\Ba\subset\Ta$ by definition of a basis it follows that $\Ta\in T$ and so 
    $$
    \bigcap_{\Ta_\beta\in T}\Ta_\beta\subset\Ta.
    $$
    Hence $\bigcap_{\Ta_\beta\in T}\Ta_\beta=\Ta$.
\end{proof}

\begin{ex}{13.7}
\end{ex}
\begin{sol}
    
\end{sol}

\begin{ex}{13.8}
    ${}$
    \begin{enumerate}
        \item Show that the collection 
        $$
            \Ba=\left\{(a,b)\mid a<b, a,b\in\Q \right\} 
        $$
        generates the standard topology on $\R$.
    
        \item Show that the collection
        $$
            \mathcal{C}=\left\{[a,b)\mid a<b, a,b\in\Q\right\} 
        $$
        generates a topology different from the lower limit topology.
    \end{enumerate}
\end{ex}
\begin{sol}
    ${}$
    \begin{enumerate}
        \item Let $\Ta$ be the standard topology on $\R$ and $\Ta_\Ba$ the topology generated by $\Ba$. 
        Let $x\in(a,b)\in\Ta$. Since the rationals are dense in $\R$, there exist $a', b'\in\Q$ such that $x\in(a',b')\subset(a,b)$. 
        Hence $\Ta\subset\Ta_\Ba$. The other inclusion is trivial since every basis element $(a,b)\in\Ba$ is a basis element of $\Ta$. We conclude that $\Ta=\Ta_\Ba$.

        \item Let $\Ta_c$ be the topology generated by $\mathcal{C}$ and let $\Ba$ be the basis of $\Ta_l$, the lower limit topology on $\R$. 
        Then for any $[a,b)\in\mathcal{C}$, $[a,b)\in\Ba$, and so $\Ta_\mathcal{C}\subset\Ta_l$. To show that this inclusion is strict we need to prove the statement
        \begin{align*}
            \lnot&\left(\forall B\in\Ba\,\forall x\in B\,\exists C\in\mathcal{C}: x\in C\subset B\right)\\
            \iff&\exists B\in\Ba\, \exists x\in B\,\forall C\in\mathcal{C}:x\not\in C\vee C\not\subset B\\
            \iff&\exists B\in\Ba\, \exists x\in B\,\forall C\in\mathcal{C}:x\in C\implies C\not\subset B
        \end{align*}
        Let $[x,b)\in\Ba$ with $x\not\in\Q$. Then $[a,c)\in\mathcal{C}$ can contain $x$ only if $a<x$ since $x$ is irrational. Therefore there is no element in $\mathcal{C}$ that contains $x$ and is a subset of $[x,b)$.
        This proves that $\Ta_\mathcal{C}\subsetneq\Ta_l$.
    \end{enumerate}
\end{sol}

\subsection{The Subspace Topology}
\begin{ex}{16.1}
    Show that if $Y$ is a subspace of $X$, and $A$ is a subset of $Y$ then the topology $A$ inherits as a subspace of $Y$ is the same as the topology it inherits as a subspace of $X$.
\end{ex}
\begin{proof}
    Let $\Ta$ be a topology on $X$, $\Ta_Y$ subspace topology on $Y$. Let $\Ta_A'$ be the topology $A$ inherits as a subset of $Y$.
    Then
    \begin{align*}
        \Ta_A'&=\left\{A\cap U\mid U\in\Ta_Y\right\}\\
        &=\big\{A\cap U\mid U\in\left\{Y\cap V\mid V\in \Ta\right\} \big\}\\
        &=\left\{A\cap U\mid U= Y\cap V, V\in\Ta\right\}\\
        &=\left\{A\cap\left(Y\cap V\right)\mid V\in\Ta\right\}\\
        &=\left\{\left(A\cap Y\right)\cap V\mid V\in\Ta\right\}\\
        &=\left\{A\cap V\mid V\in\Ta\right\}    
    \end{align*}
    which is by definition the topology $A$ inherits as a subset of $X$.
\end{proof}

\begin{ex}{16.3}
    
\end{ex}

\begin{ex}{16.4}
    Show that $\pi_1:X\times Y\to X$ is an open map.
\end{ex}
\begin{proof}
    Let $U$ be open in $X\times Y$ and take $(x,y)\in U$. Then there exists a basis element $B_x\times B_y$ such that $(x,y)\in B_x\times B_y\subset U$.
    For any $b\in B_x$, $(b,y)\in B_x\times B_y\subset U$ and so $b=\pi_1(b,y)\in \pi_1(U)$. It follows that $B_x\subset\pi_1(U)$.
    Since the basis of a product topology is the the product of open sets, $B_x$ is open in $X$ which means that for every $x\in\pi_1(U)$ there is an open set $B_x\in X$ such that $x\in B_x\subset \pi_1(U)$.
    From Exercise 13.1 it follows that $\pi_1(U)$ is open in $X$.    
\end{proof}

\subsection{Closed Sets and Limit Points}
\begin{ex}{17.1}
    Let $\mathcal{C}$ be a collection of subsets of $X$. Suppose that $X,\varnothing\in\mathcal{C}$ and that $\mathcal{C}$ is closed under finite unions and arbitrary intersections.
    Prove that
    $$
        \Ta=\{X-C\mid C\in\mathcal{C}\}
    $$
    is a topology.
\end{ex}
\begin{proof}
    Since $X-X=\varnothing$ and $X-\varnothing=X$ it follows that $X,\varnothing\in\Ta$. Let $U_\alpha$ be some collection of elements of $\Ta$.
    Then $U_\alpha = X-C_\alpha$ for some $C_\alpha\in\mathcal{C}$. Then
    $$\bigcup_\alpha U_\alpha = \bigcup_\alpha\left(X-C_\alpha\right)=X-\bigcap_\alpha C_\alpha\in\Ta $$
    Since $\mathcal{C}$ is closed under arbitrary intersections. Lastly let $U_i=X-C_i$, $1\leq i\leq n$ be a finite collection in $\Ta$. Then
    $$\bigcap_{i=1}^n U_i=X-\bigcup_{i=1}^n C_i\in\Ta.$$
    It follows that $\Ta$ is a topology on $X$.
\end{proof}

\begin{ex}{17.2}
    Show that if $A$ is closed in $Y$ and $Y$ is closed in $X$, then $A$ is closed in $X$.
\end{ex}
\begin{proof}
    Suppose that $A$ is closed in $Y$ and $Y$ is closed in $X$. Then $Y-A$ is open in $Y$ and so $Y-A=U\cap Y$ for some $U\subset X$ open.
    Hence $A = (X-U)\cap Y$. Since $X-U$ and $Y$ are closed in $A$ and arbitrary intersections of closed sets are closed, it follows that $A$ is closed in $X$.
\end{proof}

\begin{ex}{17.6}
    Let $A,B$ and $A_\alpha$ denote subsets of a space $X$. Prove the following
    \begin{enumerate}
        \item If $A\subset B$, then $\bar{A}\subset\bar{B}$
        \item $\overline{A\cup B}=\bar{A}\cup\bar{B}$
        \item $\overline{\bigcup_\alpha A_\alpha}\supset \bigcup_\alpha \overline{A_\alpha}$
    \end{enumerate}
\end{ex}
\begin{proof}
    ${}$
    \begin{enumerate}
        \item Suppose that $A\subset B$. Let $x\in\overline{A}$. Then for every neighborhood $U$ containing $x$ intersects $A$. 
            Hence $U\cap A$ is non empty, so take $y\in U\cap A$. Then $y\in U\cap B$ since $A\subset B$ and it follows that every neighborhood of $x$ intersects $B$.
            Therefore $x\in\overline{B}$ and the result follows.
        \item Let $x\in \overline{A\cup B}$. Then every neighborhood $U$ of $x$ intersects $A\cup B$. Since $A\subset\bar{A}$ and $B\subset\bar{B}$ it follows that $A\cup B\subset\bar{A}\cup\bar{B}$ and so $U$ intersects $\bar{A}\cup\bar{B}$.
            Hence $x\in\bar{A}\cup\bar{B}$ and so $\overline{A\cup B}\subset\bar{A}\cup\bar{B}$.

            Let $x\in \bar{A}\cup\bar{B}$. Then $x$ is in $\bar{A}$ or $\bar{B}$ and every neighborhood $U$ of $x$ intersects either $A$ or $B$. Therefore $U$ intersects $A\cup B$ and it follows that $x\in \overline{A\cup B}$.
            Therefore $\bar{A}\cup\bar{B}\subset\overline{A\cup B}$ and equality follows.
        \item Let $x\in\bigcup_\alpha\overline{A_\alpha}$. Then $x\in\overline{A_\alpha}$ for at least one $\alpha$. Hence every neighborhood $U$ of $x$ intersects $A_\alpha$ and so $U$ intersects $\bigcup_\alpha A_\alpha$. 
            Hence $x\in\overline{\bigcup_\alpha A_\alpha}$ and so $\bigcup_\alpha \overline{A_\alpha}\subset\overline{\bigcup_\alpha A_\alpha}$.

            To show that the other inclusion doesn't hold in general let $A_\alpha=\left(\frac1\alpha, 1\right)$.
            Then 
            $$\bigcup_{\alpha=1}^\infty\overline{A_\alpha}=\bigcup_{\alpha=1}^\infty\overline{\left(\frac1\alpha, 1\right)}=\bigcup_{\alpha=1}^\infty\left[\frac1\alpha, 1\right]=(0, 1]$$
            and
            $$\overline{\bigcup_{\alpha=1}^\infty A_\alpha}=\overline{(0,1)}=[0,1].$$
    \end{enumerate}
\end{proof}

\begin{ex}{17.8}
    
\end{ex}

\begin{ex}{17.11}
    Show that the product of two Hausdorff spaces is Hausdorff
\end{ex}
\begin{proof}
    Let $X,Y$ be Hausdorff spaces and take $(x_1,y_1),(x_2, y_2)\in X\times Y$.
    Then there are open subsets $U_1,U_2$ in $X$ such that $x_1\in U_1$, $x_2\in U_2$ and $U_1\cap U_2=\varnothing$.
    Similarly there exists $V_1,V_2\subset Y$ that are open in $Y$ with $y_1\in V_1, y_2\in V_2$ and $V_1\cap V_2=\varnothing$.
    Then $(x_1,y_1)\in U_1\times V_1, (x_2,y_2)\in U_2\times V_2$ and
    $$\left(U_1\times V_1\right)\cap\left(U_2\times V_2\right)=\left(U_1\cap U_2\right)\times\left(V_1\cap V_2\right)=\varnothing\times\varnothing=\varnothing.$$
    Since $U_1\times V_1$ and $U_2\times V_2$ are open it follows that $X\times Y$ is a Hausdorff space.
\end{proof}

\begin{ex}{17.12}
    Show that a subspace of Hausdorff space is Hausdorff.
\end{ex}
\begin{proof}
    Let $X$ be a Hausdorff space and $Y\subset X$. Then for every $x_1, x_2\in Y$ there exists neighborhoods $U_1$ and $U_2$ of $x_1$ and $x_2$ (respectively) that are disjoint.
    Then $U_1\cap Y$ and $U_2\cap Y$ are open in $Y$, contain $x_1$ and $x_2$ (respectively) and are clearly disjoint. Hence $Y$ is Hausdorff.
\end{proof}

\begin{ex}{17.13}
    Show that $X$ is Hausdorff if and only if $\Delta = \{(x,x)\mid x\in X\}$ is closed in $X\times X$.
\end{ex}
\begin{proof}
    Suppose $X$ is Hausdorff and consider $(x,y)\not\in\Delta$. Since $X$ is Hausdorff, there exists disjoint neighborhoods $U$ and $V$ that contain $x$ and $y$ (respectively).
    Then take $(x', y')\in U\times V$. Since $U\cap V=\varnothing$ it follows that $x'\neq y'$ and so $(x',y')\not\in\Delta$.
    Since $U\times V$ is open in $X\times X$ it follows that $(x,y)$ is not a limit point of $\Delta$. Hence $\Delta$ contains all of its limit points and so it is closed.

    Now suppose that $\Delta$ is closed in $X\times X$ and consider $(x,y)\not\in\Delta$. Since $\Delta$ is closed, it contain all of its limit points and so $(x,y)$ is not a limit point of $\Delta$. 
    Hence there exists a neighborhood $T$ of $(x,y)$ that does not intersect $\Delta$. Then there is a basis element $U\times V$ such that $(x,y)\in U\times V\subset T$.
    Since $U\times V$ does not intersect $\Delta$ it follows that for every $(x', y')\in U\times V$ $x'\neq y'$. Hence $U\cap V=\varnothing$. 
    It follows that for every $x$ and $y$ in $X$ there exists disjoint neighborhoods $U$ and $V$ that contain $x$ and $y$ respectively. Therefore $X$ is Hausdorff.
\end{proof}

\begin{ex}{17.14}
    In the finite complement topology on $\R$, to what point or points does the sequence $x_n=\frac1n$ converge?
\end{ex}
\begin{sol}
    This sequence converges to every point in $\R$! To see why, suppose there exists $a\in\R$ that the sequence does not converge to.
    Then there exists an open neighborhood $U$ of $a$ such that for all $N\in\N$ there exists an $n\geq N$ for which $x_n\not\in U$.
    There has to be an infinite number of such $x_n$, for otherwise we could just take $N'$ bigger than the largest $n$. But then $\R-U$ is not finite nor empty, so it must be all of $\R$.
    Therefore $U=\varnothing$ which is a contradiction since we assumed that $a\in U$.
\end{sol}

\begin{ex}{17.19}
    For $A\subset X$, the \textit{boundary} of $A$ is 
    $$\Bd A=\overline{A}\cap\overline{\left(X-A\right)}.$$
    Show that
    \begin{enumerate}
        \item $\Int A\cap \Bd A=\varnothing$ and $\overline{A}=\Int A\cup \Bd A$.
        \item $\Bd A=\varnothing\iff A$ is both open and closed.
        \item $U$ is open $\iff\Bd U = \overline{U}-U$
        \item If $U$ is open, is it true that $U=\Int\overline{U}$? Justify your answer. 
    \end{enumerate}    
\end{ex}

\begin{proof}
   ${}$
   \begin{enumerate}
       \item Let $x\in\Int A$.  Then there is a neighborhood $U$ of $x$ that is contained in $A$, and so $U$ does not intersect $X-A$.
            Hence $x$ is not a limit of point of $X-A$ and so $x\not\in \overline{X-A}$. Therefore $x$ is not in $\overline{A}\cap\overline{\left(X-A\right)}$, and so $\Int A\cap\Bd A=\varnothing$.

            Let $x\in X$. If $x\in\Int A$ then clearly $x\in\Int A\cup \overline{\left(X-A\right)}$.
            So suppose $x\not\in\Int A$. Then $x\in X-\Int A\subset\overline{X-\Int A}$.
            Consider any neighborhood $U$ containing $x$. Since $x$ is not in the interior of $A$, $U$ is not a subset of $A$, hence $U$ intersects $X-A$.
            It follows that $x$ is a limit point of $X-A$ and so $x\in\overline{X-A}$. Therefore $X=\Int A\cup\overline{\left(X-A\right)}$ and it follows that
            \begin{align*}
                \Int A\cup \Bd A&=\Int A\cup\left(\overline{A}\cap\overline{\left(X-A\right)}\right)\\
                &=\left(\Int A\cup \overline{A}\right)\cap\left(\Int A\cup\overline{\left(X-A\right)}\right)\\
                &=\overline{A}\cap\left(\Int A\cup\overline{\left(X-A\right)}\right)\\
                &=\overline{A}\cap X\\
                &=\overline{A}.
            \end{align*}
        \item Suppose $\Bd A=\varnothing$. Then 
                $$\overline{A}=\Int A\cup\Bd A=\Int A.$$
            Since $\Int A\subset A\subset\overline{A}$ it follows that $A=\Int A=\overline{A}$ and so $A$ is both close and open.

            Now suppose that $A$ is both closed and open. Then $A=\Int A$ and $A=\overline{A}$ so $\Int{A}=\overline{A}$.
            From $\overline{A}=\Int A\cup \Bd A$ it follows that $\Bd A\subset \overline{A}$ but 
                $$\varnothing=\Int A\cap \Bd A= \overline{A}\cap \Bd A.$$
            Hence $\Bd A=\varnothing$.
        \item Suppose $U$ is open. Then $U=\Int U$ and $\overline{U}=U\cup \Bd U$. Since $U\cap\Bd U=\varnothing$ it follows that $\Bd U =\overline{U}-U$.
        
            Now suppose that $\Bd U = \overline{U}-U$ and take $x\in U$. Then $x\in\overline{U}$ and
                $$x\not\in \overline{U}-U=\Bd U=\overline{U}\cap\overline{\left(X-U\right)}.$$
            So $x\not\in\overline{\left(X-U\right)}$. Then there exists a neighborhood $V$ of $x$ that does not intersect $X-U$.
            Since for every $y\in V\implies y\not\in X-U\implies y\in U$ it follows that $V\subset U$.
            We showed that for every $x\in U$ there exists an open neighborhood $V\subset U$ that contains $x$. Therefore $U$ is open.
        \item No. Consider $U=(0,1)\cup(1,2)$ which is open in the standard topology on $\R$.
            Then $\overline{U}=[0,2]$ and $\Int\overline{U}=(0,2)\neq U$.
   \end{enumerate} 
\end{proof}
\subsection{Continuous Functions}
\begin{ex}{18.3}
    Let $X$ and $X'$ denote a single set in two topologies, $\Ta$ and $\Ta'$ respectively.
    Let $i:X'\to X$ be the identity map.
    \begin{enumerate}
        \item Show that $i$ is continuous $\iff\Ta\subset\Ta'$.
        \item Show that $i$ is a homeomorphism $\iff\Ta=\Ta'$.
    \end{enumerate}    
\end{ex}
\begin{proof}
    ${}$
    \begin{enumerate}
        \item Suppose $i$ is continuous. Let $B$ be a basis element in $X$ and take $x\in X$. 
            Then $B=i^{-1}(B)$ is open in $X'$. So there exists a basis element $B'\subset X$ such that $x\in B'\subset B$.
            It follows that $\Ta'$ is finer than $\Ta$.

            Now suppose that $\Ta\subset\Ta'$. Then for any $U\in\Ta$, $U\in\Ta'$. 
            This is equivalent to saying that for any open set $U$ in $X$, $i^{-1}(U)=U$ is open in $X'$.
            Therefore $i$ is continuous.
        \item $i$ is a homeomorphism if and only if $i$ and $i^{-1}$ are continuous.
            By the result above, $i$ is continuous if and only if $\Ta\subset\Ta'$. 
            Similarly, $i^{-1}$ is continuous if and only if $\Ta'\subset\Ta$. 
            Therefore $i$ is a homeomorphism if and only if $\Ta'=\Ta$.
    \end{enumerate}
\end{proof}
