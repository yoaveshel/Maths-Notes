\subsection{The Quotient Topology}
\begin{ex}{22.2}
    ${}$
    \begin{enumerate}
        \item Let $p:X\to Y$ be a continuous map. Show that if there is a continuous map $f:Y\to X$ such that $p\circ f$ is the identity map on $Y$, then $p$ is a quotient map.
        \item If $A\subset X$, a retraction of $X$ onto $A$ is a continuous map $r:X\to A$ such that $r(a)=a$ for all $a\in A$. Show that a retraction in a quotient map.
    \end{enumerate}
\end{ex}
\begin{proof}
    ${}$
    \begin{enumerate}
        \item Let $y\in Y$. Then $(p\circ f)(y)=p\left(f(y)\right)=y$ and so $p$ is surjective. 
            Let $U$ be an open subset of $Y$. Then $p^{-1}(U)$ is open in $X$ since $p$ is continuous.
            Now suppose that $U$ is a subset of $Y$ such that $p^{-1}(U)$ is open in $X$. Then $f^{-1}\left(p^{-1}(U)\right)=(p\circ f)^{-1}(U)=U$ since $f$ is continuous.
            It follows that $p$ is a quotient map.
        \item Let $i:A\to X$ be given by $i(a)=a$ for each $a\in A$. Let $U$ be open in $X$. Then $i^{-1}(U)=U\cap A$ is open in $A$ by definition of subspace topology. Therefore $i$ is continuous.
            Since $r\circ i$ is the identity on $A$ it follows by part (a) the $r$ is a quotient map.
    \end{enumerate}
\end{proof}

\begin{ex}{22.3}
    Let $\pi_1:\R\times\R\to\R$ be projection on the first coordinate. 
    Let $A$ be the a subspace of $\R\times\R$ consisting of all points $(x,y)$ for which either $x\geq 0$ or $y=0$.
    Let $q:A\to\R$ be obtained by restricting $\pi_1$. Show that $q$ is a quotient map that is neither closed not open.
\end{ex}
\begin{proof}
    Let $x\in\R$. Then $(x,0)\in A$ and $q\left((x,0)\right)=x$. Since for $U$ open in $\R$, $\pi^{-1}(U)$ is open in $\R\times\R$ it follows that $q^{-1}(U)=\pi^{-1}(U)\cap A$ is open in $A$.
    Therefore $q$ is continuous. Let $i:\R\to\R\times\R$ be given by $x\mapsto(x,0)$. 
    Then for $U\times V$ open in $\R\times\R$, we have that
    $$i^{-1}(U\times V)=\begin{cases}
        U,&\text{if } 0\in V\\
        \varnothing,&\text{otherwise}\\
    \end{cases}$$
    is open in $\R$ and so $i$ is continuous. 
    Since $i(\R)\subset A$ it follows that $i':\R\to A$ is continuous. 
    Since $(q\circ i')(x)=q((x,0))=x$ is the identity on $\R$ it follows by Exercise 2 that $q$ is a quotient map.
    
    Consider the set 
    $$U = (-1,1)\times (1, 2)\cap A=[0,1)\times(1,2)$$
    which is open in $A$ since $(-1,1)\times (1, 2)$ is open in $\R\times\R$. 
    Then $q(U)=[0,1)$ which is not open in $\R$ since any neighborhood containing 0 is not contained in $[0,1)$.
    Next consider $f:\R\times\R\to\R$ given by $f(x,y)=xy$. Since $f$ is continuous, $f^{-1}(\{1\})$ is closed and so
    $$C=f^{-1}(\{1\})\cap A=\{(x,1/x)\mid x>0\}$$
    is closed in $X$. But $q(C)=(0,\infty)$ is not closed in $\R$. Therefore $q$ is a quotient map the is neither open nor closed.
\end{proof}

\begin{ex}{22.4}
    ${}$
    \begin{enumerate}
        \item Define an equivalence relation on the plane $X=\R^2$ as follows:
            $$x_0\times y_0\sim x_1\times y_1\quad\text{if}\quad x_0+y_0^2=x_1+y_1^2$$
            Let $X^*$ be the corresponding quotient space. It is homeomorphic to a familiar space, what is it?
        \item Repeat (a) for the equivalence relation 
            $$x_0\times y_0\sim x_1\times y_1\quad\text{if}\quad x_0^2+y_0^2=x_1^2+y_1^2.$$
    \end{enumerate}
\end{ex}
\begin{sol}
    ${}$
    \begin{enumerate}
        \item Let
        \begin{align*}
            g:\R^2&\to\R\\
            (x,y)&\mapsto x+y^2.
        \end{align*}
        If $g$ is continuous and surjective quotient map, then it would induce a homeomorphism from
        $$X^*=\{g^{-1}(\{a\})\mid a\in\R\}=\{(x,y)\mid x+y^2=a, a\in\R\}$$
        to $\R$. Let $(a,b)\subset\R$. Then
        $$U = g^{-1}\big((a,b)\big)=\{(x,y)\mid x+y^2\in (a,b)\}.$$
        Then $U$ is the region between the parabolas $x=a-y^2$ and $x=b-y^2$. 
        Then for any point $(x,y)\in U$ we can find the minimum distance to each parabola and take an $\varepsilon$-neighborhood smaller than that.
        Therefore $U$ is open and so $g$ is continuous. Let $h:\R\to\R^2$ be given by $h(x)=(x,0)$. 
        Then $h$ is continuous and $(g\circ h)(x)=g((x,0))=x$ is the identity on $\R$.
        Therefore $g$ is a quotient map and so $X^*$ is homeomorphic to $\R$.

        \item
    \end{enumerate}
\end{sol}

\begin{ex}{22.5}
    Let $p:X\to Y$ be an open map. Show that if $A$ is open in $X$, then the map $q:A\to p(A)$ obtained by restricting $p$ is an open map.
\end{ex}
\begin{proof}
    Let $U$ be open in $A$. Then $U$ is open in $X$ since $A$ is open in $X$ and so $p(U)$ is open $Y$. Then $p(U)\cap p(A)=p(U)$ is open in $p(A)$ and so $q$ is an open map.
\end{proof}
