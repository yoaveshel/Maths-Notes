\subsection{Metric Spaces}
\begin{ex}{20.1a}
    In $\R^n$ let
    $$d'(\mathbf{x},\mathbf{y})=\sum_{i=1}^n \abs{x_i-y_i}.$$
    Show that $d'$ is a metric that induces the usual topology on $\R^n$.
\end{ex}
\begin{proof}
    Clearly $d'$ define a metric. Let $\varepsilon>0$. We will show that
    \begin{equation}\label{eq:ball-inclusion}
        B_d\left(\mathbf{x},\tfrac{\varepsilon}{\sqrt{n}}\right)\subset B_{d'}(\mathbf{x},\varepsilon)\subset B_d(\mathbf{x},\mathbf{y})
    \end{equation}
    from which the result would follow.
    Recall the inequality
    \begin{equation}\label{eq:sqrt-inequality}
        \sqrt{a_1+\cdots+a_n}\leq\sqrt{a_1}+\cdots+\sqrt{a_n}\leq\sqrt{n\left(a_1+\cdots+a_n\right)},
    \end{equation}
    for $a_i\geq 1$. Let $\mathbf{y}\in B_{d'}(\mathbf{x},\varepsilon)$. Then
    \begin{align*}
        d(\mathbf{x},\mathbf{y})&=\left(\sum_{i=1}^n(x_i-y_i)^2\right)^\frac12\\
        &\leq\sum_{i=1}^n\sqrt{\left(x_i-y_i\right)^2}&&(\text{By (\ref{eq:sqrt-inequality})})\\
        &=\sum_{i=1}^n \abs{x_i-y_i}\\
        &=d'(\mathbf{x},\mathbf{y})\\
        &<\varepsilon
    \end{align*}
    which proves the right inequality of (\ref{eq:ball-inclusion}). Now let $\mathbf{y}\in B_d\left(\mathbf{x},\tfrac{\varepsilon}{\sqrt{n}}\right)$.
    Then using the left right inequality of (\ref{eq:sqrt-inequality}) we get
    \begin{align*}
        d'(\mathbf{x}, \mathbf{y})&=\sum_{i=1}^n \abs{x_i-y_i}\\
        &=\sum_{i=1}^n\sqrt{\left(x_i-y_i\right)^2}\\
        &\leq\sqrt{n}\left(\sum_{i=1}^n(x_i-y_i)^2\right)^\frac12\\
        &=\sqrt{n}d(\mathbf{x},\mathbf{y})\\
        &<\sqrt{n}\frac{\varepsilon}{\sqrt{n}}=\varepsilon
    \end{align*}
    which proves the left inequality of (\ref{eq:ball-inclusion}) and completes the proof.
\end{proof}

\begin{ex}{20.3}
    Let $X$ be a metric space with a metric $d$.
    \begin{enumerate}
        \item Show that $d:X\times X\to\R$ is continuous.
        \item Let $X'$ denote a space having the same underlying set as $X$. Show that if $d:X'\times X'\to\R$ is continuous, then the topology of $X'$ is finer than the topology of $X$.
    \end{enumerate}
\end{ex}
\begin{proof}
    ${}$
    \begin{enumerate}
        \item Let $\varepsilon>0$ and $(x,y)\in X\times X$. Then for $\delta=\varepsilon/2$ the open set $U =B(x,\delta)\times B(y,\delta)$ is a neighborhood of $(x,y)$.
            Let $d=d(x,y)$ and consider $(d-\varepsilon, d+\varepsilon)$ in $\R$. 
            Then for $(x',y')\in U$
            \begin{align*}
                d(x',y')&\leq d(x',x)+d(x,y)+d(y,y')\\
                < 2\delta+ d\\
                < \varepsilon+d
            \end{align*}
            and
            \begin{align*}
                d(x,y)\leq d(x',x)+d(x',y')+d(y,y')\\
                <d(x',y')+\varepsilon.
            \end{align*}
            Therefore $d(x',y')\in(d-\varepsilon,d+\varepsilon)$ for all $(x',y')\in X\times X$ and so $d(U)\subset (d-\varepsilon,d+\varepsilon)$.
            Since for any open set $V$ in $\R$ containing $d(x,y)$ there exists a neighborhood $U$ in $X\times X$ such that $d(U)\subset V$ it follows that $d$ is continuous.
        \item  Suppose $d:X'\times X'\to \R$ is continuous. Let $\varepsilon>0$ and consider
            $$ U = d^{-1}\left((-\infty,\varepsilon)\right)=\{(x,y)\mid d(x,y)<\varepsilon\}$$
            which is open in $X'\times X'$ since $d$ is continuous. Let $x\in X$ and $y\in B(x,\varepsilon)$.
            Since $d(x,y)<\varepsilon$ it follows that $(x,y)\in U$ and so there exists a basis $V_1\times V_2$ element of $X'\times X'$ such
            $$ (x,y)\in V_1\times V_2\subset U.$$
            Where $V_2$ is an open neighborhood in $X'$ containing $y$.
            Since for any $z\in V_2$, $(x,z)\in U$ it follows that $d(x,z)<\varepsilon$ and so $z\in B(x,\epsilon)$. Therefore $V_2\subset B(x,\epsilon)$.
            Since for all $y$ there exists an open neighborhood $V\subset B(x,\varepsilon)$ in $X'$ containing $y$ and it follows that $X\subset X'$.
    \end{enumerate}
\end{proof}

\begin{ex}{20.11}
    Show that if $d$ is a metric for $X$, then 
    $$d'(x,y)=\frac{d(x,y)}{1+d(x,y)}$$
    is a bounded metric that gives the topology of $X$.
\end{ex}
\begin{proof}
    We start by proving the $d'$ is a metric. Let
    $$f(x)=\frac{x}{1+x}.$$
    Then $d'(x,y)=(f\circ d)(x,y)$. Since $f:\R_+\to[0,1)$ is one-to-one, increasing and $f(x)=0$ iff $x=0$ the first two properties of a metric follow immediately.
    For the triangle inequality, let $a,b\geq 0$. Then
    $$f(a+b)-f(b)=\frac{a}{(1+a+b)(1+b)}\leq\frac{a}{1+a}=f(a)$$
    and so
    \begin{align*}
        d'(x,y)&=f(d(x,y))\\
        &\leq f(d(x,z)+d(z,y))\\
        &\leq f(d(x,z))+f(d(z,y))\\
        &\leq d'(x,z)+d'(z,y).
    \end{align*}
    Therefore $d'$ is a metric.

    Let $X'$ be the space generated by the metric $d'$. Since $d:X\times X\to[0,\infty)$ and $f:[0,\infty)\to[0,\infty)$ are continuous
    and $d'=f\circ d$ it follows that $d':X\times X\to[0,\infty)$ is continuous and so by Exercise 3, $X$ is finer than $X'$.

    To show that $X'$ is finer than $X$, let $\varepsilon>0$ and $\delta=\frac{\epsilon}{1+\epsilon}$.
    Then for $y\in B_{d'}(x,\delta)$, we have
    $$\frac{d(x,y)}{1+d(x,y)}<\delta$$
    and so $d(x,y)<\frac{\delta}{1-\delta}=\varepsilon$. Hence $B_{d'}(x,\delta)\subset B_{d}(x,\varepsilon)$ and so $X'$ is finer than $X$.
    It follows that $d'$ generates the same topology as $d$.

    
    
\end{proof}