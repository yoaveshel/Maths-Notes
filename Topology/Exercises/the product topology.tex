\subsection{The Product Topology}
\begin{ex}{19.3}
    If each space $X_\alpha$ is Hausdorff, then $X =\prod_\alpha X_\alpha$ is Hausdorff 
    in both the box and product topology.
\end{ex}
\begin{proof}
    Let $\mathbf{x}, \mathbf{y}\in X$ such that $\mathbf{x}\neq\mathbf{y}$. 
    Then $x_\beta\neq y_\beta$ for at least one $\beta\in J$.
    Then there exists disjoint open neighborhoods $W_x$ and $W_y$ around $x_\beta$ 
    and $y_\beta$ in  $X_\beta$. Let
    $$
    U_\alpha=
    \begin{cases}
        W_x,&\alpha=\beta\\
        X_\alpha,&\alpha\neq\beta\\
    \end{cases}
    \quad\text{and}\quad 
    V_\alpha= 
    \begin{cases}
        W_y,&\alpha=\beta\\
        X_\alpha,&\alpha\neq\beta\\
    \end{cases}.
    $$
    Then $\prod_\alpha U_\alpha$ and $\prod_\alpha V_\alpha$ are disjoint and open in both the 
    product and box topologies
    (open in the box topology since they are a product of open sets and in the product topology 
    since they have finitely many components that are not all of $X_\alpha$).
    To prove that they are disjoint, suppose there exists 
    $\mathbf{z}\in\left(\prod_\alpha U_\alpha\right)\cap\left(\prod_\alpha V_\alpha\right)$.
    Then $z_\beta\in U_\beta\cap V_\beta=W_x\cap W_y=\varnothing$.
    It follows that $X$ is Hausdorff.
\end{proof}

\begin{ex}{19.7}
    Let $\R^\infty$ be a subset of $\R^\omega$ consisting of all sequences that are 
    "eventually zero", that is, all sequences $(x_1,x_2,\dots)$ such that $x_i=0$ for 
    all $i\geq n$ for some $n\in\N$.
    What is the closure of $\R^\infty$ in $\R^\omega$ in the box and product topologies?
\end{ex}
\begin{sol}
    We start with the box topology. Let $\mathbf{x}\in\R^\omega\setminus\R^\infty$. 
    Then let $J$ be an index set such that $x_j\neq 0$ for all $j\in J$. 
    Note that $J$ is infinite since $\mathbf{x}\not\in\R^\infty$.
    Then there exists an open neighborhood $U_j$ around $x_j$ that does not contain 0.
    Let $U=\prod_{i\in N} V_i$ where
    $$V_i=\begin{cases}
        U_i,& i\in J\\
        A_i\text{ open in $\R$ and } x_i\in A_i,&i\not\in j
    \end{cases}.$$
    Since for any $(y_1,y_2,\dots)\in U$, $y_i\neq 0$ for infinitely many $i$, it follows that 
    $(y_1,y_2,\dots)\not\in\R^\infty$ and so $U\cap\R^\infty=\varnothing$.
    Therefore for any $\mathbf{x}\in\R^\omega\setminus\R^\infty$ there exists a neighborhood $U$ 
    containing $\mathbf{x}$ such that $U\subset\R^\omega\setminus\R^\infty$ and it follows that 
    $\R^\omega\setminus\R^\infty$ is open and so $\R^\infty$ is closed.
    Therefore $\overline{\R^\infty}=\R^\infty$ in the box topology.

    We now consider the product topology. 
    We will show that $\Int \R^\omega\setminus\R^\infty =\varnothing$ and so $\R^\infty$ is dense in $\R^\omega$.
    Let $U$ be an basis element in $\R^\omega$ and $J$ an index set such that $U_j\subsetneq\R$ for all $j\in J$.
    Then $J$ is finite by definition of the product topology. 
    So there exists a sequence $\mathbf{y}=(y_1,y_2,\dots)$ in $\R^\infty$ such that $y_i\in U_i$ for $i\in J$ and $y_i=0$ otherwise.
    Then $\mathbf{y}\in U$. Therefore there is no open set contained in $\R^\omega\setminus\R^\infty$ and it follows that $\Int \R^\omega\setminus\R^\infty =\varnothing$.
    Hence $\R^\infty$ is dense in $\R^\omega$ and so $\overline{\R^\infty}=\R^\omega$. 
\end{sol}

\begin{ex}{19.10}
    Let $A$ be a set; let $\left\{X_\alpha\right\}_{\alpha\in J}$ be an indexed family of spaces; 
    and let $\{f_\alpha\}$ be an indexed family of functions $f_\alpha:A\to X_\alpha$.
    \begin{enumerate}
        \item Show that there is a unique coarsest topology $\Ta$ on $A$ relative to which each of the 
        functions $f_\alpha$ is continuous.
        \item Let
            $$\Ss_\beta =\left\{f^{-1}_\beta\left(U_\beta\right)\mid U_\beta \text{ is open in } X_\beta\right\}$$
            and let $S=\bigcup_\alpha \Ss_\alpha$. Show that $\Ss$ is a subbasis for $\Ta$.
        \item Show that a map $g:Y\to A$ is continuous relative to $\Ta$ if and only if each map 
        $f_\alpha\circ g$ is continuous. 
        \item Let $f:A\to\prod X_\alpha$ be defined by
            $$ f(a)=\left(f_\alpha(a)\right)_{\alpha\in J}.$$
            Let $Z$ denote the subspace $f(A)$  of the product space $\prod X_\alpha$. 
            Show that the image under $f$ of each element of $\Ta$ is an open set of $Z$.
    \end{enumerate}
\end{ex}
\begin{sol}
    ${}$
    \begin{enumerate}
        \item Let $\mathcal{C}$ be the collection of all topologies on $A$ such that each $f_\alpha$ is 
        continuous.
            This collection is not empty since every function is continuous relative to $\mathcal{P}(A)$. 
            Let $\Ta=\bigcap_{\Ta'\in\mathcal{C}}\Ta'$.
            By Exercise 13.4 we know that $\Ta$ is the unique coarsest topology in the collection. 
            Since $\Ta\in\mathcal{C}$ each $f_\alpha$ is continuous relative to $\Ta$.
        \item We show that $\Ss$ is a subbasis by proving that $\Ta'\in\mathcal{C}$ if and only if 
        $\Ss\subset\Ta$. Since $\Ta$ is the coarsest topology in $\mathcal{C}$, it is the smallest topology containing $\Ss$ and so $\Ss$ is a basis.
        
        Suppose $\Ta\in\mathcal{C}$ and let $U\in\Ss$. Then $U=f^{-1}_\beta\left(U_\beta\right)$ 
        for some $U_\beta$ open in $X_\beta$. Since $\Ta'\in\mathcal{C}$ it follows that $U$ is open 
        in $\Ta'$ and so $\Ss\subset\Ta$.

        Conversely suppose that $\Ta'$ is a topology on $A$ such that $\Ss\subset\Ta'$. Then for any 
        $U_\beta$ open in $X_\beta$, $f^{-1}_\beta\left(U_\beta\right)\in\Ss$ and so it is open in $\Ta$.
        Hence every $f_\alpha$ is continuous relative to $\Ta'$ and so $\Ta'\in\mathcal{C}$.
        \item Suppose $g:Y\to A$ is continuous and let $U_\beta$ be an open subset of $X_\beta$. 
        Then $U=f^{-1}_\beta\left(U_\beta\right)$ is open in $A$ and so $g^{-1}(U)$ is open in $Y$. 
        Since $\beta$ was arbitrary, it follows that $f_\alpha\circ g:Y\to X_\alpha$ is continuous 
        for all $\alpha$.

        Conversely suppose that  $f_\alpha\circ g:Y\to X_\alpha$ is continuous for all $\alpha$ and 
        let $U$ be open in $A$.
        By part (b) we know that $U$ is an arbitrary union of finite intersections of elements 
        $f^{1}_\beta\left(U_\beta\right)\in\Ss$. 
        There $g^{-1}(U)$ is an arbitrary union of finite intersections of elements of the form 
        $g^{-1}\left(f^{-1}_\beta\left(U_\beta\right)\right)=\left(f_\beta\circ g\right)^{-1}\left(U_\beta\right)$ 
        which are all open in $Y$.
        Hence $g^{-1}(U)$ is open in $Y$ and so $g$ is continuous.
        \item Let $U$ be an open subset of $A$ and consider $f(U)=\left(f_\alpha(U)\right)_{\alpha\in J}$.
            Let $\mathbf{y}\in f\left(U\right)$ such that $\mathbf{y}=f(a)$ for some $a\in A$. 
            Then there exists a basis element $B_A$ such that $a\in B_A$. By part (b) $B_A$ is a finite 
            intersection of preimages of open sets. 
            Hence
            $$ B_A=\prod_{\alpha\in I}f^{-1}_\alpha\left(U_\alpha\right)$$
            for some finite $I\subset J$. Then the element
            $$ V = \prod_{\alpha\in I}U_\alpha\times\prod_{\alpha\in J\setminus I}X_\alpha$$
            is open in $\prod X_\alpha$ since $I$ is finite. Then $B_Z=V\cap Z$ is a basis element of $Z$. 
            Furthermore, since $\mathbf{y}=f(a)\in f(A)=Z$ and $y_\alpha=f_\alpha(a)\in U_\alpha$ for all 
            $\alpha\in I$ and $y_\alpha\in X_\alpha$ for all $\alpha\in J\setminus I$ it follows that 
            $\mathbf{y}\in B_Z$.
            Since $\mathbf{y}$ was arbitrary, $f(U)$ is an arbitrary union of basis elements in $Z$ and 
            so it is open in $Z$. 
            
    \end{enumerate}
\end{sol}