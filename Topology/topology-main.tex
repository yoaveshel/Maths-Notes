\documentclass{article}

\usepackage[utf8]{inputenc}
\usepackage{csquotes}
\usepackage[english]{babel}
\usepackage{amsmath,amssymb,amsthm,textcomp}
\usepackage{mathtools}
\usepackage{biblatex}
\usepackage{tikz}
\usepackage{graphics, setspace}
\usepackage{listings}
\usepackage{lipsum}
\usepackage{hyperref}
\hypersetup{
    colorlinks,
    citecolor=black,
    filecolor=black,
    linkcolor=black,
    urlcolor=black
}

\DeclareMathAlphabet{\pazocal}{OMS}{zplm}{m}{n}
\DeclareMathOperator{\Ima}{Im}
\newcommand{\Ba}{\mathcal{B}}
\newcommand{\Ta}{\mathcal{T}}
\newcommand{\Aa}{\mathcal{A}}
\newcommand{\R}{\mathbb{R}}
\newcommand{\C}{\mathbb{C}}
\newcommand{\Z}{\mathbb{Z}}
\newcommand{\Q}{\mathbb{Q}}
\newcommand{\N}{\mathbb{N}}
\newcommand{\p}{\mathbb{P}}
\newcommand{\Ss}{\mathcal{S}} % Schwartz space
\newcommand{\F}{\mathcal{F}} % Fourier Transform
\newcommand{\Rf}{\mathcal{R}} % reflection
\newcommand{\E}{\mathbb{E}}
\newcommand{\Int}{\text{Int }}
\newcommand{\Bd}{\text{Bd }}

\DeclarePairedDelimiter\abs{\lvert}{\rvert}%
\DeclarePairedDelimiter\norm{\lVert}{\rVert}%
% Swap the definition of \abs* and \norm*, so that \abs
% and \norm resizes the size of the brackets, and the 
% starred version does not.
\makeatletter
\let\oldabs\abs
\def\abs{\@ifstar{\oldabs}{\oldabs*}}
%
\let\oldnorm\norm
\def\norm{\@ifstar{\oldnorm}{\oldnorm*}}
\makeatother

\renewcommand{\theenumi}{\alph{enumi}}
\renewcommand{\labelenumii}{\roman{enumii}.}

\newtheorem{theorem}{Theorem}[section]
\newtheorem{corollary}{Corollary}[theorem]
\newtheorem{lemma}[theorem]{Lemma}
\newtheorem*{definition}{Definition}
\newtheorem*{remark}{Remark}

\theoremstyle{remark}
\newtheorem*{sol}{Solution}

\newenvironment{ex}[1]
    {\noindent\textbf{Exercise #1}\normalsize\newline}
    {\vspace{0.5 em}}

\title{Topology - X400416}
\author{Yoav Eshel}
\date{\today}

\setcounter{secnumdepth}{1}

\begin{document}
\maketitle
\tableofcontents
\newpage

These notes are based on Topology ($2^\text{nd}$ edition) by James R. Munkres. 
\section{Summary}
% write down a summary of important definition/theorems as well as useful techniques for solving exercises
\section{Topological Spaces}
The motivation behind defining a topological space is to generalize the notion of a metric space.
Recall that a metric on a set $X$ is a map $d:X\times X \to [0,\infty)$ that satisfies
\begin{enumerate}
    \item $d(x,y)=d(y,x)$
    \item $d(x,x)=0$
    \item $d(x,y)>0, x\not=y$
    \item $d(x,y)\leq d(x,z)+d(z,y)$
\end{enumerate}
Then we say that a set $U\subset X$ is open if for all $x\in U$ and some $r>0$ 
$$
    B(x,r):=\{y\in X\mid d(x,y)<r\}\subset U.
$$
In other words, around every point in $U$ there is a "ball" that is contained in $U$. 
In Analysis I one learns about continuity using the classic $\varepsilon$-$\delta$ definition which requires a metric.
As it turns out, we don't really need a metric to define a continuous function, only open sets:
\begin{definition}
    A function between metric spaces is continuous if and only if the preimage of an open set is open.
\end{definition}
Using this definition, different seeming metric can yield the same notions of which functions are continuous! 
We call the collection of open subsets of $X$ defined by some metric $d:X\times X\to [0,\infty)$ a \textbf{topology}.
This open sets satisfy some important properties. Namely: (1) $X$ and $\varnothing$ are open, (2) arbitrary unions of open sets are open and (3) finite intersections of open sets is open.
It turns out that a metric is not required to define a topology, only these three properties:
\begin{definition}
    Let $X$ be a set. Then a topology on $X$ is a set $\mathcal{T}\subset\mathcal{P}(x)$ such that
    \begin{enumerate}
        \item $\varnothing\in\Ta, X\in\Ta$
        \item If $\{U_{\alpha}\}\subset\Ta$ then  $\bigcup_{\alpha}U_{\alpha}\in\Ta$
        \item If $\{U_i\}_{i=0}^n\subset\Ta$ then $\bigcap_{i=0}^n U_i\in\Ta$
    \end{enumerate}
    A topological space is the pair $(X,\Ta)$
\end{definition}
Then we say that $U\subset X$ is open if $U\in\Ta$. 
Note that which sets are open depends on the topology, which might conflict with your notion of open set as defined above. 
For example, in the topology $\mathcal{P}(\R)$ every subset of the real line is open, while in the topology $\{\varnothing, \R\}$ only the empty set and $\R$ are open.
We often say that $U$ is open in $X$ without giving a specific topology, which simply means that the statement that follows will hold for any topology we define on $X$ and any element in that topology.

If $\Ta$ and $\Ta'$ are two topologies on $X$ such that $\Ta\subseteq\Ta'$ than we say that $\Ta'$ is \textbf{finer} (or \textbf{strictly finer} if the containment is proper) than $\Ta$. 
We similarly say that $\Ta$ is \textbf{coarser} (or \textbf{strictly coarser}) than $\Ta'$.
It might also be that case that two topologies are not \textbf{comparable}.

\section{Basis for a Topology}
Specifying topologies directly is often not possible, due the enormous size of many topologies.
So we often define a topology using a smaller subset called a \textbf{basis}.
\begin{definition}
    If $X$ is a set, then a \textbf{basis} of a topology is a collection $\mathcal{B}$ of subsets of $X$ such that
    \begin{enumerate}
        \item $\forall x\in X, \exists B\in\mathcal{B}$ such that $x\in B$
        \item If $x\in B_1\cap B_2$ with $B_1,B_2\in\mathcal{B}$ then there exists $B_3\in\mathcal{B}$ with $x\in B_3\subset B_1\cap B_2$.
    \end{enumerate}
\end{definition}
If $\Ta$ is a topology generated by a basis $\Ba$ then $U$ is open if for all $x\in U$ there exists $B\in\Ba$ such that $x\in B\subset U$. 
Also $B\in\Ta$ for all $B\in\Ba$. The proof that $\Ta$ is indeed a topology is not included.
An alternative construction of a topology from a basis is given by the following lemma
\begin{lemma}
    Let $X$ be a set and $\Ba$ a basis for a topology $\Ta$. Then $\Ta$ equals the collection of all unions of elements in $\Ba$.
\end{lemma}
Using this lemma is sometime easier in practice; given a basis $\Ba$ and $U\subset X$, if one can write $U$ as a union of elements in $\Ba$ then $U$ is open.

We can also go in the reverse direction: from topology to a basis.
\begin{lemma}
    Let $X$ be a topological space. If $\mathcal{C}$ is a collection of open sets such that for each open set $U$ and each $x\in U$ there is $C\in\mathcal{C}$ such that $x\in C\subset U$, then $\mathcal{C}$ is a basis.
\end{lemma}

When topologies are given in terms of basis, we can already determine which one is finer using the following criterion
\begin{lemma}
    Let $\Ba$ and $\Ba'$ be bases for the topologies $\Ta$ and $\Ta'$ respectively. Then
    $\Ta\subset\Ta'$ if and only if for each $x\in X$ and $B\in\Ba$ containing $x$ there exists $B'\in\Ba'$ such that $x\in B'\subset B$.
\end{lemma}
It might be tricky to remember the direction of the inclusion. One way to think about it is since $\Ta'$ has more subsets of $X$ it needs to have smaller basis elements.

Lastly, we define the notion of a \textbf{subbasis}.
\begin{definition}
    A subbasis $\Ss$ for a topology on $X$ is a collection of subsets of $X$ whose union equals $X$. The topology generated by $\Ss$ is the collection of all unions of finite intersection of elements of $\Ss$.
\end{definition}

To conclude this section we define 3 topologies on the real line using the notion of a basis:
\begin{enumerate}
    \item The \textbf{standard topology} generated by the collection of all open intervals $(a,b)$ with $a<b$ (it is not a recursive definition. Here we use open in the familiar metric sense).
    \item The \textbf{lower limit topology} is generated by half-open intervals $[a,b)$. When $\R$ is given in the lower limit topology we denote it $\R_l$.
    \item The \textbf{K-topology} is generate by open intervals and sets of the form $(a,b)-K$ where $K=\{1/n\mid n\in\N\}$. When $\R$ is given in this topology we denote it $\R_k$.
\end{enumerate}
One maybe surprising property is that both $\R_l$ and $\R_k$ are finer than the standard topology, but are not comparable with one another.

\section{The Product Topology on $X\times Y$}\label{sec:XtimesY}
Let $X$ and $Y$ be topological spaces. 
The product topology on $X\times Y$ is generated by the collection $\Ba$ of all sets of the form $U\times V$ with $U$ open in $X$ and $V$ open in $Y$.
Alternatively, if $\Ba$ is a basis for $X$ and $\mathcal{C}$ is a basis for $Y$, then
$$
    \mathcal{D}=\{B\times C\mid B\in\Ba, C\in\mathcal{C}\}
$$
is a basis for $X\times Y$.

\section{The Subspace Topology}
If $X$ is a topological space with topology $\Ta_x$ and $Y\subset X$ then the subspace topology on $Y$ is defined as
$$
    \Ta_Y=\{Y\cap U\mid U\in\Ta_X\}.
$$
The fact that the collection $\Ta_Y$ has all the properties of a topology follows from the $\Ta_X$ being a topology.
Then if $\Ba_X$ is a basis for a topology on $X$, the basis of the subspace topology on $Y$ is given by
$$
    \Ba_Y=\{Y\cap B\mid B\in\Ba_X\}.
$$

Open sets in the subspace topology on $Y$ are not necessarily open in $X$.
If $A$ is an open set in $Y$, then $A$ is open in $X$ if $Y$ is open in $X$.

Another fun property of subspace topology is that it "commutes" with the product topology:
If $A$ is a subspace of $X$ and $B$ is a subspace of $Y$ then the product topology on $A\times B$ is equal to the subspace topology on $A\times B$ as a subspace of $X\times Y$.

\section{Closed Sets and Limit Points}
If $X$ is a topological space and $A\subset X$ then $A$ is closed if $X-A$ is open.
Closed sets have similar properties to open sets:
\begin{theorem}
    Let $X$ be a topological space. Then
    \begin{enumerate}
        \item $X$ and $\varnothing$ are closed
        \item Arbitrary intersections of closed sets are close
        \item Finite unions of closed sets are closed.
    \end{enumerate}
\end{theorem}
One can just as well define a topology in terms of closed sets, but the definition using open sets is much more common.
Of course, mathematics wouldn't be fun if there wasn't any space for confusion. In a topology, a set can be open, closed, neither or both. 
So don't think of sets as doors. 

If $Y$ is a subspace of a topological space $X$, then a closed set in $X$ is not necessarily closed in $Y$.
A set $A$ is closed in $Y$ if and only if it equals the intersection of a closed set of $X$ with $Y$. 
This is easy to verify since if $A$ is closed in $Y$ then $Y-A$ is open in $Y$ and so it equals $U\cap Y$ for some open set $U$ in $X$. 
Then $X-U$ is closed and $A=Y\cap(X-U)$. The other direction is similarly proved.

Let $A\subset X$. Then the smallest closed set that contain $A$ is called the closure of $A$ and is denoted 
$$\bar{A}=\bigcap_{\substack{C\text{ closed}\\A\subset C}} C.$$
The smallest open set containing $A$ is called the interior of $A$ and is denoted 
$$\text{Int} A = \bigcup_{\substack{U\text{ open}\\U\subset A}}U.$$ 
Then clearly
$$
    \text{Int}A\subset A\subset\bar{A}.
$$
A point $x\in X$ is in the closure of $A$ if and only if for any basis element $B$ containing $x$ the intersection $A\cap B$ is non empty.

To add to the soup of metaphors, we define the \textbf{neighborhood} of $x\in X$ is an open set $U$ containing $x$. Then a \textbf{limit point} of $A\subset X$ is defined as
\begin{definition}
    A limit point of $A\subset X$ is a point $x\in A$ such that every neighborhood of $x$ intersects $A$ in a point different than $x$
\end{definition}
In other words, $x$ is a limit point of $A$ if for every open set $U$ containing $x$ the intersection $U\cap(A-{x})$ is non empty. Now denote the set of limit points of $A$ by $A'$.
Then $\bar{A}=A'\cup A$. This gives us an alternative condition for a set being closed. Namely, if $A'\subset A$ then $\bar{A}=A$ and so $A$ is closed.
Alternatively, if $A$ is closed, then $\bar{A}=A$ and so $A'\subset A$.
Hence a set is closed if and only if it contains all of its limit points.

We conclude this section with a definition of convergence in a topological space.
\begin{definition}
    Let $(X,\Ta)$ be a topological space. If $x_n$ is a sequence in $X$ then $x_n\to x$ as $n\to\infty$ if
    $$
        \forall U\in\Ta, x\in U\,\exists N\in\N: \{x_n\}_{n\geq N}\subset U
    $$
\end{definition}
Note that this definition does not require any metric. The downside is that limit are not necessarily unique and sequences can converge to any number of points.
To avoid this horrific phenomenon, we add an extra condition to rid our definition of topological spaces of such situations. 
\begin{definition}
    A topological space $X$ is called \textbf{Hausdorff space} if for every $x_1,x_2\in X, x_1\neq x_2$ there exists disjoint neighborhoods $U_1$ and $U_2$ that contain $x_1$ and $x_2$ respectively.
\end{definition}
In essence, this means that we can distinguish between every two points in our topology. Limits of sequences in Hausdorff spaces are unique, which is quiet pleasant.

\section{Continuous Functions}
Recall from the introduction that we defined a function $f:X\to Y$ to be continuous if for each open set $V$ in $Y$ the set
$$f^{-1}(V)=\{x\in X\mid f(x)\in V\} $$
is open in $X$. To prove continuity it suffices to show that for any basis element (or even any subbasis element!) $B\subset Y$, $f^{-1}(B)$ is open in $X$. 
Using this definition, a given function may or may not be continuous depending on the topologies specified for its domain and range. 
If we wish to emphasize it we can say that $f$ is continuous relative to specific topologies.
To illustrate this point, let $\R$ denote the standard topology on the real line and let
$$
    f:\R\to\R_l\quad\text{and}\quad g:\R_l\to\R
$$
be the identity maps, i.e. $x=f(x)=g(x)$ for all $x\in\R$.
Then $f$ is not continuous since $f^{-1}\left([a,b)\right)=[a,b)$ which is not open in the standard topology.
However $g$ is continuous since $g^{-1}\left((a,b)\right)=(a,b)$ which is open in $\R_l$.

There are other equivalent conditions for a function to be continuous
\begin{theorem}
    Let $X$ and $Y$ be topological spaces and $f:X\to Y$. The the following are equivalent
    \begin{enumerate}
        \item $f$ is continuous
        \item For every subset $A$ of $X$ it holds that $f\left(\overline{A}\right)\subset\overline{f(A)}$
        \item For every closed subset $B$ of $Y$, the set $f^{-1}(B)$ is closed in $X$.
        \item For each $x\in X$ and each neighborhood $V$ of $f(x)$ there is a neighborhood $U$ of $x$ such that $f(U)\subset V$.
    \end{enumerate}
\end{theorem}

A bijection $f:X\to Y$ is called a \textbf{homeomorphism} (the e is not a typo. It is indeed a different word to homomorphism!) is both $f$ and $f^{-1}$ are continuous.
An isomorphism in algebra is a bijective map that preserve algebraic structure (between groups, rings, fields, etc.).
In topology, a homeomorphism is a bijective map that preserve topological structure. However in algebra any homomorphism (without an e) that is one-to-one and onto is an isomorphism.
This is, in general, not true in topology. There are continuous bijective maps whose inverses are not continuous. 
If $f:X\to Y$ is injective continuous map, and $g:X\to f(X)$ obtained by restricting the range of $f$ is a homeomorphism, then we say that $f$ is an \textbf{imbedding}.

There are a few important rules for constructing continuous functions
\begin{enumerate}
    \item Constant mappings are continuous
    \item Inclusion mappings are continuous (i.e. if $A$ is a subspace of $X$ and $i:A\to X$ is the identity then $i$ is continuous)
    \item Composition of continuous maps are continuous
    \item Continuous maps restricted to a subspace of their domain are continuous
    \item Restricting or expanding the range of continuous maps gives continuous maps
    \item The map $f:X\to Y$ is continuous if $X$ can be written as the union of open sets $U_\alpha$ such that $f\mid_{U_\alpha}$ is continuous for each $\alpha$
\end{enumerate}

\begin{lemma}{(The pasting lemma)}
    Let $X=A\cup B$ be a topological space and $A, B$ are closed. Consider
    $$ f:A\to Y\quad\text{and}\quad g:B\to Y$$
    continuous such that $f(x)=g(x)$ for all $x\in A\cap B$. Then $h:X\to X$ defined by
    $$ h(x)=\begin{cases}
        f(x),&x\in A\\
        g(x),&x\in B\\
    \end{cases}$$
    is continuous.
\end{lemma}
The lemma also holds if $A$ and $B$ are both open.
\begin{lemma}{(Maps into products)}
    Let $f:A\to X\times Y$ be given by
    $$f(a)=(f_1(a),f_2(a)).$$
    Then $f$ is continuous if and only if $f_1$ and $f_2$ are continuous.
\end{lemma}
Note that there is no useful criterion for maps out of products.

\section{The Product Topology}
In Section \ref{sec:XtimesY} we defined the topology on $X\times Y$ as the topology generated by products of open sets.
However, when considering arbitrary products there are two ways to proceed. Let $J$ be an arbitrary index set and $X_\alpha$ a topological space for each $\alpha\in J$. Then
\begin{definition}
    The box topology on $\prod_{\alpha\in J}X_\alpha$ is generated by sets of the form
    $$\prod_{\alpha\in J}U_\alpha$$
    where each $U_\alpha$ is open in $X_\alpha$.
\end{definition}

\begin{definition}
    The product topology on $\prod_{\alpha\in J}X_\alpha$ is generated by the subasis of all subsets of the form $\pi_\beta^{-1}(U_\beta), \beta\in J$
    where $\pi_\beta: \prod_{\alpha\in J}X_\alpha\to X_\beta$ is the projection mapping. 
    Therefore a basis for the product topology is the collection of subsets of the form
    $$\prod_{\alpha\in J}V_\alpha$$
    where 
    $$V_\alpha =\begin{cases}
        U_\alpha\text{ open in }X_\alpha,&\alpha\in I\\
        X_\alpha,&\alpha\not\in I
    \end{cases}$$
    for some finite $I\subset J$.
\end{definition}
It follows that these two topologies agree for finite cartesian products but differ for the infinite one.
The product topology is quite important, while the box topology main use is as a counter example.

However there are some properties that hold for both topologies
\begin{enumerate}
    \item If $A_\alpha$ is a subspace of $X_\alpha$ for each $\alpha$ then $\prod A_\alpha$ is a subspace of $\prod X_\alpha$ when both products are given in the same topology.
    \item If each $X_\alpha$ is Hausdorff, then $\prod X_\alpha$ is Hausdorff in either topology.
    \item If $A_\alpha\subset X_\alpha$ for each $\alpha$ then
     $$\prod\overline{A_\alpha}=\overline{\prod A_\alpha}$$
     in either topology.
\end{enumerate}

A property that does not hold for the box topology is this
\begin{theorem}
    Let $\prod X_\alpha$ have the product topology. Let $f:A\to\prod X_\alpha$ be given by 
    $$f(a)=(f_\alpha(a))_{\alpha\in J}.$$
    Then $f$ is continuous if and only if $f_\alpha$ is continuous $\forall\alpha\in J$.    
\end{theorem}
\section{The Metric Topology}


\section{The Quotient Topology}
\begin{definition}
    Let $X$ and $Y$ be topological space and $p:X\to Y$ be a surjective map. 
    The map $p$ is called a \textbf{quotient map} if a subset $U$ of $Y$ is open if and only if $p^{-1}(U)$ is open in $X$.
\end{definition}
We say that a subset $C$ of $X$ is \textbf{saturated} if every set $p^{-1}({y})$ the intersects $C$ is contained in $C$.
So we can also say the $p$ is a quotient map if $p$ is continuous and $p$ maps saturated open sets of $X$ to open sets of $Y$.
We can also go the other way and use a quotient map to define a topology.
\begin{definition}
    If $X$ is a space and $A$ is and if $p:X\to A$ is a surjective map, then there exists exactly one topology on $A$ relative to which $p$ is a quotient map.
    We call this topology the \textbf{quotient topology} induced by $p$.
\end{definition}
\begin{definition}
    Let $X$ be a space and $X^*$ a partition of $X$. Let $p:X\to X^*$ be a the surjective map the carries each point of $X$ to the element of $X^*$ containing it. 
    In the quotient topology induced by $p$, the space $X^*$ is called the \textbf{quotient space} $X$.
\end{definition}
Since $X^*$ defines an equivalence relation on $X$, one can think of $p$ as the map that send each element of $X$ to its equivalence class.
Hence a subset $U$ of $X^*$ is a collection of equivalence classes, and so $U$ is open if the the union of the equivalence classes contained in $U$ is open in $X$.

\newpage
\section{Exercises}

\subsection{Symmetric Polynomial}
    \begin{ex}{14.10}
        Express the symmetric polynomials $\sum_n T_1^2T_2$ and $\sum_{n} T_1^3T_2$ in the elementary symmetric polynomials.
    \end{ex}
    \begin{sol}
        To get the polynomial $\sum_n T_1^2T_2$ we start with
        $$
            s_1s_2=\sum_n T_1\sum_n T_1T_2 = \sum_n T_1^2T_2+3\sum_n T_1T_2T_3 = \sum_n T_1^2T_2+3s_3
        $$
        Thus 
        $$
            \sum_n T_1^2T_2 = s_1s_2-3s_3
        $$

        Similarly, to transform the polynomial $\sum_{n} T_1^3T_2$ we start with
        \begin{align*}
            s_1^2s_2&=\left(\sum_nT_1\right)^2\sum_nT_1T_2\\
            &=\left(\sum_n T_1^2+2\sum_n T_1T_2\right)\sum_nT_1T_2\\
            &=\sum_nT_1^2\sum_n T_1T_2+2s_2^2\\
            &=\sum_nT_1^3T_2+\sum_n T_1^2T_2T_3+2s_2^2.
        \end{align*}
        And since
        $$
            s_1s_3=\sum_nT_1\sum_nT_1T_2T_3=\sum_nT_1^2T_2T_3+4\sum_n T_1T_2T_3T_4
        $$
        it follows that $\sum_n T_1^2T_2T_3=s_1s_3-4s_4$ and so
        $$
            \sum_{n} T_1^3T_2=s_1^2s_2-s_1s_3+4s_4-2s_2^2
        $$
    \end{sol}

    \begin{ex}{14.14}
        Prove: For $n\in\Z_{>0}$, we have $\Delta(X^n+a)=(-1)^{\frac12n(n-1)}n^na^{n-1}$.
    \end{ex}
    \begin{proof}
        Let $f(X)=X^n+a$ and let $\alpha_i$ be its roots. Then $f'(X)=nX^{n-1}$ and
        $$
            \Delta(f)=(-1)^{n(n-1)/2}R(f,f').
        $$
        Let $f_1(X)=a$ and then $f\equiv f_1\mod(f')$ since $f = f_1+f'\cdot\left(\frac1n X\right)$.
        Simplifying the resultant we get
        \begin{align*}
            R(f,f')&=R(f',f)&&(\text{Property }1)\\
            &=n^{n}R(f',f_1)&&(\text{Property }3)\\
            &=n^{n}\cdot\left(n^0\prod_{i=1}^{n-1}f_1(\alpha_i)\right)&&(\text{Property }2)\\
            &=n^n a^{n-1}
        \end{align*}
        and the result follows.
    \end{proof}

    \begin{ex}{14.15}
        Calculate the discriminant of the polynomial $f(X)=X^4+pX+q\in\Q(p,q)[X]$.
    \end{ex}
    \begin{sol}
        Then $f'(X)=4X^3+p$ and so 
        $$f_1(X)=f-f'\cdot h = X^4+pX+q+(4X^3+p)(\frac14 X) = \frac{3p}{4}X+q.$$
        Then the resultant is
        \begin{align*}
            R(f,f')&=R(f',f)&&(\text{Property } 1)\\
            &=4^{4-1}R(f', f_1)&&(\text{Property } 3)\\
            &=4^3\left((-1)^{3\cdot 1}R(f_1,f')\right)&&(\text{Property } 1)\\
            &=-4^3\left(\left(\frac{3p}{4}\right)^3\prod_{i=1}^{1}f'\left(\frac{-4q}{3p}\right)\right)&&(\text{Property } 2)\\
            &=-3^3p^3\left(4\left(\frac{-4q}{3p}\right)^3+p\right)\\
            &=4^4q^3-3^3p^4.
        \end{align*}
        Therefore the discriminant of $f$ is
        $$
            \Delta(f) = (-1)^{4\cdot 3/2}R(f,f') = R(f, f') = 4^4q^3-3^3p^4.
        $$
    \end{sol}

    \begin{ex}{14.16}
        For every $n>1$, determine an expression for the discriminant of the polynomial $f(X) = X^n+pX+q\in\Q(p,q)[X]$.
    \end{ex}
    \begin{sol}
        Let $f(X)=X^n+pX+q\in\Q(p,q)[X]$ for $n>1$. 
        Then $f'(X)=nX^{n-1}+p$ and $f\equiv f_1\mod(f')$ where
        $$f_1 = f-f'\cdot h = X^n+pX+q-\left(nX^{n-1}+p\right)\left(\frac1n X\right)=\frac{p(n-1)}{n}X+q.$$
        The resultant of $f$ and $f'$ is given by
        \begin{align*}
            R(f,f') &= R(f', f)&&(\text{Property } 1)\\
            &=n^{n-1}R(f', f_1)&&(\text{Property } 3)\\
            &=n^{n-1}\left((-1)^{n-1}R(f_1, f')\right)&&(\text{Property } 1)\\
            &=(-n)^{n-1}\left(\frac{p(n-1)}{n}\right)^{n-1}\prod_{i=1}^1 f'\left(-\frac{nq}{(n-1)p}\right)&&(\text{Property } 2)\\
            &=(-1)^{n-1}p^{n-1}(n-1)^{n-1}\left(\frac{(-1)^{n-1}n^nq^{n-1}}{(n-1)^{n-1}p^{n-1}}+p\right)\\
            &=n^nq^{n-1}+(-1)^{n-1}p^n(n-1)^{n-1}.
        \end{align*}
        Hence the discriminant of $f$ is
        $$
            \Delta(f)=(-1)^{n(n-1)/2}R(f,f')=(-1)^{n(n-1)/2}\left(n^nq^{n-1}+(-1)^{n-1}p^n(n-1)^{n-1}\right)
        $$
    \end{sol}

    \begin{ex}{14.17}
        Let $f\in\Z[X]$ be a monic polynomial. Prove that the following are equivalent
        \begin{enumerate}
            \item $\Delta(f)\neq 0$.
            \item The polynomial $f$ has no double zeroes in $\C$.
            \item The decomposition of $f$ in $\Q[X]$ has no multiple prime factors.
            \item The polynomial $f$ and its derivative $f'$ are relatively prime in $\Q[X]$.
            \item The polynomial $f\mod p $ and $f' \mod p$ are relatively prime in $\mathbb{F}_p[X]$ for almost all prime numbers $p$.
        \end{enumerate}
    \end{ex}
    \begin{proof}
        Let  $f\in\Z[X]$ be monic and $\{\alpha_1,\alpha_2,\dots,\alpha_n\}$ it roots in $\C$.

        $(1)\Rightarrow  (2)$. Suppose that $\alpha_i=\alpha_j$ for some $i\neq j$. Then 
        $$\Delta(f)=\prod_{1\leq i<j\leq n}(\alpha_i-\alpha_j)= 0,$$ 
        which is a contradiction.
        Therefore if $f$ has non-zero discriminant it has no double zeroes in $\C$. 

        $(2)\Rightarrow (3)$.
        
        $(3)\Rightarrow (4)$.

        $(4)\Rightarrow (5)$. If $f$ and $f'$ are relatively prime in $\Q[X]$ then 

        $(1)\Rightarrow (1)$. 
    \end{proof}

    \begin{ex}{14.19}
        Let $f\in\Q[X]$ be a monic polynomial with $n=\deg(f)$ distinct complex roots. Prove: the sign of $\Delta(f)$ is equal to $(-1)^s$ where $2s$ is the number of non-real zeroes of $f$.
    \end{ex}
    \begin{proof}
        Let $\{\alpha_1,\dots,\alpha_{n}\}$ be all the roots of $f$.
        Then each term $(\alpha_i-\alpha_j)^2$ in the discriminant falls into one of 3 cases
        \begin{enumerate}
            \item Both $\alpha_i$ and $\alpha_j$ are non-real. Then
            \begin{enumerate}
                \item If $\alpha_j=\overline{\alpha_i}$ then $\alpha_i-\alpha_j$ is purely complex and $(\alpha_i-\alpha_j)^2$ is negative.
                \item If $\alpha_j\neq\overline{\alpha_i}$ then $\overline{\alpha_i}$ and $\overline{\alpha_j}$ are also roots of $f$ and the term
                $$(\alpha_i-\alpha_j)^2(\overline{\alpha_i}-\overline{\alpha_j})^2=\left((\overline{\alpha_i-\alpha_j})(\alpha_i-\alpha_j)\right)^2=\abs{\alpha_i-\alpha_j}^2 $$
                is positive.
            \end{enumerate}
            \item $\alpha_i$ is non-real and $\alpha_j$ is real. Then $\overline{\alpha_i}$ is a root of $f$ and the term
            $$(\alpha_i-\alpha_j)^2(\overline{\alpha_i}-\alpha_j)^2=\abs{\alpha_i-\alpha_j}^2 $$
            is positive.
            \item Both $\alpha_i$ and $\alpha_j$ are real. Then $(\alpha_i-\alpha_j)^2$ is positive.
        \end{enumerate}
        Since the only negative terms are of the form $(\alpha_i-\overline{\alpha_i})^2$ and there are $2s$ non-real roots the sign of the determinant is $(-1)^s$.

    \end{proof}

    \begin{ex}{14.20}
        Prove: $f(X)=X^3+pX+q\in\R[X]$ has three (counted with multiplicity) real zeroes $\iff$ $4p^3+27q^\leq 0$.
    \end{ex}
    \begin{proof}
        By Ex. 16 we know that $\Delta(f)=(-1)^3\left(3^3q^2+2^2p^3\right)=-27q^2-4p^3$. 
        Let $a,b$ and $c$ be the roots of $f$. If $a,b,c\in\R$ then 
        $$
        -27q^2-4p^3=\Delta(f)=(a-b)^2(a-c)^2(b-c)^2\geq 0
        $$
        and so $4p^3+27q^\leq 0$.

        Now suppose that $a=x+yi$ and $b=x-yi$ are complex conjugates and $c$ is real. Then 
        \begin{align*}
            -27q^2-4p^3&=\Delta(f)\\
            &=(a-b)^2(a-c)^2(b-c)^2\\
            &=-4y^2\left((a-c)(\overline{a-c})\right)^2\\
            &=-4y^2\abs{a-c}^2\\
            &\leq 0.
        \end{align*}
        Hence $4p^3+27q^\geq 0$ and the result follows by contraposition.
    \end{proof}
        
    \begin{ex}{14.21}
        Express $p_4=\sum_nT_1^4$ in elementary symmetric polynomials
    \end{ex}
    \begin{sol}
        Let $n\geq 4$. Starting with
        \begin{align*}
            s_1^4 &= \left(\sum_nT_1\right)^4\\& = \sum_n T_1^4+4\sum_n T_1^3T_2+12\sum_n T_1^2T_2T_3+6\sum_nT_1^2T_2^2+24\sum_nT_1T_2T_3T_4.
        \end{align*}
        To understand how to coefficients of the sum are obtained, consider the number of ways the $T_i$ can be arranged. 
        For example, $T_1^4=T_1T_1T_1T_1$ can only be arranged in 1 way but $T_1^2T_2T_3=T_1T_1T_2T_3$ can be arrange in $\frac{4!}{2}=12$ ways (where we divided by 2 since the two $T_1$ can be swapped in any given arrangement).
        Then
        $$
            s_1^2s_2=\left(\sum_n T_1\right)^2s_2=\left(\sum_nT_1^2+2\sum_n T_1T_2\right)s_2 = \sum_n T_1^3T_2+\sum_nT_1^2T_2T_3+2s_2^2.
        $$
        So far we have
        \begin{align*}
            p_4 &= s_1^4-4\left(s_1^2s_2-2s_2^2-\sum_nT_1^2T_2T_3\right)-12\sum_n T_1^2T_2T_3-6\sum_nT_1^2T_2^2-24\sum_nT_1T_2T_3T_4\\
            &=s_1^4-4s_1^2s_2+8s_2^2-24s_4-6\sum_nT_1^2T_2^2-8\sum_n T_1^2T_2T_3.
        \end{align*}
        So continuing with $\sum_nT_1^2T_2^2$ we get
        $$
            s_2^2 = \left(\sum_n T_1T_2\right)^2=\sum_n T_1^2T_2^2+2\sum_n T_1^2 T_2T_3+6\sum_n T_1T_2T_3T_4.
        $$
        Finding the coefficients here is slightly trickier since $s_2$ contains pairs not all arrangements are allowed. 
        For example, $T_1^2T_2^2$ can only come from the pair $T_1T_2$. On the other hand $T_1T_2T_3T_4$ can come from $T_1T_2$ and $T_3T_4$ or $T_1T_4$ and $T_2T_3$ and so on.
        We choose the first pair (${4\choose 2}=6$ ways) which also fixes the second pair and so there are 6 ways to get $T_1T_2T_3T_4$.
        Hence
        \begin{align*}
            p_4 &= s_1^4-4s_1^2s_2+8s_2^2-24s_4-6\left(s_2^2-2\sum_nT_1^2T_2T_3-6s_4\right)-8\sum_n T_1^2T_2T_3\\
            &=s_1^4-4s_1^2s_2+2s_2^2+12s_4+4\sum_n T_1^2T_2T_3.
        \end{align*}
        Using Exercise 14.10 we get
        \begin{align*}
            p_4 &=s_1^4-4s_1^2s_2+2s_2^2+12s_4+4(s_1s_3-4s_4)\\
            &=s_1^4-4s_1^2s_2+2s_2^2-4s_4+4s_1s_3
        \end{align*}
    \end{sol}

    \begin{ex}{14.22}
        A rational function $f\in\Q[T_1,\dots,T_n]$ is called symmetric if it is invariant under all permutations of the variables $T_i$. Prove that every symmetric rational function is a rational function in the elementary symmetric functions.
    \end{ex}
    \begin{proof}
        Let $f\in\Q[T_1,\dots,T_n]$ be a symmetric rational function. 
        Then $f=g/h$ for $g,h$ polynomials. If $h$ is a symmetric polynomial then $g=fh$ is symmetric as well.
        By the fundamental theorem of symmetric polynomial both $g$ and $h$ can be written in terms of elementary symmetric polynomials and we're done.
        If $h$ is not symmetric, then let 
        $$\tilde{h}=\prod_{\sigma\in S_n\setminus\{e\}}\sigma(h)$$
        and then $h\tilde{h}$ is symmetric so $f=\frac{g\tilde{h}}{h\tilde{h}}$ which is again the case above.
    \end{proof}

    \begin{ex}{14.23}
        Write $\sum_{n}T_1^{-1}$ and $\sum_n T_1^{-2}$ as rational functions in $\Q[s_1,\dots,s_n]$
    \end{ex}
    \begin{sol}
        Starting with
        $$
            \sum_{n}T_1^{-1}=\frac{1}{T_1}+\cdots+\frac{1}{T_n}.
        $$
        We multiply by $1=\frac{s_n}{s_n}$ and simplify
        \begin{align*}
            \frac{s_n}{s_n}\sum_{n}T_1^{-1}&=\frac{T_1T_2\cdots T_n}{T_1T_2\cdots T_n}\left(\frac{1}{T_1}+\cdots+\frac{1}{T_n}\right)\\
            &=\frac{s_{n-1}}{s_n}
        \end{align*}

        For the second expression we present to approaches.
        \begin{enumerate}
            \item Observing that 
                $$\left(\sum_n T_1^{-1}\right)^2=\sum_{n} T_1^{-2}+2\sum_{n}T_1^{-1}T_2^{-1}$$
            we can write using the previous part
                $$ \sum_n T_1^{-2} = \frac{s_{n-1}^2}{s_n^2}-2\sum_{n}T_1^{-1}T_2^{-1}$$
            and multiplying by the second term by $\frac{s_{n}}{s_{n}}$ we get
                $$ \sum_n T_1^{-2} = \frac{s_{n-1}^2}{s_n^2} - 2\left(\frac{1}{T_1T_2}+\cdots+\frac{1}{T_{n-1}T_n}\right)\frac{T_1\cdots T_n}{T_1\cdots T_n}=\frac{s_{n-1}^2}{s_n^2} - 2\frac{s_{n-2}}{s_n}.$$
            Hence $\sum_n T_1^{-2}=\frac{s_{n-1}^2-2s_{n-2}s_n}{s_n^2}$.
            \item The second approach is slightly more involved. We start by multiplying by 1 in a clever (but different) way
                $$\left(\sum_n T_1^{-2}\right)\frac{s_n^2}{s_n^2}=\left(\frac{1}{T_1^2}+\cdots+\frac{1}{T_n^2}\right)\frac{T_1^2\cdots T_n^2}{T_1^2\cdots T_n^2}=\frac{\sum_n T_1^2\cdots T_{n-1}^2}{s_n^2}.$$
            Then $\sum_n T_1^2\cdots T_{n-1}^2$ is obviously (condescending much?) a symmetric polynomial and so we can use our trusty algorithm. Starting with
            \begin{align*}
                s_1^{2-2}s_2^{2-2}\cdots s_{n-1}^{2-0}&=s_{n-1}^2\\
                &=\left(\sum_n T_1\cdots T_{n-1}\right)^2\\
                &=\sum_n T_1^2\cdots T_{n-1}^2 + 2\sum_n T_1^2\cdots T_{n-2}^2T_{n-1}T_n.
            \end{align*}
            Moving to the second term
            \begin{align*}
                s_1^{2-2}\cdots s_{n-2}^{2-1}s_{n-1}^{1-1}s_n^1&=s_{n-2}s_n\\
                &=\left(\sum_n T_1\cdots T_{n-2}\right)T_1\cdots T_n\\
                &=\sum_n T_1^2\cdots T_{n-2}^2 T_{n-1}T_n
            \end{align*}
            and it follows that
            $$\sum_n T_1^2\cdots T_{n-1}^2 = s_{n-1}^2-2s_{n-2}s_n.$$
            So we conclude that
            $$ \sum_n T_1^{-2} = \frac{s_{n-1}^2-2s_{n-2}s_n}{s_n^2}$$
            which is reassuring.
        \end{enumerate}
        Note that in the first approach we stumbled upon something rather interesting:
        $$
            \sum_n T_1^{-1}\cdots T_k^{-1} = \frac{s_{n-k}}{s_n}
        $$
        the proof of which is left as an exercise to the reader.
    \end{sol}

    \begin{ex}{14.24}
        
    \end{ex}
\subsection{Field Extensions}
    \begin{ex}{21.18}
        Let $K\subset L$ be an algebraic extension. For $\alpha, \beta\in L$ prove that we have
        $$ \left[K(\alpha,\beta):K\right]\leq\left[K(\alpha):K\right]\cdot\left[K(\beta):K\right].$$

        Show that equality does not always hold. Does equality always hold if $[K(\alpha):K]$ and $[K(\beta):K]$ are relatively prime?
    \end{ex}
    \begin{proof}
        Let $f$ and $g$ be the minimal polynomials of $\al$ and $\be$ (respectively) in $K[x]$ and $f'$ be the minimal polynomial of $\alpha$ in $K(\beta)[x]$.
        If $\deg f'> \deg f$ then $f$ is a lower degree polynomial in $K(\beta)[x]$ with $f(\alpha)=0$ which is a contradiction. Hence $\deg f'\leq \deg f$ and so
        \begin{align*}
            \left[K(\alpha,\beta):K\right]&=\left[K(\al, \be):K(\be)\right]\cdot\left[K(\be):K\right]\\
            &=\deg f'\cdot \deg g\\
            &\leq\deg f\cdot \deg g\\
            &=\left[K(\al):K\right]\cdot \left[K(\be):K\right],   
        \end{align*}
        as desired.

        To show that equality does not always hold consider $\Q(\sqrt{2}, \sqrt[4]{2})$.
        Then $[\Q(\sqrt{2}):\Q]=2$ and $[\Q(\sqrt[4]{2}):\Q]=4$ but
        $$[\Q(\sqrt{2}, \sqrt[4]{2}):\Q]=4\cdot[\Q(\sqrt{2}, \sqrt[4]{2}):\Q(\sqrt[4]{2})]=4<8$$
        since $\left(\sqrt[4]{2}\right)^2=\sqrt{2}\in\Q(\sqrt[4]{2})$

        Lastly, suppose that $\deg f$ and $\deg g$ are relatively prime. Since
        \begin{align*}
            [K(\al,\be):K]&=[K(\al,\be), K(\al)]\cdot\deg f\\
            &=[K(\al,\be), K(\be)]\cdot\deg g
        \end{align*}
        it follows that $[K(\al,\be):K]$ is divisible by $\deg f$ and $\deg g$ and since they are relatively prime it is also divisible by $\deg f\cdot \deg g$.
        But we know that $[K(\al,\be):K]\leq\deg f\cdot\deg g$ and so $[K(\al,\be):K]=\deg f\cdot \deg g$.
    \end{proof}

    \begin{ex}{21.19}
        Let $K\subset K(\al)$ be an extension of odd degree. Prove that $K(\al^2)=K(\al)$.
    \end{ex}
    \begin{proof}
        Let $f$ be the minimal polynomial of $\al$ in $K[x]$. Then $\deg f=2n+1$ for some $n\in\Z_+$. 
        Since $\al^2\in K(\al)$ we get the tower $K(\al)/K\left(\al^2\right)/K$ and so\
        $$ \left[K(\al):K\right]=\left[K(\al):K\left(\al^2\right)\right]\cdot\left[K\left(\al^2\right):K\right].$$
        Let $g$ be the minimal polynomial of $\al$ in $K\left(\alpha^2\right)$. Then $\deg g\leq 2$ since $x^2-\al^2\in K\left(\alpha^2\right)$ is a polynomial with a root $\al$.
        Since $\left[K(\al):K\right]$ is odd, it is not divisible by two and so $\deg g = 1$. Hence $\left[K(\al):K\left(\al^2\right)\right]=1$ and it follows that $K(\al)=K\left(\al^2\right)$.
    \end{proof}

    \begin{ex}{21.23}
        Show that every quadratic extension of $\Q$ is of the form $\Q\left(\sqrt{d}\right)$ with $d\in\Z$.
        For what $d$ do we obtain the cyclotomic field $\Q(\zeta_3)$?
    \end{ex}
    \begin{proof}
        Let $K/\Q$ be a quadratic extension. Take $\al\in K\setminus\Q$. Then 
        $$ \Q\subset\Q(\al)\subset K $$
        and so
        $$2=\left[K:\Q\right]=\left[K:\Q(\al)\right]\left[\Q(\al):\Q\right].$$
        If $\left[\Q(\al):\Q\right]=1$ then $\Q(\al)=\Q$ and so $\al\in\Q$, which contradicts our assumption. 
        It follows that $\left[K:\Q(\al)\right]=1$ and so $K=\Q(\al)$. 
        Let 
        $$f(x)=x^2+a_1 x+a_0\in\Q[x]$$
        be the minimal polynomial of $\al$. 
        Let $d=\frac{a_1^2}{4}-a_0\in\Q$ and note that $a_0=-\al a_1-\al^2$. Then
        \begin{align*}
            \sqrt{d}&=\sqrt{\frac{a_1^2}{4}-a_0}\\
            &=\sqrt{\frac{a_1^2}{4}+a_1\al+\al^2}\\
            &=\frac{a_1+2\al}{2}.
        \end{align*}
        Hence $\sqrt{d}\in\Q(\al)$. By similar calculations we get $\al=\frac{2\sqrt{d}-a_1}{2}\in\Q(\sqrt{d})$.
        Hence $K=\Q(\al)=\Q(\sqrt{d})$. Of course, it is not yet the case the $d$ is an integer.
        Suppose that $d=\frac{p}{q}$. Since $\sqrt{d}=\frac{1}{q^2}\sqrt{qp}\in\Q(\sqrt{qp})$ we have
        $$K=\Q(\al)=\Q(\sqrt{d})=\Q(\sqrt{qp})$$
        with $qp\in\Z$ as desired.
    \end{proof}

    \begin{ex}{21.24}
        Is every cubic extension of $\Q$ of the form $\Q\left(\sqrt[3]{d}\right)$ for some $d\in\Q$?
    \end{ex}
    \begin{sol}
        No. Let $\alpha$ be a root of the monic irreducible polynomial $f(x)=x^3-3x+1\in\Q[x]$ (possible roots are $\pm 1$ and they both clearly don't work).
        There are three choices for $\alpha$ all in $\R$ (why? Using Exercise 14.16 the determinant is $4\cdot(-3)^3+27\cdot 1=-81<0$ and so by Exercise 14.20 $f$ has three real roots).
        Therefore there are three embeddings $\varphi:\Q(\alpha)\to\C$ and $\text{Im }\varphi\subset\R$.
        
        Assume for contradiction that there exists an isomorphism $\phi:\Q(\alpha)\to\Q(\sqrt[3]{d})$ for some $d\in\Q$.
        Since $\sqrt[3]{d}\not\in\Q$, $x^3-d$ is irreducible and so $f_\Q^{\sqrt[3]{d}}=x^3-d$.
        Since $f_\Q^{\sqrt[3]{d}}$ has one real and two non-real roots (again, using exercises 14.16 and 14.20 with the fact that $27\cdot(-d)^2>0$) there are three embeddings of $\Q(\sqrt[3]{d})$ into $\C$ to of which are not subsets of $\R$.
        
        Let $\Phi:\Q(\sqrt[3]{d})\to\C$ be one of the latter. 
        Then $\Phi\circ\phi:\Q(\alpha)\to\C$ is an imbedding of $\Q(\alpha)$ into $\C$ whose image is not a subset of $\R$.
        Therefore we conclude that $\phi$ doesn't exists.
    \end{sol}

    \begin{ex}{21.26}
        Let $M=\Q(\al)=\Q(1+\sqrt{2}+\sqrt{3})$. Show that $M$ is of degree 4 over $\Q$, determine the minimal polynomial and write $\sqrt{2}$ and $\sqrt{3}$ in the basis $\{1,\al, \al^2,\al^3\}$.
        Also prove that the group $G=\text{Aut}_\Q(M)$ is isomorphic to $V_4$ and that $f^\al_\Q=\prod_{\sigma\in G}X-\sigma(\al)\in\Q[X]$.
    \end{ex}
    \begin{sol}
        Let $\be = \al-1=\sqrt{2}+\sqrt{3}$. Then clearly $M=\Q(\al)=\Q(\be)$. Let
        \begin{align*}
            f(x)&=(x-\sqrt{2}-\sqrt{3})(x+\sqrt{2}-\sqrt{3})(x-\sqrt{2}+\sqrt{3})(x+\sqrt{2}+\sqrt{3})\\
            &=x^4-10x^2+1\in\Q[x]
        \end{align*}
        and so $f(\be)=0$ by construction. 
        
        Is $f$ the minimal polynomial of $\be$ in $\Q[x]$? It is if we can prove that $[M:\Q]=4$.
        From
        $$ (\sqrt{2}+\sqrt{3})(\sqrt{3}-\sqrt{2})=1 $$
        It follows that $\be^{-1}=\sqrt{3}-\sqrt{2}$. Therefore
        $$ \sqrt{2}=\frac12(\be-\be^{-1})\quad\text{and}\quad\sqrt{3}=\frac12(\be+\be^{-1})$$
        and so $M=\Q(\sqrt{2}+\sqrt{3})=\Q(\sqrt{2},\sqrt{3})$. 
        Hence we have the towers $M/\Q(\sqrt{2})/Q$ and $M/\Q(\sqrt{3})/Q$. 
        Let $g(x)=x^2-3$. Suppose it is not the minimal polynomial of $\sqrt{3}$ in $\Q(\sqrt{2})$.
        Then there exists $a+b\sqrt{2}\in\Q(\sqrt{2})$ such that
        $$ 0 = g(a+b\sqrt{2})=a^2+2b^2-3+2ab\sqrt{2}.$$
        But since
        \begin{equation*}
            \begin{cases}
                a^2+2b^2-3=0\\
                2ab=0
            \end{cases}
        \end{equation*}
        has no solutions it follows that no such element exists.
        Therefore $g$ is the minimal polynomial of $\sqrt{3}$ and $[M:\Q(\sqrt{2})]=\deg g=2$.
        Since $x^2-2$ is the minimal polynomial of $\sqrt{2}$ in $\Q$ we conclude that 
        $$[M:\Q]=[M:\Q(\sqrt{2})]\cdot[\Q(\sqrt{2}):\Q)]=4$$ 
        and therefore $f$ is the minimal polynomial of $\be$.

        Thus $f(x-1)$ is the minimal polynomial of $\al$ in $\Q$. 
        From $f(\be)=0$ it follows that $1=\beta(10\beta-\beta^3)$ and so $\be^{-1}=10\beta-\beta^3$.
        Hence
        $$\sqrt{2}=\frac12\left(\be-\be^{-1}\right)=\frac12\left(\be-10\be+\be^3\right)=\frac12\left(-9(\al-1)+(\al-1)^3\right)$$
        and
        $$\sqrt{3}=\frac12\left(\be+\be^{-1}\right)=\frac12\left(11(\al-1)-(\al-1)^3\right)$$

        Let $G=\text{Aut}(M)$ and take $\sigma\in G$. Then by definition $\sigma(1)=1$ and it follows by induction and the properties of isomorphism that $\sigma(a)=a$ for all $a\in\Z$.
        Since $1=\sigma(1)=\sigma(a\cdot a^{-1})=\sigma(a)\cdot\sigma(a)^{-1}=a\cdot a^{-1}$ it also follows that $\sigma\left(\frac{p}{q}\right)=\frac{p}{q}$. 
        Hence $\sigma$ restricted to $\Q$ is simply the identity map. 
        Therefore $\sigma$ is completely determined by $\sigma(\sqrt{2})$ and $\sigma(\sqrt{3})$.
        Since $0=\sigma(0)=\sigma(\sqrt{2}^2-2)=\sigma(\sqrt{2})^2-2$ the only options are $\sigma(\sqrt{2})=\pm\sqrt{2}$.
        Similarly we conclude that $\sigma(\sqrt{3})=\pm\sqrt{3}$. This gives four possible automorphism.
        Take $\sigma,\tau\in G$ such that $\sigma(\sqrt{2})=-\sqrt{2}, \sigma(x)=x$ $\forall x\in M\setminus\{\sqrt{2}\}$ and $\tau(\sqrt{3})=-\sqrt{3},\tau(x)=x$ $\forall x\in M\setminus\{\sqrt{3}\}$. 
        Since 
        $$\sigma\circ\sigma=\tau\circ\tau=\sigma\circ\tau\circ\sigma\circ\tau=e$$
        where $e$ is the identity map it follows that $G$ is isomorphic to $V_4$, the Klein four-group.

        Lastly, consider
        \begin{align*}
            \tilde{f}&=\prod_{\sigma\in G}x-\sigma(\al)\\
            &=(x-1-\sqrt{2}-\sqrt{3})(x-1+\sqrt{2}-\sqrt{3})(x-1-\sqrt{2}+\sqrt{3})\\&\qquad\qquad (x-1+\sqrt{2}+\sqrt{3}).
        \end{align*}
        Hence $\tilde{f}(x)=f(x-1)$ which we already proved is the minimal polynomial of $\al$ in $\Q[x]$.

    \end{sol}

    \begin{ex}{21.28}
        Prove $\Q(\sqrt{2},\sqrt[3]{3})=\Q(\sqrt{2}\sqrt[3]{3})=\Q(\sqrt{2}+\sqrt[3]{3})$.
        Determine the minimum polynomials of $\sqrt{2}\sqrt[3]{3}$ and $\sqrt{2}+\sqrt[3]{3}$ over $\Q$.
    \end{ex}
    \begin{proof}
        Clearly we have that $\Q(\sqrt{2}\sqrt[3]{3})\subset\Q(\sqrt{2},\sqrt[3]{3})$ 
        and $\Q(\sqrt{2}+\sqrt{3})\subset\Q(\sqrt{2},\sqrt[3]{3})$. 
        Since $x^2-2$ is irreducible (Eisenstein with $p=2$) and $x^3-3$ is irreducible (Eisenstein with $p=3$) and $(3,2)=1$ it follows that $[\Q(\sqrt{2},\sqrt[3]{3}):\Q]=6$.

        Now consider $f(x)=x^6-72$. Then $f(\sqrt{2}\sqrt[3]{3})=0$ and so $[\Q(\sqrt{2}\sqrt[3]{3}):\Q]\leq 6$.
        Suppose that $f(x)=a(x)b(x)$ in $\Q[x]$ for $a(x),b(x)$ non constant. Furthermore suppose without loss of generality that $\deg a\geq \deg b$.
        Reducing $f$ modulo $7$ we find that 
        $$\overline{f}(x)=x^6-2=x^6-9=(x^3-3)(x^3+3)=\overline{a}(x)\overline{b}(x)\in\mathbb{F}_7[x]$$
        Reducing $f$ modulo $5$ we get
        $$\overline{f}(x)=x^6-2=x^6+8=(x^4-2x^2+4)(x^2+2)=\overline{a}(x)\overline{b}(x)\in\mathbb{F}_5[x].$$
        Since $f$ modulo 5 has no cubic terms it follows that $\deg a = 6$ and $\deg b = 1$ and so $f$ is irreducible. Therefore $[\Q(\sqrt{2},\sqrt[3]{3}):\Q]=\deg f=6$
        and since $\Q(\sqrt{2}\sqrt[3]{3})\subset\Q(\sqrt{2},\sqrt[3]{3})$ it follows that $\Q(\sqrt{2}\sqrt[3]{3})=\Q(\sqrt{2},\sqrt[3]{3})$. 

        Let $\alpha=\sqrt{2},\beta=\sqrt[3]{3}$ and $\gamma=\alpha+\beta$. Let $L=\mathbb{Q}(\alpha, \beta)$, $K=\mathbb{Q}(\gamma)$ and suppose that $\alpha,\beta\not\in K$. Since if one of $\alpha,\beta$ is in $K$, we get the other one for free it follows that $L=K(\alpha)=K(\beta)$. Then the minimal polynomial of $\alpha$ in $K[X]$ is of degree 2 since $\alpha$ is a root of $X^2-2$ and we assumed $\alpha\not\in K$. Since $X^3-3$ has one real root and two non-real roots it follows that it is the minimal polynomial of $\beta$ in $K[X]$. Hence we conclude
        $$2=[K(\alpha):K]=[L:K]=[K(\beta):K]=3,$$
        clearly a contradiction. Therefore $K=L$ and so
        $\Q(\sqrt{2}\sqrt[3]{3})\subset\Q(\sqrt{2},\sqrt[3]{3}).$
    \end{proof}

    \begin{ex}{21.29}
        Take $K=\Q(\al)$ with $f^\al_\Q=x^3+2x^2+1$.
        \begin{enumerate}
            \item Determine the inverse of $\al+1$ in the basis $\{1,\al,\al^2\}$ of $K$ over $\Q$.
            \item Determine the minimal polynomial of $\al^2$ over $\Q$.
        \end{enumerate}
    \end{ex}
    \begin{sol}
        ${}$
        \begin{enumerate}
            \item Since
                \begin{align*}
                    0&=\al^3+2\al^2+1\\
                    &=(\al+1)(\al^2+\al-1)+2.
                \end{align*}
                It follows that $(\al+1)^{-1}=-\frac12(\al^2+\al-1)$.
            \item From $\al^3+2\al^2+1=0$ it follows that $\al^3=-2\al^2-1$.
                Squaring both sides we get that $\al^6=4\al^4+4\al^2+1$ or alternatively
                $$\left(\al^2\right)^3-4\left(\al^2\right)^2-4\left(\al^2\right)-1=0.$$
                By Ex. 19 we know that $\Q(\al)=\Q(\al^2)$.
                Therefore the minimal polynomial of $\al^2$ over $\Q$ has degree 3 and it follows that
                $$ f^{\al^2}_\Q(x)=x^3-4x^2-4x-1. $$
        \end{enumerate}
    \end{sol}

    \begin{ex}{21.30}
        Define the cyclotomic field $\Q(\zeta_5)$ and let $\al=\zeta_5^2+\zeta_5^3$.
        \begin{enumerate}
            \item Show that $\Q(\al)$ is a quadratic extension of $\Q$ and determine $f^\al_\Q$.
            \item Prove: $\Q(\al)=\Q(\sqrt{5})$
            \item Prove: $\cos(2\pi/5)=\frac{\sqrt{5}-1}{4}$ and $\sin(2\pi/5)=\sqrt{\frac{5+\sqrt{5}}{8}}$
        \end{enumerate}
    \end{ex}
    \begin{proof}
        ${}$ 
        \begin{enumerate}
            \item The degree of the 5th cyclotomic polynomial
            $$\Phi_5(x)=\prod_{\substack{1\leq k\leq 5\\(k,5)=1}}\left(x-e^{\frac{2\pi k}{5}i}\right)=x^4+x^3+x^2+x+1$$
            is 4 and since $\Phi_5=f^{\zeta_5}_\Q$ it follows that $[\Q(\zeta_5):\Q]=4$.
            Thus 
            $$[\Q(\zeta_5):\Q(\alpha)]\mid4.$$ 
            Note that $\zeta_5^3=\frac{1}{\zeta_5^2}=\overline{\zeta_5^2}$. 
            Hence $\al=\zeta_5^2+\zeta_5^3=\zeta_5^2+\overline{\zeta_5^2}\in\R$ and so $\Q(\al)\subsetneq\Q(\zeta_5)$.
            Together with the fact that $\zeta_5$ is a root of $x^3+x^2-\al\in\Q(\al)$ it follows that
            $$1<[\Q(\zeta_5):\Q(\al)]\leq 3\implies [\Q(\zeta_5):\Q(\al)]=2.$$
            Finally, since $\Q(\zeta_5)/\Q(\al)/\Q$ is a tower of fields and
            $$[\Q(\al):\Q]=\frac{[\Q(\zeta_5):\Q]}{[\Q(\zeta_5):\Q(\al)]}=2$$
            it follows that $\Q(\al)$ is a quadratic extension.

            Let $w=\zeta_5^2$. Then $\al=w+\frac1w$ and $\Phi_5(w)=0$ by definition of $\Phi_5$. Since $w\neq0$ it follows that
            \begin{align*}
                0&=1+w+w^2+w^3+w^4\\
                0&=\frac{1}{w^2}+\frac1w+1+w+w^2\\
                0&=\left(w+\frac1w\right)^2+w+\dfrac1w-1\\
                0&=\al^2+\al-1.
            \end{align*}
            Since $x^2+x-1$ is monic polynomial of degree 2 we conclude that $f^\al_\Q=x^2+x-1$.
        \item By construction $\al$ is a root of $x^2+x-1$ and so
            $$\al\in\left\{\frac{-1\pm\sqrt{5}}{2}\right\}.$$
            Since we can write $\al$ as polynomial in $\sqrt{5}$ and vice versa it follows that $\Q(\al)=\Q(\sqrt{5})$.
        \item Let $\zeta_5=e^{\frac{6\pi}{5}i}$. Then $w=\zeta_5^2=e^{\frac{2\pi}{5}i}$ and 
            $$\cos\frac{2\pi}{5}=\frac{w+\overline{w}}{2}=\frac{\al}{2}.$$
            Thus $2\cos\frac{2\pi}{5}$ is a root of $f^\al_Q$. Since $\frac{2\pi}{5}$ is in the first quadrant, $\cos\frac{2\pi}{5}$ is positive and so
            $$\cos\frac{2\pi}{5}=\frac{-1+\sqrt{5}}{4}.$$
            Therefore we also have
            \begin{align*}
                \sin\frac{2\pi}{5}&=\sqrt{1-\cos^2\frac{2\pi}{5}}\\
                &=\sqrt{\frac{5+\sqrt{5}}{8}}.
            \end{align*}
        \end{enumerate}
    \end{proof}

    \begin{ex}{21.31}
        Let $\overline{K}$ be an algebraic closure of $K$ and $L\subset\overline{K}$ a field that contains $K$. Prove that $\overline{K}$ is an algebraic closure of $L$.
    \end{ex}
    \begin{proof}
        Let $\overline{L}=\{\alpha\in\overline{K}\mid \alpha\text{ algebraic over }\}$ be the algebraic closure of $L$.
        By definition we have that $\overline{L}\subset\overline{K}$ so it is left to show the other inclusion. Let $\alpha\in\overline{K}$.
        Then $\alpha$ is algebraic over $K$ by definition, and so there exists $f\in K[x]$ such that $f(\alpha)=0$. Then $f\in L[x]$ since $K\subset L$ and so $\alpha$ is algebraic over $L$.
        Therefore $\alpha\in\overline{L}$ and so $\overline{L}=\overline{K}$.
    \end{proof}

    \begin{ex}{21.32}
        Let $K\subset L$ be a field extension and $\overline{K}$ the algebraic closure of $K$ in $L$. 
        Prove that every $\alpha\in L\setminus\overline{K}$ is transcendental over $\overline{K}$.
    \end{ex}
    \begin{proof}
        Suppose there exists $\alpha\in L\setminus\overline{K}$ that is algebraic over $\overline{K}$.
        Let
        $$f(x)=x^n+a_{n-1}x^{n-1}+\cdots+a_0$$
        be the minimal polynomial of $\alpha$ in $\overline{K}[x]$. Let 
        $$K_1=K(a_0,\dots,a_{n-1})\aand K_2=K_1(\alpha).$$
        Then $K_1/K$ is an algebraic extension since $a_0,\dots,a_n\in\overline{K}$ and $K_2/K_1$ is algebraic since $f\in K_1[x]$.
        So we have the tower of fields $K_2/K_1/K$  and it follows that $K_2/K$ is an algebraic extension and so $\alpha$ is algebraic over $K$.
        By definition of algebraic closure, $\alpha\in\overline{K}$ which contradicts our assumption. 
        Therefore $\alpha$ must be transcendental over $\overline{K}$.
    \end{proof}

    \begin{ex}{21.35}
        Let $f\in K[x]$ be a polynomial of degree $n\geq 1$. Prove: $[\Omega_K^f:K]$ divides $n!$. 
    \end{ex}
    \begin{proof}
        If $n=1$, the $K$ is splitting field of $f$ and $[K:K]=1$ divides $n!=1$.
        Suppose the statement holds for some $n\geq 1$. There are two cases
        \begin{enumerate}
            \item Suppose $f$ is irreducible, $\deg f = n+1$ and $\alpha$ is a root of $f$.
                Then $K(\alpha)$ is an extension of degree $n+1$ and $f(x)=(x-\alpha)g(x)\in K(\alpha)[x]$.
                Let $M$ be the splitting field of $g$ over $K(\alpha)$. 
                Then $[M:K(\alpha)]$ divides $n!$ by the induction hypothesis.
                But $M$ is also the splitting field of $f$ over $K$ and so
                $$[M:K]=[M:K(\alpha)][K(\alpha):K]=[M:K(\alpha)](n+1)$$
                which divides $(n+1)!$.
            \item Suppose $f$ is a reducible polynomial of degree $n+1$. Let $f(x)=h(x)g(x)$ with $h$ irreducible. 
                Construct the tower of fields $M/L/K$ such that $L$ is the splitting field of $h$ over $K$ and $M$ is the splitting field of $g$ over $L$.
                Then $[L:K]$ divides $\deg h!$ and $[M:L]$ divides $\deg g!$ by induction hypothesis. 
                Hence 
                $$[M:K]=[M:L][L:K]$$ 
                divides $\deg h!\cdot\deg g!$.
                And since 
                $${n+1\choose \deg h}=\frac{(n+1)!}{\deg h!(n+1-\deg h)!}=\frac{(n+1)!}{\deg h!\deg g!}$$
                is an integer it follows that $[M:K]$ divides $(n+1)!$.
        \end{enumerate}
    \end{proof}

    \begin{ex}{21.36}
        Let $d\in\Z$ be an integer that is not a third power in $\Z$. Prove that the splitting field $\Omega_\Q^{x^3-d}$ has degree 6 over $\Q$.
        What is the degree if $d$ is a third power?
    \end{ex}
    \begin{proof}
        Let $f(x)=x^3-d$. Suppose $f$ has a root in $r/s\in\Q$. Then $s\mid 1$ and $r\mid d$ so $f(r)=r^3-d=0\implies d=r^3$ which contradicts our assumption.
        Therefore $f(x)$ has no roots in $\Q$ and since $\deg f=3$ it follows that $f$ is irreducible. Then $\Q[X]/(f)$ is a field and $\alpha\equiv x\mod\left(x^3-d\right)$ is a zero of $f$.
        Therefore $f$ splits in $\Q(\alpha)$ as
        $$x^3-d=(x-\alpha)(x^2+\alpha x+\alpha^2)$$
        and $$[\Q(\alpha):\Q]=3$$. 
        Let $h(x)=x^2+x+1$. Then $h(x+1)=x^2+3x+3$ is irreducible in $\Q[x]$ (Eisenstein with $p=3$) and so $h(x)$ is irreducible in $\Q[x]$.
        Since $\Q[x]/(h)$ is a quadratic extension, it cannot be a subfield of the cubic extension $\Q[x]/(f)$ and so $h(x)$ has no zeros in $\Q(a)$.
        Hence it is irreducible in $\Q(a)[x]$. 
        It follows that  $\Q(\alpha)[x]/(h)\cong\Q(\alpha)(\beta)$ is a quadratic extension for $\beta\equiv x\mod\left(x^2+x+1\right)$.
        Then 
        $$f(x)=(x-\alpha)(x-\alpha\beta)(x+\alpha\beta+\alpha)$$
        and so $\Q(\alpha, \beta)$ is the splitting of $f$.
        Moreover
        $$[\Q(\alpha, \beta):\Q]=[\Q(\alpha,\beta):\Q(\alpha)]\cdot[\Q(\alpha):\Q]=2\cdot 3= 6$$
        as desired.

        If $d=r^3$ for some $r\in\Z$ then $f(x)$ is reducible since
        $$f(x)=(x-r)(x^2+rx+r^2)\in\Q[x].$$
        Then $r\beta$ is a root of $X^2+rx+r^2$ and so the quadratic extension $\Q(\beta)\cong\Q[x]/(x^2+x+1)$ is the splitting field of $f$.
    \end{proof}

    \begin{ex}{21.37}
        Determine the degree of the splitting field of $x^4-2$ over $\Q$.
    \end{ex}
    \begin{sol}
        Since
        $$x^4-2=(x-\sqrt[4]{2})(x+\sqrt[4]{2})(x-\sqrt[4]{2}i)(x+\sqrt[4]{2}i),$$
        the splitting field of $x^4-2$ is $\Q(\sqrt{2}, i)$.
        We know that $[\Q(\sqrt{2}):\Q]=2$ and $[\Q(i):\Q]=2$. Therefore the degree of $\Q(\sqrt{2},i)$ over $\Q(\sqrt{2})$ is less than 2.
        It can't be 1 since $\Q(\sqrt{2})\subset\R$ and $i\not\in\R$ and so $[\Q(\sqrt{2},i):\Q(\sqrt{2})]=2$. Therefore
        $$[\Q(\sqrt{2},i):\Q]=[\Q(\sqrt{2},i):\Q(\sqrt{2}][\Q(\sqrt{2}):\Q]=4.$$ 
    \end{sol}
    
    \begin{ex}{21.38}
        Determine the degree of the splitting field of $x^4-4$ and $x^4+4$. Explain why the notation $\Q(\sqrt[4]{4})$ and $\Q(\sqrt[4]{-4})$  is not used for the fields obtained through the adjunction of a zero of, respectively, $x^4-4$ and $x^4+4$ to $\Q$.
    \end{ex}
    \begin{sol}
        Note that
        $$ x^4-4=\left(x+\sqrt{2}\right)\left(x-\sqrt{2}\right)\left(x+\sqrt{2}i\right)\left(x-\sqrt{2}i\right).$$
        Since $i=\left(\sqrt{2}\right)^{-1}\sqrt{2}i$ the splitting field of $x^4-4$ is $\Q\left(\sqrt{2},\sqrt{2}i\right)=\Q\left(\sqrt{2},i\right)$. 
        Similarly
        $$x^4+4=\left(x-1-i\right)\left(x-1+i\right)\left(x+1-i\right)\left(x+1+i\right),$$
        and so the splitting filed of $x^4+4$ is $\Q(i)$. To compute the degree of the splitting fields note that:
        \begin{enumerate}
            \item $(x+1)^2+1$ is irreducible in $\Q[x]$ (Eisenstein with $p=2$) hence $x^2+1$ is irreducible in $\Q[x]$ and so $x^2+1$ is the minimal polynomial of $i$ over $\Q$.
            \item $x^2-2$ is the minimal polynomial of $\sqrt{2}$ over $\Q$ (Eisenstein with $p=2$)
            \item $\left[\Q\left(\sqrt{2},i\right):\Q\left(\sqrt{2}\right)\right]\leq 2$ since the minimal polynomial of $i$ over $\Q$ is of degree two by (1). 
            However $\Q\left(\sqrt{2}\right)\subset\R$ and $i\not\in\R$ so the degree cannot be one. Therefore $\left[\Q\left(\sqrt{2},i\right):\Q\left(\sqrt{2}\right)\right]=2$.
        \end{enumerate}
        It follows that
        $$[\Q(i):\Q]=2$$
        and
        $$\left[\Q\left(\sqrt{2}, i\right):\Q\right]=\left[\Q\left(\sqrt{2}, i\right):\Q\left(\sqrt{2}\right)\right]\left[\Q\left(\sqrt{2}\right):\Q\right]=4.$$   

        Outside the fact that the notation $\Q\left(\sqrt[4]{-4}\right)$ is ambiguous (which of the four roots does it stand for?), it is also misleading.
        It might seem like an extension of degree four, but as shown above it is of degree 2, regardless of which of the roots you assign to $\sqrt[4]{-4}$.
        Similarly the degree of $\Q\left(\sqrt[4]{4}\right)$ is two and not four since $\sqrt[4]{4}=\sqrt{2}$. 
        Therefore it is clearer and to simply write $\Q(\sqrt{2})$ and $\Q(1+i)$ for the adjunction of a zero of, respectively, $x^4-4$ and $x^4+4$ to $\Q$.
    \end{sol}
\subsection{Finite Fields}
\begin{ex}{22.6}
    Give an explicit isomorphism $\F_5[x]/(x^2+x+1)\xrightarrow\sim \F_5(\sqrt{2})$
\end{ex}
\begin{sol}
    Let $\varphi:\F_5[x]/(x^2+x+1)\xrightarrow \F_5(\sqrt{2})$ be an isomorphism and $\alpha$ the equivalence class of $x$ in $\F_5[x]/(x^2+x+1)$. 
    Since $\varphi$ is identity on $\F_5$, we only need to find where $\alpha$ is mapped to.
    Suppose
    $$\varphi(\alpha)=c+d\sqrt{2}.$$
    Then $\left(c+d\sqrt{2}\right)^2+c+d\sqrt{2}+1=0$. Hence 
    \begin{equation*}
        \begin{cases}
            d(2c+1)=0\\
            c^2+2d^2+c+1=0\\
        \end{cases}.
    \end{equation*}
     Since $d=0$ would be a contradiction it follows from the first equation that $c=2$.
     Substituting into the second we get $4+2d^2+3=2d^2+2$ and so $2d^2=3$. Therefore $d=2,3$ and either value will give us an isomorphism.
     So let
     $$\varphi(a+b\alpha)=a+b(2+2\sqrt{2})= 2a+2b\sqrt{2}.$$
\end{sol}

\begin{ex}{22.7}
    
\end{ex}

\begin{ex}{22.8}
    
\end{ex}

\begin{ex}{22.11}
    Let $p$ be a prime. Show that $\F_p(x)/(x^2+x+1)$ is a field if and only if $p\equiv 2\mod 3$.
\end{ex}
\begin{proof}
    $(\Rightarrow)$ Suppose $\F_p(x)/(x^2+x+1)$ is a field. So $f(x)=x^2+x+1$ is irreducible in $\F_p[x]$. 
    Therefore $\F_p$ does not contain a non-trivial cube root of unity and so 3 doesn't divide $\abs{\F_p^*}=p-1$. 
    Since $f(x)=(x+2)^2$ in $\F_3[x]$ it can't be the case that $p$ is congruent to $0\mod 3$ and so $p\equiv 2\mod 3$. 
    
    $(\Leftarrow)$ Suppose $\F_p(x)/(x^2+x+1)$ is not a field. Then $f(x)=x^2+x+1$ is reducible in $\F_p[x]$ and so $f$ has a root $\alpha\in\F_p$.
    Then $\alpha\neq 0$ and $\alpha^3=1$. Therefore 3 divides $\abs{\F_p^*}=p-1$ and so $p\equiv 1\mod 3$.

\end{proof}

\begin{ex}{22.12}
    Let $q$ be a prime power.
    \begin{enumerate}
        \item For what $q$ is the quadratic extension $\F_{q^2}$ of $\F_q$ of the form $\F_q(\sqrt{x})$ with $x\in\F_q$?
        \item For what $q$ is the cubic extension $\F_{q^3}$ of $\F_q$ of the form $\F_q(\sqrt[3]{x})$ with $x\in\F_q$?
    \end{enumerate}
\end{ex}
\begin{sol}
    ${}$
    \begin{enumerate}
        \item Let $\varphi:\F_q\to\F_q$ be given by $\varphi(a)=a^2$. If $q=p^n$ is even, then $p=2$ and $\varphi$ is a field isomorphism.
            Therefore $\F_q(\sqrt{b})=\F_q$ for all $x\in\F_q$. If $q$ is odd, then $(-1)^2=1=1^2$ so the map is not injective and so it's not surjective.
            Therefore there exists $b\in\F_q$ such that $\sqrt{b}\not\in\F_q$. Then $x^2-b$ is the minimal polynomial of $b$ and so $\F_q(\sqrt{b})$ is a quadratic extension.
            Hence $\F_q(\sqrt{b})=\F_{q^2}$.
        \item Let $\varphi:\F_q\to\F_q$ be given by $\varphi(a)=a^2$. 
            If $q\equiv 0\mod 3$ then $\varphi$ is a field isomorphism and so $\F_q(\sqrt[3]{b})=\F_q$ for all $b\in\F_q$.
            If $q\equiv 1\mod 3$ then $\abs{\F_q^*}\equiv 0\mod 3$. 
            Hence there exists an element $a\in\F_q$ of order three so $\varphi(a)=1=\varphi(1)$ and so $\varphi$ is not injective.
            It follows that there exists $b\in\F_q$ such that $\sqrt[3]{b}\not\in\F_q$. Then $F_q(\sqrt[3]{b})$ is a cubic extension and so it is equal to $F_{q^3}$.
            Lastly if $q\equiv 2\mod 3$ then $\abs{\F_q^*}\equiv 1\mod 3$ and so there is no element of order three in $\F_q$.
            Therefore $\varphi$ is injective hence surjective and so every element has a cube root. It follows that $\F_q(\sqrt[3]{b})=\F_q, \forall b\in\F_q$. 
    \end{enumerate} 
\end{sol}

\begin{ex}{22.13}
    
\end{ex}

\begin{ex}{22.15}
    Determine all the primes for which $\F_p[x]/(x^4+1)$ is a field. 
\end{ex}
\begin{sol}
    If $p=2$ then $x^4+1=(x^2+1)^2=(x+1)^4$ so $p$ is odd. 
    Then $x^4+1$ divides $x^8-1$ in $\F_p[x]$. 
    Since $p^2-1=(p+1)(p-1)$ is a product of two consecutive even numbers it follows that $8\mid p^2-1$.
    Hence $x^8-1$ splits completely in $\F_{p^2}[x]$. So
    $$x^8-1=(x-1)(x-\beta)\cdots(x-\beta^7)=(x^4+1)(x^4-1)$$
    for some $\beta\in\F_{p^2}$. 
    If $x^4+1$ is irreducible in $\F_p[x]$, then for any root $\alpha$ of $x^4+1$, $\F_p(\alpha)$ is an extension of degree four.
    But $x^4+1$ splits completely in a quadratic extension and so it is reducible in $\F_p[x]$.

    \noindent\textbf{Alternative solution}
    
    If $-1$ is a square in $\F_p$ then $a^2=-1$ for some $a\in\F_p$. So
    $$x^4+1=x^4-a^2=(x^2+a)(x^2-a).$$
    If $2$ is a square in $\F_p$ then $b^2=2$ for some $b\in\F_p$ and so
    $$x^4+1=(x^2+1)^2-(bx)^2=(x^2+1+bx)(x^2+1-bx).$$
    Lastly, if neither $-1$ nor $2$ are squares in $\F_p$, then $p$ is odd (since $-1=1=1^2$ in $\Z/2\Z$).
    Then $\F_p^*=\{1,\alpha, \alpha^2,\dots,\alpha^{p-1}\}$ is cyclic subgroup of even order. 
    Since $-1$ and $2$ are odd powers of $\alpha$, it follows that their product $-2$ is an even power of $\alpha$ and so it is a square.
    So let $c\in\F_p$ such that $c^2=-2$. Then
    $$x^4+1=(x^2-1)-(cx)^2=(x^2-1-cx)(x^2-1+cx).$$
    Therefore $x^4+1$ is reducible modulo every prime.
\end{sol}
\subsection{Separable and Normal Extensions}
    
\end{document}