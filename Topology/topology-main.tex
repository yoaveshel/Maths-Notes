\documentclass{article}

\usepackage[utf8]{inputenc}
\usepackage{csquotes}
\usepackage[english]{babel}
\usepackage{amsmath,amssymb,amsthm,textcomp}
\usepackage{mathtools}
\usepackage{biblatex}
\usepackage{tikz}
\usepackage{graphics, setspace}
\usepackage{listings}
\usepackage{lipsum}
\usepackage{hyperref}
\hypersetup{
    colorlinks,
    citecolor=black,
    filecolor=black,
    linkcolor=black,
    urlcolor=black
}

\DeclareMathAlphabet{\pazocal}{OMS}{zplm}{m}{n}
\DeclareMathOperator{\Ima}{Im}
\newcommand{\Ba}{\mathcal{B}}
\newcommand{\Ta}{\mathcal{T}}
\newcommand{\Aa}{\mathcal{A}}
\newcommand{\R}{\mathbb{R}}
\newcommand{\C}{\mathbb{C}}
\newcommand{\Z}{\mathbb{Z}}
\newcommand{\Q}{\mathbb{Q}}
\newcommand{\N}{\mathbb{N}}
\newcommand{\p}{\mathbb{P}}
\newcommand{\Ss}{\mathcal{S}} % Schwartz space
\newcommand{\F}{\mathcal{F}} % Fourier Transform
\newcommand{\Rf}{\mathcal{R}} % reflection
\newcommand{\E}{\mathbb{E}}

\DeclarePairedDelimiter\abs{\lvert}{\rvert}%
\DeclarePairedDelimiter\norm{\lVert}{\rVert}%
% Swap the definition of \abs* and \norm*, so that \abs
% and \norm resizes the size of the brackets, and the 
% starred version does not.
\makeatletter
\let\oldabs\abs
\def\abs{\@ifstar{\oldabs}{\oldabs*}}
%
\let\oldnorm\norm
\def\norm{\@ifstar{\oldnorm}{\oldnorm*}}
\makeatother

\newtheorem{theorem}{Theorem}[section]
\newtheorem{corollary}{Corollary}[theorem]
\newtheorem{lemma}[theorem]{Lemma}
\newtheorem*{definition}{Definition}
\newtheorem*{remark}{Remark}

\theoremstyle{remark}
\newtheorem*{sol}{Solution}

\newenvironment{ex}[1]
    {\noindent\textbf{Exercise #1}\normalsize\newline}
    {\vspace{0.5 em}}

\title{Topology - X400416}
\author{Yoav Eshel}
\date{\today}

\begin{document}
\maketitle
\tableofcontents
\newpage
\normalsize
These notes are based on Topology ($2^\text{nd}$ edition) by James R. Munkres. 
\section{Topological Spaces}
The motivation behind defining a topological space is to generalize the notion of a metric space.
Recall that a metric on a set $X$ is a map $d:X\times X \to [0,\infty)$ that satisfies
\begin{enumerate}
    \item $d(x,y)=d(y,x)$
    \item $d(x,x)=0$
    \item $d(x,y)>0, x\not=y$
    \item $d(x,y)\leq d(x,z)+d(z,y)$
\end{enumerate}
Then we say that a set $U\subset X$ is open if for all $x\in U$ and some $r>0$ 
$$
    B(x,r):=\{y\in X\mid d(x,y)<r\}\subset U.
$$
In other words, around every point in $U$ there is a "ball" that is contained in $U$. 
In Analysis I one learns about continuity using the classic $\varepsilon$-$\delta$ definition which requires a metric.
As it turns out, we don't really need a metric to define a continuous function, only open sets:
\begin{definition}
    A function between metric spaces is continuous if and only if the preimage of an open set is open.
\end{definition}
Using this definition, different seeming metric can yield the same notions of which functions are continuous! 
We call the collection of open subsets of $X$ defined by some metric $d:X\times X\to [0,\infty)$ a \textit{topology}.
This open sets satisfy some important properties. Namely: (1) $X$ and $\varnothing$ are open, (2) arbitrary unions of open sets are open and (3) finite intersections of open sets is open.
It turns out that a metric is not required to define a topology, only these three properties:
\begin{definition}
    Let $X$ be a set. Then a topology on $X$ is a set $\mathcal{T}\subset\mathcal{P}(x)$ such that
    \begin{enumerate}
        \item $\varnothing\in\Ta, X\in\Ta$
        \item If $\{U_{\alpha}\}\subset\Ta$ then  $\bigcup_{\alpha}U_{\alpha}\in\Ta$
        \item If $\{U_i\}_{i=0}^n\subset\Ta$ then $\bigcap_{i=0}^n U_i\in\Ta$
    \end{enumerate}
    A topological space is the pair $(X,\Ta)$
\end{definition}
Then we say that $U\subset X$ is open if $U\in\Ta$. 
Note that which sets are open depends on the topology, which might conflict with your notion of open set as defined above. 
For example, in the topology $\mathcal{P}(\R)$ every subset of the real line is open, while in the topology $\{\varnothing, \R\}$ only the empty set and $\R$ are open.
We often say that $U$ is open in $X$ without giving a specific topology, which simply means that the statement that follows will hold for any topology we define on $X$ and any element in that topology.

If $\Ta$ and $\Ta'$ are two topologies on $X$ such that $\Ta\subseteq\Ta'$ than we say that $\Ta'$ is \textit{finer} (or \textit{strictly finer} if the containment is proper) than $\Ta$. 
We similarly say that $\Ta$ is \textit{coarser} (or \textit{strictly coarser}) than $\Ta'$.
It might also be that case that two topologies are not \textit{comparable}.

\section{Basis for a Topology}
Specifying topologies directly is often not possible, due the enormous size of many topologies.
So we often define a topology using a smaller subset called a \textit{basis}.
\begin{definition}
    If $X$ is a set, then a \textit{basis} of a topology is a collection $\mathcal{B}$ of subsets of $X$ such that
    \begin{enumerate}
        \item $\forall x\in X, \exists B\in\mathcal{B}$ such that $x\in B$
        \item If $x\in B_1\cap B_2$ with $B_1,B_2\in\mathcal{B}$ then there exists $B_3\in\mathcal{B}$ with $x\in B_3\subset B_1\cap B_2$.
    \end{enumerate}
\end{definition}
If $\Ta$ is a topology generated by a basis $\Ba$ then $U$ is open if for all $x\in U$ there exists $B\in\Ba$ such that $x\in B\subset U$. 
Also $B\in\Ta$ for all $B\in\Ba$. The proof that $\Ta$ is indeed a topology is not included.
An alternative construction of a topology from a basis is given by the following lemma
\begin{lemma}
    Let $X$ be a set and $\Ba$ a basis for a topology $\Ta$. Then $\Ta$ equals the collection of all unions of elements in $\Ba$.
\end{lemma}
Using this lemma is sometime easier in practice; given a basis $\Ba$ and $U\subset X$, if one can write $U$ as a union of elements in $\Ba$ then $U$ is open.

We can also go in the reverse direction: from topology to a basis.
\begin{lemma}
    Let $X$ be a topological space. If $\mathcal{C}$ is a collection of open sets such that for each open set $U$ and each $x\in U$ there is $C\in\mathcal{C}$ such that $x\in C\subset U$, then $\mathcal{C}$ is a basis.
\end{lemma}

When topologies are given in terms of basis, we can already determine which one is finer using the following criterion
\begin{lemma}
    Let $\Ba$ and $\Ba'$ be bases for the topologies $\Ta$ and $\Ta'$ respectively. Then
    $\Ta\subset\Ta'$ if and only if for each $x\in X$ and $B\in\Ba$ containing $x$ there exists $B'\in\Ba'$ such that $x\in B'\subset B$.
\end{lemma}
It might be tricky to remember the direction of the inclusion. One way to think about it is since $\Ta'$ has more subsets of $X$ it needs to have smaller basis elements.

Lastly, we define the notion of a \textit{subbasis}.
\begin{definition}
    A subbasis $\Ss$ for a topology on $X$ is a collection of subsets of $X$ whose union equals $X$. The topology generated by $\Ss$ is the collection of all unions of finite intersection of elements of $\Ss$.
\end{definition}

To conclude this section we define 3 topologies on the real line using the notion of a basis:
\begin{enumerate}
    \item The \textit{standard topology} generated by the collection of all open intervals $(a,b)$ with $a<b$ (it is not a recursive definition. Here we use open in the familiar metric sense).
    \item The \textit{lower limit topology} is generated by half-open intervals $[a,b)$. When $\R$ is given in the lower limit topology we denote it $\R_l$.
    \item The \textit{K-topology} is generate by open intervals and sets of the form $(a,b)-K$ where $K=\{1/n\mid n\in\N\}$. When $\R$ is given in this topology we denote it $\R_k$.
\end{enumerate}
One maybe surprising property is that both $\R_l$ and $\R_k$ are finer than the standard topology, but are not comparable with one another.

\section{The Product Topology on $X\times Y$}

\section{The Subspace Topology}
If $X$ is a topological space with topology $\Ta_x$ and $Y\subset X$ then the subspace topology on $Y$ is defined as
$$
    \Ta_Y=\{Y\cap U\mid U\in\Ta_X\}.
$$
The fact that the collection $\Ta_Y$ has all the properties of a topology follows from the $\Ta_X$ being a topology.
Then if $\Ba_X$ is a basis for a topology on $X$, the basis of the subspace topology on $Y$ is given by
$$
    \Ba_Y=\{Y\cap B\mid B\in\Ba_X\}.
$$

Open sets in the subspace topology on $Y$ are not necessarily open in $X$.
If $A$ is an open set in $Y$, then $A$ is open in $X$ if $Y$ is open in $X$.

\section{Closed Sets and Limit Points}
If $X$ is a topological space and $A\subset X$ then $A$ is closed if $X-A$ is open.
Closed sets have similar properties to open sets:
\begin{theorem}
    Let $X$ be a topological space. Then
    \begin{enumerate}
        \item $X$ and $\varnothing$ are closed
        \item Arbitrary intersections of closed sets are close
        \item Finite unions of closed sets are closed.
    \end{enumerate}
\end{theorem}
One can just as well define a topology in terms of closed sets, but the definition using open sets is much more common.
Of course, mathematics wouldn't be fun if there wasn't any space for confusion. In a topology, a set can be open, closed, neither or both. 
So don't think of sets as doors. 

If $Y$ is a subspace of a topological space $X$, then a closed set in $X$ is not necessarily closed in $Y$.
A set $A$ is closed in $Y$ if and only if it equals the intersection of a closed set of $X$ with $Y$. 
This is easy to verify since if $A$ is closed in $Y$ then $Y-A$ is open in $Y$ and so it equals $U\cap Y$ for some open set $U$ in $X$. 
Then $X-U$ is closed and $A=Y\cap(X-U)$. The other direction is similarly proved.

Let $A\subset X$. THen the smallest closed set that contain $A$ is called the closure of $A$ and is denoted $\bar{A}=\bigcap_{C\text{ closed}, A\subset C} C$.
The smallest open set containing $A$ is called the interior of $A$ and is denoted $\text{Int} A = \bigcup_{U\text{ open}, U\subset A}$. Then clearly
$$
    \text{Int}A\subset A\subset\bar{A}.
$$
A point $x\in X$ is in the closure of $A$ if and only if for any basis element $B$ containing $x$ the intersection $A\cap B$ is non empty.

To add to the soup of metaphors, we define the \textit{neighborhood} of $x\in X$ is an open set $U$ containing $x$. Then a \textit{limit point} of $A\subset X$ is defined as
\begin{definition}
    A limit point of $A\subset X$ is a point $x\in A$ such that every neighborhood of $x$ intersects $A$ in a point different than $x$
\end{definition}
In other words, $x$ is a limit point of $A$ if for every open set $U$ containing $x$ the intersection $U\cap(A-{x})$ is non empty. Now denote the set of limit points of $A$ by $A'$.
Then $\bar{A}=A'\cup A$. This gives us an alternative condition for a set being closed. Namely, if $A'\subset A$ then $\bar{A}=A$ and so $A$ is closed.
Alternatively, if $A$ is closed, then $\bar{A}=A$ and so $A'\subset A$.
Hence a set is closed if and only if it contains all of its limit points.

We conclude this section with a definition of convergence in a topological space.
\begin{definition}
    Let $(X,\Ta)$ be a topological space. If $x_n$ is a sequence in $X$ then $x_n\to x$ as $n\to\infty$ if
    $$
        \forall U\in\Ta, x\in U\,\exists N\in\N: \{x_n\}_{n\geq N}\subset U
    $$
\end{definition}
Note that this definition does not require any metric. The downside is that limit are not necessarily unique and sequences can converge to any number of points.
To avoid this horrific phenomenon, we add an extra condition to rid our definition of topological spaces of such situations. Topological spaces in which limits are unique are called Hausdorff spaces.


\newpage
\section{Exercises}
\subsection{Symmetric Polynomial}
    \begin{ex}{14.10}
        Express the symmetric polynomials $\sum_n T_1^2T_2$ and $\sum_{n} T_1^3T_2$ in the elementary symmetric polynomials.
    \end{ex}
    \begin{sol}
        To get the polynomial $\sum_n T_1^2T_2$ we start with
        $$
            s_1s_2=\sum_n T_1\sum_n T_1T_2 = \sum_n T_1^2T_2+3\sum_n T_1T_2T_3 = \sum_n T_1^2T_2+3s_3
        $$
        Thus 
        $$
            \sum_n T_1^2T_2 = s_1s_2-3s_3
        $$

        Similarly, to transform the polynomial $\sum_{n} T_1^3T_2$ we start with
        \begin{align*}
            s_1^2s_2&=\left(\sum_nT_1\right)^2\sum_nT_1T_2\\
            &=\left(\sum_n T_1^2+2\sum_n T_1T_2\right)\sum_nT_1T_2\\
            &=\sum_nT_1^2\sum_n T_1T_2+2s_2^2\\
            &=\sum_nT_1^3T_2+\sum_n T_1^2T_2T_3+2s_2^2.
        \end{align*}
        And since
        $$
            s_1s_3=\sum_nT_1\sum_nT_1T_2T_3=\sum_nT_1^2T_2T_3+4\sum_n T_1T_2T_3T_4
        $$
        it follows that $\sum_n T_1^2T_2T_3=s_1s_3-4s_4$ and so
        $$
            \sum_{n} T_1^3T_2=s_1^2s_2-s_1s_3+4s_4-2s_2^2
        $$
    \end{sol}
        
    \begin{ex}{14.21}
        Express $p_4=\sum_nT_1^4$ in elementary symmetric polynomials
    \end{ex}
    \begin{sol}
        Let $n\geq 4$. Starting with
        \begin{align*}
            s_1^4 &= \left(\sum_nT_1\right)^4\\& = \sum_n T_1^4+4\sum_n T_1^3T_2+12\sum_n T_1^2T_2T_3+6\sum_nT_1^2T_2^2+24\sum_nT_1T_2T_3T_4.
        \end{align*}
        To understand how to coefficients of the sum are obtained, consider the number of ways the $T_i$ can be arranged. 
        For example, $T_1^4=T_1T_1T_1T_1$ can only be arranged in 1 way but $T_1^2T_2T_3=T_1T_1T_2T_3$ can be arrange in $\frac{4!}{2}=12$ ways (where we divided by 2 since the two $T_1$ can be swapped in any given arrangement).
        Then
        $$
            s_1^2s_2=\left(\sum_n T_1\right)^2s_2=\left(\sum_nT_1^2+2\sum_n T_1T_2\right)s_2 = \sum_n T_1^3T_2+\sum_nT_1^2T_2T_3+2s_2^2.
        $$
        So far we have
        \begin{align*}
            p_4 &= s_1^4-4\left(s_1^2s_2-2s_2^2-\sum_nT_1^2T_2T_3\right)-12\sum_n T_1^2T_2T_3-6\sum_nT_1^2T_2^2-24\sum_nT_1T_2T_3T_4\\
            &=s_1^4-4s_1^2s_2+8s_2^2-24s_4-6\sum_nT_1^2T_2^2-8\sum_n T_1^2T_2T_3.
        \end{align*}
        So continuing with $\sum_nT_1^2T_2^2$ we get
        $$
            s_2^2 = \left(\sum_n T_1T_2\right)^2=\sum_n T_1^2T_2^2+2\sum_n T_1^2 T_2T_3+6\sum_n T_1T_2T_3T_4.
        $$
        Finding the coefficients here is slightly trickier since $s_2$ contains pairs not all arrangements are allowed. 
        For example, $T_1^2T_2^2$ can only come from the pair $T_1T_2$. On the other hand $T_1T_2T_3T_4$ can come from $T_1T_2$ and $T_3T_4$ or $T_1T_4$ and $T_2T_3$ and so on.
        We choose the first pair (${4\choose 2}=6$ ways) which also fixes the second pair and so there are 6 ways to get $T_1T_2T_3T_4$.
        Hence
        \begin{align*}
            p_4 &= s_1^4-4s_1^2s_2+8s_2^2-24s_4-6\left(s_2^2-2\sum_nT_1^2T_2T_3-6s_4\right)-8\sum_n T_1^2T_2T_3\\
            &=s_1^4-4s_1^2s_2+2s_2^2+12s_4+4\sum_n T_1^2T_2T_3.
        \end{align*}
        Using Exercise 14.10 we get
        \begin{align*}
            p_4 &=s_1^4-4s_1^2s_2+2s_2^2+12s_4+4(s_1s_3-4s_4)\\
            &=s_1^4-4s_1^2s_2+2s_2^2-4s_4+4s_1s_3
        \end{align*}
    \end{sol}

    \begin{ex}{14.22}
        A rational function $f\in\Q[T_1,\dots,T_n]$ is called symmetric if it is invariant under all permutations of the variables $T_i$. Prove that every symmetric rational function is a rational function in the elementary symmetric functions.
    \end{ex}
    \begin{proof}
        Let $f\in\Q[T_1,\dots,T_n]$ be a symmetric rational function. 
        Then $f=g/h$ for $g,h$ polynomials. If $h$ is a symmetric polynomial then $g=fh$ is symmetric as well.
        By the fundamental theorem of symmetric polynomial both $g$ and $h$ can be written in terms of elementary symmetric polynomials and we're done.
        If $h$ is not symmetric, then let 
        $$\tilde{h}=\prod_{\sigma\in S_n\setminus\{e\}}\sigma(h)$$
        and then $h\tilde{h}$ is symmetric so $f=\frac{g\tilde{h}}{h\tilde{h}}$ which is again the case above.
    \end{proof}

    \begin{ex}{14.23}
        Write $\sum_{n}T_1^{-1}$ and $\sum_n T_1^{-2}$ as rational functions in $\Q[s_1,\dots,s_n]$
    \end{ex}
    \begin{sol}
        Starting with
        $$
            \sum_{n}T_1^{-1}=\frac{1}{T_1}+\cdots+\frac{1}{T_n}.
        $$
        We multiply by $1=\frac{s_n}{s_n}$ and simplify
        \begin{align*}
            \frac{s_n}{s_n}\sum_{n}T_1^{-1}&=\frac{T_1T_2\cdots T_n}{T_1T_2\cdots T_n}\left(\frac{1}{T_1}+\cdots+\frac{1}{T_n}\right)\\
            &=\frac{s_{n-1}}{s_n}
        \end{align*}

        For the second expression we present to approaches.
        \begin{enumerate}
            \item Observing that 
                $$\left(\sum_n T_1^{-1}\right)^2=\sum_{n} T_1^{-2}+2\sum_{n}T_1^{-1}T_2^{-1}$$
            we can write using the previous part
                $$ \sum_n T_1^{-2} = \frac{s_{n-1}^2}{s_n^2}-2\sum_{n}T_1^{-1}T_2^{-1}$$
            and multiplying by the second term by $\frac{s_{n}}{s_{n}}$ we get
                $$ \sum_n T_1^{-2} = \frac{s_{n-1}^2}{s_n^2} - 2\left(\frac{1}{T_1T_2}+\cdots+\frac{1}{T_{n-1}T_n}\right)\frac{T_1\cdots T_n}{T_1\cdots T_n}=\frac{s_{n-1}^2}{s_n^2} - 2\frac{s_{n-2}}{s_n}.$$
            Hence $\sum_n T_1^{-2}=\frac{s_{n-1}^2-2s_{n-2}s_n}{s_n^2}$.
            \item The second approach is slightly more involved. We start by multiplying by 1 in a clever (but different) way
                $$\left(\sum_n T_1^{-2}\right)\frac{s_n^2}{s_n^2}=\left(\frac{1}{T_1^2}+\cdots+\frac{1}{T_n^2}\right)\frac{T_1^2\cdots T_n^2}{T_1^2\cdots T_n^2}=\frac{\sum_n T_1^2\cdots T_{n-1}^2}{s_n^2}.$$
            Then $\sum_n T_1^2\cdots T_{n-1}^2$ is obviously (condescending much?) a symmetric polynomial and so we can use our trusty algorithm. Starting with
            \begin{align*}
                s_1^{2-2}s_2^{2-2}\cdots s_{n-1}^{2-0}&=s_{n-1}^2\\
                &=\left(\sum_n T_1\cdots T_{n-1}\right)^2\\
                &=\sum_n T_1^2\cdots T_{n-1}^2 + 2\sum_n T_1^2\cdots T_{n-2}^2T_{n-1}T_n.
            \end{align*}
            Moving to the second term
            \begin{align*}
                s_1^{2-2}\cdots s_{n-2}^{2-1}s_{n-1}^{1-1}s_n^1&=s_{n-2}s_n\\
                &=\left(\sum_n T_1\cdots T_{n-2}\right)T_1\cdots T_n\\
                &=\sum_n T_1^2\cdots T_{n-2}^2 T_{n-1}T_n
            \end{align*}
            and it follows that
            $$\sum_n T_1^2\cdots T_{n-1}^2 = s_{n-1}^2-2s_{n-2}s_n.$$
            So we conclude that
            $$ \sum_n T_1^{-2} = \frac{s_{n-1}^2-2s_{n-2}s_n}{s_n^2}$$
            which is reassuring.
        \end{enumerate}
        Note that in the first approach we stumbled upon something rather interesting:
        $$
            \sum_n T_1^{-1}\cdots T_k^{-1} = \frac{s_{n-k}}{s_n}
        $$
        the proof of which is left as an exercise to the reader.
    \end{sol}

\subsection{Field Extensions}

\subsection{Finite Fields}

\subsection{Separable and Normal Extensions}
    
\end{document}