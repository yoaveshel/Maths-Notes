\documentclass{article}

\usepackage[utf8]{inputenc}
\usepackage{csquotes}
\usepackage[english]{babel}
\usepackage{amsmath,amssymb,amsthm,textcomp}
\usepackage{mathtools}
\usepackage{biblatex}
\usepackage{tikz}
\usepackage{graphics, setspace}
\usepackage{listings}
\usepackage{lipsum}
\usepackage{bookmark}
\usepackage{hyperref}
\hypersetup{
    colorlinks,
    citecolor=black,
    filecolor=black,
    linkcolor=black,
    urlcolor=black
}

\DeclareMathAlphabet{\pazocal}{OMS}{zplm}{m}{n}
\DeclareMathOperator{\Ima}{Im}
\newcommand{\Ba}{\mathcal{B}}
\newcommand{\Ta}{\mathcal{T}}
\newcommand{\Aa}{\mathcal{A}}
\newcommand{\R}{\mathbb{R}}
\newcommand{\C}{\mathbb{C}}
\newcommand{\Z}{\mathbb{Z}}
\newcommand{\N}{\mathbb{N}}
\newcommand{\Q}{\mathbb{Q}}
\newcommand{\p}{\mathbb{P}}
\newcommand{\Ss}{\mathcal{S}} % Schwartz space
\newcommand{\F}{\mathcal{F}} % Fourier Transform
\newcommand{\Rf}{\mathcal{R}} % reflection
\newcommand{\E}{\mathbb{E}}

\DeclarePairedDelimiter\abs{\lvert}{\rvert}%
\DeclarePairedDelimiter\norm{\lVert}{\rVert}%
% Swap the definition of \abs* and \norm*, so that \abs
% and \norm resizes the size of the brackets, and the 
% starred version does not.
\makeatletter
\let\oldabs\abs
\def\abs{\@ifstar{\oldabs}{\oldabs*}}
%
\let\oldnorm\norm
\def\norm{\@ifstar{\oldnorm}{\oldnorm*}}
\makeatother

\newtheorem{theorem}{Theorem}[section]
\newtheorem{corollary}{Corollary}[theorem]
\newtheorem{lemma}[theorem]{Lemma}
\newtheorem*{definition}{Definition}
\newtheorem*{remark}{Remark}

\theoremstyle{remark}
\newtheorem*{sol}{Solution}

\newenvironment{ex}[1]
    {\noindent\large\textbf{Exercise #1}\normalsize\newline}
    {\vspace{0.5 em}}

\title{Complex Analysis - X400386}
\author{Yoav Eshel}
\date{\today}

\begin{document}
    \maketitle
    \tableofcontents
    \newpage

    These notes are based on Complex Variables and Applications by James Ward Brown ($9^\text{th}$ edition).
    \section{Regions in the Complex Plane}
    For $z_0\in\C$, its \textit{neighborhood} is the set
    $$
        \abs{z-z_0}\leq\varepsilon,\quad\varepsilon>0.
    $$
    It consists of all the point that lie inside (but not on!) the circle of radius $\varepsilon$ centered at $z_0$.
    A \textit{deleted neighborhood} is the same set as the neighborhood minus the origin, i.e
    $$
        0<\abs{z-z_0}<\varepsilon.
    $$
    A point $z_0$ is an \textit{interior point} of some set $S$, if there exists a $z_0$-neighborhood that contains only points in $S$.
    It is an \textit{exterior point} if there exists a neighborhood containing no points in $S$.
    If $z_0$ is neither, then it is a \textit{boundary point}.

    A set is \textit{open} if it does not contain any of its boundary points ($\iff$ every point is an interior point) and \textit{closed} if it contains all of them.
    A set can be neither or both. The disk $0<\abs{z}\leq 1$ is neither while the entire complex plane is both.

    An open set $S$ is \textit{connected} if there for any two points in the set there is a finite number of straight lines (polygonal line), all in $S$, that connect them.
    A nonempty open set that is connected is called a \textit{domain} (e.g. every neighborhood is a domain). A connected set that is not open is called a \textit{region}. 

    A set is \textit{bounded} if it can be contained in some circle, otherwise it is \textit{unbounded}.
    
    A point $z_0$ is an \textit{accumulation point} of a set $S$ if every deleted neighborhood of $z_0$ contains at least one point of $S$ (i.e. we can get arbitrarily close to $z_0$ while staying in $S$).
    For example, the point $0$ is an accumulation point of the set $S=\left\{\frac{1}{n}\mid n\in\N\right\}$.
    
    \section{Analytic Functions}
    We say that $f(z)\to w_0$ as $z\to z_0=(x_0,y_0)$ if for every $\varepsilon>0$ there exists $\delta>0$ s.t.
    $$
        0<\abs{z-z_0}<\delta\implies\abs{f(z)-w_0}<\varepsilon\vspace{0.5em}
    $$
    where $\abs{z-z_0}=\abs{x-x_0+(y-y_0)i}=\sqrt{(x-x_0)^2+(y-y_0)^2}$. If such $w_0$ exists then it is unique.

    If $f:\C\to\C$ is given by
    $$
        f(z)=f(x+yi)=u(x,y)+iv(x,y)
    $$
    for $u,v$ real valued and
    $$
        \lim_{(x,y)\to(x_0,y_0)} u(x,y)=u_0,\quad \lim_{(x,y)\to(x_0,y_0)} v(x,y)=v_0
    $$
    exist, then
    $$
        \lim_{(x,y)\to(x_0,y_0)} f(x+yi) = u_0+v_0i.
    $$
    Therefore the limit rules for real valued functions are easily applied to complex functions.

    If we center a unit sphere on the origin of the complex plane, then every point $z$ on the plane is mapped to a unique point $P$ on the surface of the sphere, where $P$ is the intersection of the line from $z$ to the north pole of the sphere. 
    Note that points outside the unit circle are mapped to the northern hemisphere while points inside the unit circle are mapped to the southern hemisphere. Then the unit circle is mapped to the equator and the origin is mapped to the south pole.
    The \textit{point of infinity} is defined as the point that is mapped to the north pole. Then for small $\varepsilon$ the set $\abs{z}>\frac{1}{\varepsilon}$ is called a \textit{neighborhood of $\infty$}.
    Thus we can simply replace the neighborhoods in the definition of a limit by neighborhoods of $\infty$.

    The derivative in complex analysis is defined in the same manner as in usual calculus. In other words, if
    $$
    f'(z):=\lim_{h\to0}\frac{f(z+h)-f(z)}{h}
    $$
    exists then $f$ is differentiable at $z$.

    \subsection{Cauchy-Riemann Equations}
    Let $f:\C\to\C$ given by $f(x+yi)=u(x,y)+v(x,y)i$. Suppose $f$ is differentiable at some $z_0$ and $'f(z_0)=a+bi$.
    Then we can approximate $f$ using
    \begin{align*}
        f(z_0+h)&=f(z_0)+f'(z_0)\cdot h+\mathcal{O}(h)
    \end{align*}
    Let $h=x+yi$. Then
    \begin{align*}
        (a+bi)(x+yi)&=\begin{pmatrix}
            ax-by\\
            ay+bx\\
        \end{pmatrix}\\
        &=\begin{pmatrix}
            a&-b\\
            b&a\\
        \end{pmatrix}
        \begin{pmatrix}
            x\\ y \\
        \end{pmatrix}
    \end{align*}
    and it follows that the matrix $\begin{pmatrix}a&-b\\b&a   \end{pmatrix}$ is the Jacobian of $f$ viewed as a function from $\R^2$ to $\R^2$!
    Hence $\partial_x u = \partial_y v$ and $\partial_y u = -\partial_x v$. This is a necessary condition for a complex function to be differentiable.
    If the partial derivatives exist and are continuous around $z_0$ and  satisfy the relation above then $f'(z_0)$ exists.
    In other words, it means that a complex differentiable function is locally a rotation and a scaling.
    It follows that if $f$ is differentiable then
    $$
        f'(x+yi)=\partial_x u(x,y)+i\partial_x v(x,y)
    $$

    \newpage
    \section{Exercises}
    \subsection{Symmetric Polynomial}
    \begin{ex}{14.10}
        Express the symmetric polynomials $\sum_n T_1^2T_2$ and $\sum_{n} T_1^3T_2$ in the elementary symmetric polynomials.
    \end{ex}
    \begin{sol}
        To get the polynomial $\sum_n T_1^2T_2$ we start with
        $$
            s_1s_2=\sum_n T_1\sum_n T_1T_2 = \sum_n T_1^2T_2+3\sum_n T_1T_2T_3 = \sum_n T_1^2T_2+3s_3
        $$
        Thus 
        $$
            \sum_n T_1^2T_2 = s_1s_2-3s_3
        $$

        Similarly, to transform the polynomial $\sum_{n} T_1^3T_2$ we start with
        \begin{align*}
            s_1^2s_2&=\left(\sum_nT_1\right)^2\sum_nT_1T_2\\
            &=\left(\sum_n T_1^2+2\sum_n T_1T_2\right)\sum_nT_1T_2\\
            &=\sum_nT_1^2\sum_n T_1T_2+2s_2^2\\
            &=\sum_nT_1^3T_2+\sum_n T_1^2T_2T_3+2s_2^2.
        \end{align*}
        And since
        $$
            s_1s_3=\sum_nT_1\sum_nT_1T_2T_3=\sum_nT_1^2T_2T_3+4\sum_n T_1T_2T_3T_4
        $$
        it follows that $\sum_n T_1^2T_2T_3=s_1s_3-4s_4$ and so
        $$
            \sum_{n} T_1^3T_2=s_1^2s_2-s_1s_3+4s_4-2s_2^2
        $$
    \end{sol}
        
    \begin{ex}{14.21}
        Express $p_4=\sum_nT_1^4$ in elementary symmetric polynomials
    \end{ex}
    \begin{sol}
        Let $n\geq 4$. Starting with
        \begin{align*}
            s_1^4 &= \left(\sum_nT_1\right)^4\\& = \sum_n T_1^4+4\sum_n T_1^3T_2+12\sum_n T_1^2T_2T_3+6\sum_nT_1^2T_2^2+24\sum_nT_1T_2T_3T_4.
        \end{align*}
        To understand how to coefficients of the sum are obtained, consider the number of ways the $T_i$ can be arranged. 
        For example, $T_1^4=T_1T_1T_1T_1$ can only be arranged in 1 way but $T_1^2T_2T_3=T_1T_1T_2T_3$ can be arrange in $\frac{4!}{2}=12$ ways (where we divided by 2 since the two $T_1$ can be swapped in any given arrangement).
        Then
        $$
            s_1^2s_2=\left(\sum_n T_1\right)^2s_2=\left(\sum_nT_1^2+2\sum_n T_1T_2\right)s_2 = \sum_n T_1^3T_2+\sum_nT_1^2T_2T_3+2s_2^2.
        $$
        So far we have
        \begin{align*}
            p_4 &= s_1^4-4\left(s_1^2s_2-2s_2^2-\sum_nT_1^2T_2T_3\right)-12\sum_n T_1^2T_2T_3-6\sum_nT_1^2T_2^2-24\sum_nT_1T_2T_3T_4\\
            &=s_1^4-4s_1^2s_2+8s_2^2-24s_4-6\sum_nT_1^2T_2^2-8\sum_n T_1^2T_2T_3.
        \end{align*}
        So continuing with $\sum_nT_1^2T_2^2$ we get
        $$
            s_2^2 = \left(\sum_n T_1T_2\right)^2=\sum_n T_1^2T_2^2+2\sum_n T_1^2 T_2T_3+6\sum_n T_1T_2T_3T_4.
        $$
        Finding the coefficients here is slightly trickier since $s_2$ contains pairs not all arrangements are allowed. 
        For example, $T_1^2T_2^2$ can only come from the pair $T_1T_2$. On the other hand $T_1T_2T_3T_4$ can come from $T_1T_2$ and $T_3T_4$ or $T_1T_4$ and $T_2T_3$ and so on.
        We choose the first pair (${4\choose 2}=6$ ways) which also fixes the second pair and so there are 6 ways to get $T_1T_2T_3T_4$.
        Hence
        \begin{align*}
            p_4 &= s_1^4-4s_1^2s_2+8s_2^2-24s_4-6\left(s_2^2-2\sum_nT_1^2T_2T_3-6s_4\right)-8\sum_n T_1^2T_2T_3\\
            &=s_1^4-4s_1^2s_2+2s_2^2+12s_4+4\sum_n T_1^2T_2T_3.
        \end{align*}
        Using Exercise 14.10 we get
        \begin{align*}
            p_4 &=s_1^4-4s_1^2s_2+2s_2^2+12s_4+4(s_1s_3-4s_4)\\
            &=s_1^4-4s_1^2s_2+2s_2^2-4s_4+4s_1s_3
        \end{align*}
    \end{sol}

    \begin{ex}{14.22}
        A rational function $f\in\Q[T_1,\dots,T_n]$ is called symmetric if it is invariant under all permutations of the variables $T_i$. Prove that every symmetric rational function is a rational function in the elementary symmetric functions.
    \end{ex}
    \begin{proof}
        Let $f\in\Q[T_1,\dots,T_n]$ be a symmetric rational function. 
        Then $f=g/h$ for $g,h$ polynomials. If $h$ is a symmetric polynomial then $g=fh$ is symmetric as well.
        By the fundamental theorem of symmetric polynomial both $g$ and $h$ can be written in terms of elementary symmetric polynomials and we're done.
        If $h$ is not symmetric, then let 
        $$\tilde{h}=\prod_{\sigma\in S_n\setminus\{e\}}\sigma(h)$$
        and then $h\tilde{h}$ is symmetric so $f=\frac{g\tilde{h}}{h\tilde{h}}$ which is again the case above.
    \end{proof}

    \begin{ex}{14.23}
        Write $\sum_{n}T_1^{-1}$ and $\sum_n T_1^{-2}$ as rational functions in $\Q[s_1,\dots,s_n]$
    \end{ex}
    \begin{sol}
        Starting with
        $$
            \sum_{n}T_1^{-1}=\frac{1}{T_1}+\cdots+\frac{1}{T_n}.
        $$
        We multiply by $1=\frac{s_n}{s_n}$ and simplify
        \begin{align*}
            \frac{s_n}{s_n}\sum_{n}T_1^{-1}&=\frac{T_1T_2\cdots T_n}{T_1T_2\cdots T_n}\left(\frac{1}{T_1}+\cdots+\frac{1}{T_n}\right)\\
            &=\frac{s_{n-1}}{s_n}
        \end{align*}

        For the second expression we present to approaches.
        \begin{enumerate}
            \item Observing that 
                $$\left(\sum_n T_1^{-1}\right)^2=\sum_{n} T_1^{-2}+2\sum_{n}T_1^{-1}T_2^{-1}$$
            we can write using the previous part
                $$ \sum_n T_1^{-2} = \frac{s_{n-1}^2}{s_n^2}-2\sum_{n}T_1^{-1}T_2^{-1}$$
            and multiplying by the second term by $\frac{s_{n}}{s_{n}}$ we get
                $$ \sum_n T_1^{-2} = \frac{s_{n-1}^2}{s_n^2} - 2\left(\frac{1}{T_1T_2}+\cdots+\frac{1}{T_{n-1}T_n}\right)\frac{T_1\cdots T_n}{T_1\cdots T_n}=\frac{s_{n-1}^2}{s_n^2} - 2\frac{s_{n-2}}{s_n}.$$
            Hence $\sum_n T_1^{-2}=\frac{s_{n-1}^2-2s_{n-2}s_n}{s_n^2}$.
            \item The second approach is slightly more involved. We start by multiplying by 1 in a clever (but different) way
                $$\left(\sum_n T_1^{-2}\right)\frac{s_n^2}{s_n^2}=\left(\frac{1}{T_1^2}+\cdots+\frac{1}{T_n^2}\right)\frac{T_1^2\cdots T_n^2}{T_1^2\cdots T_n^2}=\frac{\sum_n T_1^2\cdots T_{n-1}^2}{s_n^2}.$$
            Then $\sum_n T_1^2\cdots T_{n-1}^2$ is obviously (condescending much?) a symmetric polynomial and so we can use our trusty algorithm. Starting with
            \begin{align*}
                s_1^{2-2}s_2^{2-2}\cdots s_{n-1}^{2-0}&=s_{n-1}^2\\
                &=\left(\sum_n T_1\cdots T_{n-1}\right)^2\\
                &=\sum_n T_1^2\cdots T_{n-1}^2 + 2\sum_n T_1^2\cdots T_{n-2}^2T_{n-1}T_n.
            \end{align*}
            Moving to the second term
            \begin{align*}
                s_1^{2-2}\cdots s_{n-2}^{2-1}s_{n-1}^{1-1}s_n^1&=s_{n-2}s_n\\
                &=\left(\sum_n T_1\cdots T_{n-2}\right)T_1\cdots T_n\\
                &=\sum_n T_1^2\cdots T_{n-2}^2 T_{n-1}T_n
            \end{align*}
            and it follows that
            $$\sum_n T_1^2\cdots T_{n-1}^2 = s_{n-1}^2-2s_{n-2}s_n.$$
            So we conclude that
            $$ \sum_n T_1^{-2} = \frac{s_{n-1}^2-2s_{n-2}s_n}{s_n^2}$$
            which is reassuring.
        \end{enumerate}
        Note that in the first approach we stumbled upon something rather interesting:
        $$
            \sum_n T_1^{-1}\cdots T_k^{-1} = \frac{s_{n-k}}{s_n}
        $$
        the proof of which is left as an exercise to the reader.
    \end{sol}

\subsection{Field Extensions}

\subsection{Finite Fields}

\subsection{Separable and Normal Extensions}

\end{document}